\begin{abstract} 

This document describes a sizable grammar of English written in the
TAG formalism and implemented for use with the XTAG system. This
report and the grammar described herein supersedes the TAG grammar
described in \cite{tech-rept95}. The English grammar described in this
report is based on the TAG formalism developed in \cite{joshi75},
which has been extended to include lexicalization (\cite{schabes88}),
and unification-based feature structures (\cite{vijay91}).  The range
of syntactic phenomena that can be handled is large and includes
auxiliaries (including inversion), copula, raising and small clause
constructions, topicalization, relative clauses, infinitives, gerunds,
passives, adjuncts, it-clefts, wh-clefts, PRO constructions, noun-noun
modifications, extraposition, determiner sequences, genitives,
negation, noun-verb contractions, sentential adjuncts and imperatives.
The XTAG grammar is continuously updated with the addition of new analyses
and modification of old ones, and an online version of this report can
be found at the XTAG web page: {\tt http://www.cis.upenn.edu/\~{}xtag/}.

\end{abstract}

