\section{Introduction}
\subsection{TAG formalism}
The English grammar described in this report is based on the TAG formalism developed in
Joshi Levy, Takahashi\shortcite{josh75}. As first shown by Joshi (1985) and Kroch and Joshi (1985), the properties of TAGs permit us to encapsulate diverse syntactic phenomena in a very natural way. TAG's extended domain of locality and its factoring of recursion from local dependencies lead, among other things, to a localization of so-called unbounded dependencies.
The primitive elements of the TAG formalism are known as {\sc
elementary trees}.  Elementary trees are of two types: {\sc initial
trees} and {\sc auxiliary trees}.  In describing natural language,
{\sc initial trees} are phrase structure trees of simple sentences
containing no recursion, while recursive structures are represented by
{\sc auxiliary trees} as illustrated in Figure~\ref{fig1}.  Initial
trees are defined as having exactly one terminal node on their
frontier, the anchor (indicated by $\beta$, and as having all ohter nodes on the frontier
marked as substitution sitesNodes on
the frontier of an {\sc initial tree} would be marked by a
($\downarrow$), to indicate a site for substitution while exactly one
node of an {\sc auxiliary tree} will be marked by an ($\ast$), to
indicate a foot node. Further, the foot node has the same label as the
root node of the tree. The trees define the domain of locality over
which constraints are specified.


 The trees in I  are called initial trees. Initial trees represent minimal linguistic structures which are defined to have at least one terminal at the frontier (the anchor) and to have all non-terminal nodes at the frontier  filled by substitution.  An initial tree is called an X-type initial tree if its root is labeled with type X.  All basic categories or constituents which serve as arguments to more complex initial or auxiliary trees are X-type initial trees. A particular case is the S-type initial trees (e.g. the left tree in Figure~\ref{elt-tree}). They are rooted in S, and it is a requirement of the grammar that a valid input string has to be derived from at least one S-type initial tree. The trees in A  are called {\bf auxiliary trees}.They can represent constituents that are adjuncts to basic structures(e.g. adverbials). They can also represent basic sentential structures corresponding to verbs or predicates taking sentential complements.Auxiliary trees (e.g. the right tree in Figure~3) are characterized as follows:\begin{list}{$\bullet$}{\setlength{\topsep}{0in} \setlength{\itemsep}{0in}}\item internal nodes are labeled by non-terminals;\item leaf nodes are labeled by terminals or by  non-terminal nodesfilled by substitution except for exactly one node (called the {\bf footnode}) labeled by a non-terminal on which only adjunction can apply;furthermore the label of the foot node is the same as the label of theroot node.\footnote{A null adjunction constraint (NA) is putsystematically on the footnode of an auxiliary tree. This disallowsadjunction of a tree on the footnode.}\end{list}The {\bf tree set} of a TAG $G$, ${\cal T}(G)$ is defined to be theset of all derived trees starting from S-type initial trees in $I$ whosefrontier consists of terminal nodes (all substitution nodes having beenfilled). The {\bf string language} generated by a TAG, ${\cal L}(G)$,is defined to be the set of all terminal strings on the frontier of thetrees in ${\cal T}(G)$.

\subsection{Lexicalization}
Most current linguistic theories give lexical accounts of several
phenomena that used to be considered purely syntactic. The information
put in the lexicon is thereby increased in both amount and complexity:
see, for example, lexical rules in LFG (Bresnan and Kaplan, 1983)\nocite{bk83}, GPSG
(Gazdar, Klein, Pullum and Sag, 1985)\nocite{gkps85}, HPSG (Pollard
and Sag, 1987)\nocite{ps87}, Combinatory Categorial Grammars (Steedman
1985, 1988\nocite{st88}), Karttunen's version of Categorial Grammar
\nocite{kart86}(Karttunen 1986, 1988), some versions of GB theory
(Chomsky 1981\nocite{c81}), and Lexicon-Grammars (Gross 1984).
\nocite{gross84}

Following Schabes, Abeill� and Joshi (1988) we say that a grammar is
�	`lexicalized' if it consists of:\footnote{By `lexicalization' we mean
that in each structure there is a lexical item that is realized. We do
not mean simply adding feature structures (such as head) and unification
equations to the rules of the formalism.}


\begin{list}{$\bullet$}{\setlength{\topsep}{0in} \setlength{\itemsep}{0in}}
\item a finite
set of structures each associated with a lexical item; each lexical
item will be called the {\it anchor}\footnote{In previous
publications, the term `head' was used instead of the term `anchor'.
From now on, we will use the term anchor instead; the term `head'
introduces some confusion because the lexical items which are the
source of the elementary trees may not necessarily be the same as the
traditional syntactic head of those structures. In fact, the notion of
anchor is in some ways closer to the notion of function in Categorial Grammars.} of
the corresponding structure; the structures define the domain of
locality over which constraints are specified; constraints are local
with respect to their anchor;
\item an operation or operations for composing
the structures.
\end{list}

\vspace{0.5cm}

`Lexicalized' grammars (Schabes, Abeill\'{e} and Joshi,1988)\nocite{saj88}, systematically associate each elementary structure with a lexical anchor. These elementary structures specify extended domains of locality (as compared to CFGs) over which constraints can be stated. The `grammar', consists of a lexicon where each lexical item is associated with a finite number of structures for which that item is the anchor. There are no separate grammar rules at this level of description, although there are, of course, `rules'which tell us how these structures are combined.  In general, this isthe level of description that we will be describing in this paper. Ata higher level of description, the grammar rules and principles that are implicit in the form of the lexicon would be stated explicitly. For example, there are principles which govern which trees are groupedtogether into tree families, and rules which describe the relations between structure types across tree families (see subsection 2.1for a discussion of tree families.) The information explicitly provided in this more abstract representation of the grammar may bethought of as an interpretation of the data in the lower-level representation.\footnote{There may also be some linguistic generalizations which can not be stated as an explicit representation of implicit lexical information.  These would likely be statements concerning the constraints on syntactic structures that are needed to evaluate the acceptability of a new lexical item or structure.  We have not yet needed to account for any statements of this kind.}

Not every grammar is in a `lexicalized' form.\footnote{Notice the similarity of the definition of `lexicalized' grammar with the off line parsibility constraint (Kaplan and Bresnan 1983)\nocite{kb83}. As consequences of our definition, each structure has at least one lexical item (its anchor) attached to it and all sentences are finitely ambiguous.} In the process of lexicalizing a grammar,the `lexicalized' grammar is required to produce not only the same language as the original grammar, but also the same structures (or tree set).

For example, a CFG, in general, will not be in a `lexicalized' form.The domain of locality of CFGs can be easily extended by using a tree rewriting grammar (Schabes, Abeill\'{e} and Joshi, 1988) that uses only substitution as a combining operation. This tree rewriting grammar consists of a set of trees that are not restricted to be of depth one(as in CFGs). Substitution can take place only on non-terminal nodes ofthe frontier of each tree. Substitution replaces a node marked forsubstitution by a tree rooted by the same label as the node (seeFigure~\ref{operations}; the substitution node is marked by a down arrow$\downarrow$).

However, in the general case, CFGs cannot be `lexicalized', if only substitution is used (for further explanation why, the reader isreferred to Schabes, Abeill\'{e} and Joshi, 1988).Furthermore, in general, there is not enough freedom to choose the anchor of each structure. This is important because we want the choiceof the anchor for a given structure to be determined on purely linguistic grounds.

If adjunction is used as an additional operation to combine these structures, CFGs can be lexicalized. Adjunction builds a new tree from an auxiliary tree $\beta$ and a tree $\alpha$. It inserts anauxiliary tree into another tree (see Figure~\ref{operations}).Adjunction is more powerful than substitution. It can weakly simulate substitution, but it also generates languages that could not be generatedwith substitution.\footnote{It is also possible to encode a context-free grammar with auxiliary trees using adjunction only. However, although the languages correspond, the set of trees do not correspond.}

\begin{figure}[htb]
\begin{tabular}{|ccc||ccc|}
\hline
\multicolumn{3}{|c||}{\psfig{figure=figures/subst.text,width=3.0in}} &
\multicolumn{3}{|c||}{\psfig{figure=figures/adjunction.text,width=3.0in}}
& & & & &
\multicolumn{3}{|c||}{{\it Substitution}} &\multicolumn{3}{c|}{{\it
Adjunction}} 
\hline
\hline
\psfig{figure=figures/loved.fig,height=2.8cm} &
\psfig{figure=figures/woman.fig,height=1.95cm} 
\raisebox{1.0cm}{$\longrightarrow$} &
\psfig{figure=figures/loved-woman.fig,height=3.4cm}&
\psfig{figure=figures/loved.fig,height=2.8cm} &
\psfig{figure=figures/has-BvVX.fig,height=1.95cm} 
\raisebox{2.0cm}{$\longrightarrow$} &
\psfig{figure=figures/has-loved.fig,height=3.4cm}
& & & & &
\multicolumn{3}{|c||}{{\it Example of Substitution}}
&\multicolumn{3}{c|}{{\it Example of Adjunction}}  
\hline
\end{tabular}
\caption{{\it Combining Operations}}
\label{operations}
\end{figure}

% 
% \begin{figure}[htb]
% \begin{tabular}{|c||c|}
% \hline
% \psfig{figure=figures/subst.text,width=3.0in} &
% \psfig{figure=figures/adjunction.text,width=3.0in}
% & 
% {\it Substitution} & {\it Adjunction}
% \hline
% \end{tabular}
% \caption{{\it Combining operations}}
% \label{operations}
% \end{figure}
% 
% 							
% \begin{figure}[htb]
% \begin{tabular}{|ccc||ccc|}
% \hline
% \psfig{figure=figures/loved.fig,height=2.8cm} &
% \psfig{figure=figures/woman.fig,height=1.95cm} 
% \raisebox{1.0cm}{$\longrightarrow$} &
% \psfig{figure=figures/loved-woman.fig,height=3.4cm}&
% \psfig{figure=figures/loved.fig,height=2.8cm} &
% \psfig{figure=figures/has-BvVX.fig,height=1.95cm} 
% \raisebox{2.0cm}{$\longrightarrow$} &
% \psfig{figure=figures/has-loved.fig,height=3.4cm}
% & & & & &
% \multicolumn{3}{|c||}{{\it Substitution}} &\multicolumn{3}{c|}{{\it Adjunction}} 
% \hline
% \end{tabular}
% \caption{{\it Examples of Substitution and Adjunction}}
% \label{examples}
% \end{figure}


Substitution and adjunction enable us to lexicalize CFGs. The anchors can be  chosen on purely linguistic grounds (Schabes, Abeill\'{e} and Joshi, 1988).  The resulting system now falls in the class of mildly context-sensitive languages (Joshi, 1985, and Joshi, Vijayshanker and Weir,1990)\nocite{j83,jvw90}. Elementary structures of extended domain of locality combined with substitution and adjunction yield LexicalizedTAGs (LTAGs).

TAGs were first introduced by Joshi, Levy and Takahashi(1975)\nocite{jlt75} and Joshi (1985)\nocite{j83}.For more details on the original definition of TAGs, we refer the reader to Joshi (1985), Kroch and Joshi (1985)\nocite{kj85}, or Vijay-Shanker(1987)\nocite{v87}.It is known that Tree Adjoining Languages (TALs) are mildly context sensitive.TALs properly contain context-free languages.

TAGs with substitution and adjunction are naturally lexicalized becausethey use an extended domain of locality.\footnote{In some earlier workof Joshi (1969, 1973)\nocite{joshi69}\nocite{joshi73}, the use of the two operations `adjoining' and `replacement' (a restricted case of substitution) was investigated both mathematically and linguistically.  However, these investigations dealt with string rewriting systems and not tree rewriting systems.} A Lexicalized Tree Adjoining Grammar is a tree-based system that consists of two finite sets of trees: a set of initial trees, $I$ and a set of auxiliary trees $A$ (seeFigure~\ref{elt-tree}). The trees in $I \cup A$ are called {\bfelementary trees}. Each elementary tree is constrained to have at least  one terminal symbol which acts as its anchor.
 \begin{figure}[htb]
\begin{center}
\begin{tabular}{|c|}
\hline
\psfig{figure=figures/lex-elementary.fig,height=3.5cm}

\hline
\end{tabular}
\end{center}
 \caption{{\it Schematic initial and auxiliary trees}}
 \label{elt-tree}
 \end{figure}


The trees in $I$ are called {\bf initialtrees}. Initial trees represent minimal linguistic structures which aredefined to have at least one terminal at the frontier (the anchor)and to have all non-terminal nodes at the frontier  filled by substitution.An initial tree is called an X-type initial tree if its root is labeled with type X.All basic categories or constituents which serve as arguments to more complex initial or auxiliary trees are X-type initial trees. A particular case is the S-type initialtrees (e.g. the left tree in Figure~\ref{elt-tree}). They are rooted in $S$, and it is a requirement of the grammar that a valid input string has to be derived from at least one S-type initial tree. 


The trees in $A$ are called {\bf auxiliary trees}.They can represent constituents that are adjuncts to basic structures(e.g. adverbials). They can also represent basic sentential structures corresponding to verbs or predicates taking sentential complements.Auxiliary trees (e.g. the right tree in Figure~3) are characterized as follows:
\begin{list}{$\bullet$}{\setlength{\topsep}{0in} \setlength{\itemsep}{0in}}
\item internal nodes are labeled by non-terminals;
\item leaf nodes are labeled by terminals or by  non-terminal nodes
filled by substitution except for exactly one node (called the {\bf foot
node}) labeled by a non-terminal on which only adjunction can apply;
furthermore the label of the foot node is the same as the label of the
root node.\footnote{A null adjunction constraint (NA) is put
systematically on the footnode of an auxiliary tree. This disallows
adjunction of a tree on the footnode.}
\end{list}


The {\bf tree set} of a TAG $G$, ${\cal T}(G)$ is defined to be the set of all derived trees starting from S-type initial trees in $I$ whose frontier consists of terminal nodes (all substitution nodes having been filled). The {\bf string language} generated by a TAG, ${\cal L}(G)$, is defined to be the set of all terminal strings on the frontier of the trees in ${\cal T}(G)$.
