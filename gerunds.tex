\chapter{Gerund NP's}
\label{gerunds-chapter}

The puzzle over gerunds in the linguistics literature has been that they seem
to have both NP and clausal properties. That is to say they seem to have
certain clausal properties but occur in positions typically occupied by noun
phrases. The bold face portions of examples (\ex{1})-(\ex{3}) show examples of
gerunds as subjects in (\ex{1}) and (\ex{2}), and as the object of a
preposition in (\ex{3}).


\enumsentence{And {\bf avoiding such losses} will take a monumental
effort. (WSJ)}
\enumsentence{{\bf Mr. Nolen's nocturnal wandering} doesn't make him a
weirdo. (WSJ)}
\enumsentence{Is this a case where private markets are approving of
{\bf Washington's bashing of Wall Street}? (WSJ)}


In the English XTAG grammar we adopt a position similar to that of
\cite{Rosenbaum67} and \cite{Emonds70} - that gerunds are NP's exhaustively 
dominating a clause.  In particular, we found that any place an NP is allowed,
a gerundive clause is also allowed, and no cases in which a verb
subcategorized for gerundive clauses, but not NP's.  

\begin{figure}[htb]
\centering
\begin{tabular}{cc}
{\psfig{figure=ps/gerund-files/alphaDnx0Vnx1.ps,height=3.2in}}&
{\psfig{figure=ps/gerund-files/alphaGnx0Vnx1.ps,height=3.2in}}
\\
(a)&(b)\\
\end{tabular}
\caption{Gerund trees from the transitive tree family: $\alpha$Dnx0Vnx1 (a) and
$\alpha$Gnx0Vnx (b)}
\label{gerund-trees}
\label{2;12,1}
\label{2;13,1}
\end{figure}

Our implementation includes at least two gerundive trees in each tree family
(see Figure~\ref{gerund-trees}).  The gerund trees in a tree family have
basically the form of the declarative for that family but have NP as the
category of their top node.  The Determiner Gerund tree in
Figure~\ref{gerund-trees}(a) has an initial DetP and instantiates the direct
object as a PP. It is used for gerunds such as the one in bold face in
sentence~(\ex{1}).

\enumsentence{Some think {\bf the rapid selling of bonds} has a way to go.}

Notice that the modification of {\it selling of bonds} by the adjective {\it
rapid} supports the choice of N as the label for the node dominating V and
PP$_{1}$.

The NP gerund tree in Figure~\ref{gerund-trees}(b) has exactly the same
structure as the declarative transitive tree except for the root node label and
for feature values.  In particular, the verb is required to be {\bf
$<$mode$>$=ger}, and the subject is required to be {\bf
$<$case$>$=acc/none/gen}, i.e. either an accusative, PRO or genitive NP. The
whole NP formed by the gerund can itself have either nominative or accusative
case. The NP gerund tree is used for gerunds such as the one in bold face in
sentence~(\ex{-3}) and (\ex{-2}).

One question that arises with respect to gerunds is whether there is anything
special about their distribution as compared to other types of NP's.  In fact,
it appears that gerund NP's can occur in any NP position.  Some verbs might not
seem to be very accepting of gerund NP arguments, as in (\ex{1}), but we
believe this to be a semantic incompatibility rather than a syntactic problem
since the same structures are fine with other lexical items.

\enumsentence{?[$_{NP}$John's repairing$_{NP}$] ran.}
\enumsentence{[$_{NP}$John's tinkering$_{NP}$] worked.}

By having the root node of gerund trees be NP, the gerunds have the
same distribution as any other NP in the English XTAG grammar without
doing anything exceptional. The clause structure is captured by the
form of the trees and by inclusion in the tree families.
