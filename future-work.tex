\appendix
\chapter{Future Work}
\label{future-work}

\section{Adjective ordering}

At this point, the treatment of adjectives in the XTAG English grammar does not
include selectional or ordering restrictions.\footnote{This section is a repeat
of information found in section~\ref{adj-modifier}.} Consequently, any
adjective can adjoin onto any noun and on top of any other adjective already
modifying a noun. All of the modified noun phrases shown in (\ex{1})-(\ex{4})
currently parse.

\enumsentence{big green bugs}
\enumsentence{big green ideas}
\enumsentence{colorless green ideas}
\enumsentence{$\ast$green big ideas}

While (\ex{-2})-(\ex{0}) are all semantically anomalous, (\ex{0}) also suffers
from an ordering problem that makes it seem ungrammatical as well.  Since the
XTAG grammar focuses on syntactic constructions, it should accept
(\ex{-3})-(\ex{-1}) but not (\ex{0}).  Both the auxiliary and determiner
ordering systems are structured on the idea that certain types of lexical items
(specified by features) can adjoin onto some types of lexical items, but not
others.  We believe that an analysis of adjectival ordering would follow the
same type of mechanism.



\section{Determiner Adverbs}

There are some apparent adverbs that interact with the NP and determiner system
(\cite{quirk85}), although there is some debate in the literature as to whether
these should be classified as determiners or adverbs.\footnote{This section is
from \cite{HockeyEgedi94}.}

\enumsentence{{\bf Hardly} any attempt was made at restitution.}
\enumsentence{{\bf Only} Albert would say such a thing.}
\enumsentence{{\bf Almost} all the people had left by 5pm.}

Adverbs that modify NP's or determiners have restrictions on what types of NP's
or determiners they can modify. They divide into three classes based on the
pattern of these restrictions.  The adverbs {\it especially}, {\it even}, {\it
just} and {\it only} form a class that can modify any NP that is {\bf
$<$wh$>$=--}, including proper nouns.  A second class, consisting of adverbs
such as {\it hardly}, {\it merely} and {\it simply}, modifies NP's with
determiners that are {\bf $<$definite$>$=--} and {\bf $<$const$>$=+}, or that
are {\bf $<$gen$>$=+}.  This second class of adverbs can also modify NP's with
{\it the} as a determiner, but they do not modify NP's without determiners.
The third class, exemplified by {\it almost}, {\it approximately} and {\it
relatively}, modifies the determiner itself.  These adverbs are restricted to
modifying determiners with the {\bf $<$card$>$=+} feature, as well as {\it
all}, {\it double} and {\it half}.  The distinction between adverbs that modify
NP's and ones that modify determiners can be seen in the NP's in
({\ex{1}})~and~({\ex{2}}).

\enumsentence{[Just][half the people]}
\enumsentence{[Approximately half][the people]}




\section{More work on Determiners}

In addition to the analysis described in Chapter~\ref{det-comparitives}, there
remains work to be done to complete the analysis of determiner constructions in
English.\footnote{This section is from \cite{HockeyEgedi94}.}  Although
constructions such as determiner coordination are easily handled if
overgeneration is allowed, blocking sequences such as {\it one and some} while
allowing sequences such as {\it five or ten} still remains to be worked out.
There are still a handful of determiners that are not currently handled by our
system.  We do not have an analysis to handle {\it most}, {\it such}, {\it
certain}, {\it other} and {\it own}\footnote{The behavior of {\it own} is
sufficiently unlike other determiners that it most likely needs a tree of its
own, adjoining onto the right-hand side of genitive determiners.}.  In
addition, there is a set of lexical items that we consider adjectives ({\it
enough}, {\it less}, {\it more} and {\it much}) that have the property that
they cannot cooccur with determiners.  We feel that a complete analysis of
determiners should be able to account for this phenomenon, as well.




\section{Comparatives}

Also included in our future grammar development plans are comparatives.
Comparatives that involve ellipsis would require a general solution of the
problem of representing ellipsis, but simpler comparatives without ellipsis,
such as {\it fewer than nine\/} in (\ex{1}), should be amenable to analysis as
complex determiners, perhaps with trees similar in construction to the
partitive and genitive NP trees.

\enumsentence{Cats have {\bf fewer than} nine lives.}



\section{Time NP's}

Although in general NP's cannot simply adjoin onto sentences, there is a class
of NP's, called Time NP's, that can.  These NP's behave essentially like PP's,
and the XTAG analysis for this is fairly simple, requiring only the creation of
proper NP auxiliary trees.  Only slightly more difficult is the identification
of all possible anchors of these trees.  A {\bf $<$time$>$=+} feature will be
used to ensure that only certain nouns can select the time NP auxiliary trees.

\enumsentence{I went to Kentucky last month/$\ast$big cat.}
\enumsentence{This morning/$\ast$Big cat, we practiced juggling four balls.}



\section{{\it -ing} adjectives}

An analysis has already been provided for {\it -ed} adjectives (as in sentence~
(\ex{1})), which are restricted to the Transitive Verb family.\footnote{This
analysis may need to be extended to the Transitive Verb particle family as
well.}  A similar analysis needs to take place for the \nolinebreak[4]{\it
-ing} adjectives.  This type of adjective, however, does not seem to be as
restricted as the \nolinebreak[4]{\it -ed} adjectives, since verbs in other
tree families seem to exhibit this alternation as well (e.g. sentences~(\ex{2})
and (\ex{3})).

\enumsentence{The murdered man was a doctoral student at UPenn.}
\enumsentence{The man died.}
\enumsentence{The dying man pleaded for his life.}



\section{Punctuation}

We are currently developing an analysis of comma coordination, to
cover sentences such as (\ex{1}) and (\ex{2}).

\enumsentence{The brilliant, funny and timeless comedian had his 99th birthday
today.}
\enumsentence{NP's, PP's and VP's are all adjunction sites.}

Beyond this, we intend to add an analysis of other types of punctuation. We
believe that there are cases where the punctuation will serve as a guide to the
correct parse (e.g. comma following topicalized element), thus reducing
ambiguity.


\section{PRO control}

Within the FB-LTAG formalism, PRO-control is an interesting problem because of
the intrinsic non-local nature of control.\footnote{This section is taken from
\cite{bhatt94}.}  The controller NP and the controlled PRO are always in
different clauses.  In this sense, Control is even more non-local than Binding.

In the literature on Control, two types are often distinguished: obligatory
control, as in sentences~(\ex{1}) and (\ex{2}), and optional control, as in
sentence~(\ex{3}).

\enumsentence{Jan$_i$ promised Maria [PRO$_i$ to go].}
\enumsentence{Jan persuaded Maria$_i$ [PRO$_i$ to go].}
\enumsentence{[PRO$_{arb}$ to dance] is important.}

An analysis for obligatory control has been worked out, although it has yet to
be implemented.  The NP anchored by PRO will have the feature {\bf
$<$control$>$=+}.  The {\bf $<$control$>$} feature is also introduced in trees
that can take sentential arguments.  Depending on the verb, the control
propagation paths in the auxiliary trees are different.  In the case of subject
control (as in sentence~(\ex{-2})), the subject NP and the foot node are
constrained to have the same control features, while for object control
(e.g. sentence~(\ex{-1})), the object NP and the foot node are constrained to
have the same control features. 

Work has also been done on an XTAG analysis for optional control, but this has
not been fully worked out yet.




\section{Verb selectional restrictions}

Although we explicitly do not want to model semantics in the XTAG grammar,
there is some work along the syntax/semantics interface that would help reduce
syntactic ambiguity and thus decrease the number of semantically anomalous
parses.  In particular, verb selectional restrictions, particularly for PP
arguments and adjuncts, would be quite useful.  With the exception of the
required {\it to} in the Ditransitive with PP Shift tree family (Tnx0Vnx1Pnx2),
any preposition is allowed in the tree families that have prepositions as their
arguments.  In addition, there are no restrictions as to which prepositions are
allowed to adjoin onto a given verb.  The sentences in (\ex{1})-(\ex{3}) are
all currently accepted by the XTAG grammar.  Their violations are stronger than
would be expected from purely semantic violations, however, and the presence of
verb selectional restrictions on PP's would keep these sentences from being
accepted.

\enumsentence{\#Survivors walked of the street.}
\enumsentence{\#The man about the earthquake survived.}
\enumsentence{\#The president arranged on a meeting.}




\section{Idioms}

An analysis of idioms has already been worked out (\cite{AS89}), and one idiom
tree family is contained in the English XTAG grammar (Transitive Idioms;
section~\ref{nx0Vdn1-family}).  What remains to be done is a wide-ranging
cataloging of the many English idioms.  The list of idioms must then be divided
into appropriate tree families, based on the construction of the idiom, the
elements that are frozen and must therefore anchor the trees, and the
alternations that the idiom can undergo (e.g. passive, wh-movement, etc).  This
work has not been done.
