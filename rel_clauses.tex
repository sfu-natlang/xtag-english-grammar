\chapter{Relative Clauses}
\label{rel_clauses}

Relative clauses are NP modifiers, which involve extraction of
an argument or an adjunct. The NP head (the
portion of the NP being modified by the relative clause) is 
not directly related to the extracted element. 
For example in (\ex{1}), {\it the person} is the head NP
and is modified by the relative clause {\it whose mother $\epsilon$ 
likes Chris}. {\em The person} is not interpreted as the subject of the
relative clause which is missing an overt subject. In other cases, such
as (\ex{2}), the relationship between the head NP {\em export exhibitions}
may seem to be more direct but even there we assume that there are two
independent relationships: one between the entire relative clause
and the NP it modifies, and another between the extracted element
and its trace. The extracted element may be an overt {\em wh}-phrase
as in (\ex{1}) or a covert element as in (\ex{2}). 

\enumsentence{the person whose mother likes Chris}
\enumsentence{export exhibitions that included high-tech items}

Relative clauses are represented in the English XTAG grammar by auxiliary trees
that adjoin to NP's. These trees are anchored by the verb in the clause and
appear in the appropriate tree families for the various verb
subcategorizations. Within a tree family there will be groups of relative
clause trees based on the declarative tree and each passive tree. Within each
of these groups, there is a separate relative clause tree corresponding to each
possible argument that can be extracted from the clause. There is
no  relationship between the extracted position and the head NP.
The relationship between the relative clause and the head NP is treated
as a semantic relationship which will be provided by any reasonable
compositional theory. The relationship between the extracted element
(which can be covert) is captured by co-indexing the
{\bf $<$trace$>$} features of the extracted NP and the NP$_{w}$ node in the
relative clause tree. If for example, it is {\bf NP$_{0}$} that is extracted,
we have the following feature equations:\\
{\bf NP$_{w}$.t:$\langle$ trace $\rangle =$NP$_{0}$.t:$\langle$ trace $\rangle$}\\
{\bf NP$_{w}$.t:$\langle$ case $\rangle =$NP$_{0}$.t:$\langle$ case $\rangle$}\\
{\bf NP$_{w}$.t:$\langle$ agr $\rangle =$NP$_{0}$.t:$\langle$ agr $\rangle$}
\footnote{
No adjunct traces are represented in the XTAG analysis of adjunct extraction.
Relative clauses on adjuncts do not have traces and consequently feature
equations of the kind shown here are not present.}

Representative examples from the transitive tree family
are shown with a relevant subset of their features in
Figures~\ref{trans-rel-clause-trees}(a) and \ref{trans-rel-clause-trees}(b).
Figure~\ref{trans-rel-clause-trees}(a) involves a relative clause with a 
covert extracted element, while figure~\ref{trans-rel-clause-trees}(b)
involves a relative clause with an overt {\em wh}-phrase.\footnote{
The convention followed in naming relative clause trees is outlined
in Appendix~\ref{tree-naming}.}

\begin{figure}[htb]
\begin{tabular}{cc}
\psfig{figure=ps/rel_clauses-files/NbetaNc1nx0Vnx1.ps,height=10.0cm}&
\psfig{figure=ps/rel_clauses-files/NbetaN0nx0Vnx1.ps,height=10.0cm}\\
(a)&(b)
\end{tabular}
\caption{Relative clause trees in the transitive tree family: $\beta$Nc1nx0Vnx1
(a) and $\beta$N0nx0Vnx1 (b)}
\label{trans-rel-clause-trees}
\label{2;16,1}
\label{2;15,1}
\end{figure}

The above analysis is essentially identical to the GB analysis of 
relative clauses. One aspect of its implementation is that 
an covert {\bf $+<$wh$>$} NP and a covert Comp have to be introduced.
See  (\ex{1}) and (\ex{2}) for example.

\enumsentence{export exhibitions [ [$_{NP_{w}}$$\epsilon$]$_{i}$ [ that [ $\epsilon$$_{i}$ included high-tech items]]]}
\enumsentence{the export exhibition [ [$_{NP_{w}}$$\epsilon$]$_{i}$ [ $\epsilon$$_{C}$ [Muriel planned  $\epsilon$$_{i}$]]]}

The lexicalized nature of XTAG makes it problematic to have trees headed by
null strings. Of the two null trees, NP$_{w}$ and Comp, that we could postulate,
the former is definitely more undesirable because it would lead to 
massive overgeneration, as can be seen in (\ex{1}) and (\ex{2}).

\enumsentence{* [$_{NP_{w}}$$\epsilon$] did John eat the apple? (as a {\em wh}-question)}
\enumsentence{* I wonder [[$_{NP_{w}}$$\epsilon$] Mary likes John](as an indirect question)}

The presence of an initial headed by a null Comp does not lead to 
problems of overgeneration because relative clauses are the only 
environment with a Comp substitution node. \footnote{Complementizers
in clausal complementation are introduced by adjunction. See
section \ref{comp-distr}.}

Consequently. our treatment of relative clauses has different 
trees to handle relative clauses with an overt extracted {\em wh}-NP
and relative clauses with a covert extracted {\em wh}-NP. Relative
clauses with an overt extracted {\em wh}-NP involve substitution
of a $+${\bf $<$wh$>$} NP into the NP$_{w}$ node
\footnote{The feature equation used is
{\bf NP$_{w}$.t:$<$wh$> = +$}. Examples of NPs that could substitute under
NP$_{w}$ are {\em whose mother}, {\em who}, {\em whom}, and also 
{\em which} but not {\em when} and {\em where} which are treated as exhaustive 
$+${\em wh} PPs.
}
and have a Comp node headed 
by $\epsilon$$_{C}$ built in. Relative clauses with a covert extracted 
{\em wh}-NP have a NP$_{w}$ node headed by $\epsilon$$_{w}$ built in and
involve substitution into the Comp node. The Comp node that is introduced
by substitution can be the $\epsilon$$_{C}$ (null complementizer), {\em that},
and {\em for}. 

For example, the tree shown in
Figure~\ref{trans-rel-clause-trees}(b) is used for the relative
clauses shown in sentences (\ex{1})-(\ex{2}), while the tree shown
in Figure~\ref{trans-rel-clause-trees}(a) is used for the relative
clauses in sentences (\ex{3})-(\ex{6}). 


\enumsentence{the man who Muriel likes}
\enumsentence{the man whose mother Muriel likes}
%\enumsentence{what Muriel likes}
\enumsentence{the man Muriel likes}
\enumsentence{the book for Muriel to read}
\enumsentence{the man that Muriel likes}
\enumsentence{the book Muriel is reading}

Cases of PP pied-piping (cf. \ex{1}) are handled in a similar fashion
 by building in a PP$_{w}$ node.
\enumsentence{the demon by whom Muriel was chased}
See the tree in Figure~\ref{trans-rel-clause-trees2}. 

\begin{figure}[htb]
\begin{tabular}{cc}
\centerline{\psfig{figure=ps/rel_clauses-files/NbetaNpxnx0Vnx1.ps,height=12.0cm}}
\end{tabular}
\caption{Adjunct relative clause tree with PP-pied-piping in the transitive tree family: 
$\beta$Npxnx0Vnx1}
\label{trans-rel-clause-trees2}
\label{2;Npxnx0Vnx1}
\end{figure}

\section{Complementizers and clauses}
The co-occurrence constraints that exist between various Comps
and the clause type of the clause they occur with are 
implemented through combinations of different
clause types using the {\bf $<$mode$>$} feature, the {\bf $<$select-mode$>$}
feature, and the {\bf $<$rel-pron$>$} feature. 

Clauses are specified for the {\bf $<$mode$>$} feature which indicates
the clause type of that clause. Possible values for the {\bf $<$mode$>$}
feature are {\bf ind, inf, ppart, ger} etc. 

Comps are lexically specified for a feature named {\bf $<$select-mode$>$}.
In addition, the {\bf $<$select-mode$>$} feature of the Comp is 
equated with the  {\bf $<$mode$>$} feature of its complement S by the following equation:\\
{\bf S$_{r}$.t:$\langle$mode$\rangle =$ Comp.t:$\langle$select-mode$\rangle$}

The lexical specifications of the Comps are shown below:
\begin{itemize}
\item $\epsilon$$_{C}$, {\bf Comp.t:$\langle$select-mode$\rangle 
=$ind/inf/ger/ppart}
\item {\em that}, {\bf Comp.t:$\langle$select-mode$\rangle =$ind}
\item {\em for}, {\bf Comp.t:$\langle$select-mode$\rangle =$inf}
\end{itemize}

The following examples display the co-occurence constraints which 
the {\bf $<$select-mode$>$} specifications assigned above implement.

For $\epsilon$$_{C}$:
\enumsentence{the book Muriel likes ({\bf S.t:$<$mode$> =$ ind})}
\enumsentence{a book to like ({\bf S.t:$<$mode$> =$ inf})}
\enumsentence{the girl reading the book ({\bf S.t:$<$mode$> =$ ger})}
\enumsentence{the book read by Muriel ({\bf S.t:$<$mode$> =$ ppart})}

For {\em for}:
\enumsentence{*the book for Muriel likes ({\bf S.t:$<$mode$> =$ ind})}
\enumsentence{a book for Mary to like ({\bf S.t:$<$mode$> =$ inf})}
\enumsentence{*the girl for reading the book ({\bf S.t:$<$mode$> =$ ger})}
\enumsentence{*the book for read by Muriel ({\bf S.t:$<$mode$> =$ ppart})}

For {\em that}:
\enumsentence{the book that Muriel likes ({\bf S.t:$<$mode$> =$ ind})}
\enumsentence{*a book that (Muriel) to like ({\bf S.t:$<$mode$> =$ inf})}
\enumsentence{*the girl that reading the book ({\bf S.t:$<$mode$> =$ ger})}
\enumsentence{*the book that read by Muriel ({\bf S.t:$<$mode$> =$ ppart})}

Relative clause trees that have substitution of {\bf NP$_{w}$} have
the following feature equations:\\
{\bf S$_{r}$.t:$\langle$mode$\rangle =$ NP$_{w}$.t:$\langle$select-mode$\rangle$}\\
{\bf NP$_{w}$.t:$\langle$select-mode$\rangle =$ind}

The examples that follow are intended to provide the rationale for 
the above setting of features.
\enumsentence{
the boy whose mother chased the cat ({\bf S$_{r}$.t:$\langle$mode$\rangle =$ind})}
\enumsentence{
*the boy whose mother to chase the cat ({\bf S$_{r}$.t:$\langle$mode$\rangle =
$inf})}
\enumsentence{
*the boy whose mother eaten the cake ({\bf S$_{r}$.t:$\langle$mode$\rangle 
=$ppart})}
\enumsentence{
*the boy whose mother chasing the cat ({\bf S$_{r}$.t:$\langle$mode$\rangle =$
ger})}
\enumsentence{
the boy [whose mother]$_{i}$ Bill believes $\epsilon$$_{i}$ to chase the cat\\ ({\bf S$_{r}$.t:
$\langle$mode$\rangle =$ind})}

The feature equations that appear in trees which have substitution of 
{\bf PP$_{w}$} are:\\
{\bf S$_{r}$.t:$\langle$mode$\rangle =$ PP$_{w}$.t:$\langle$select-mode$\rangle$}\\
{\bf PP$_{w}$.t:$\langle$mode$\rangle =$ind/inf} \footnote{As is the case for
{\bf NP$_{w}$} substitution, any  $+${\bf wh}-PP can substitute under PP$_{w}$.
This is implemented by the following equation:\\
{\bf PP$_{w}$.t:$\langle$wh$\rangle = +$}

Not all cases of pied-piping involve substitution of {\bf PP$_{w}$}.
In some cases, the P may be built in. In cases where part of the pied-piped
PP is part of the anchor, it continues to function as an anchor even after
pied-piping i.e. the P node and the {\bf NP$_{w}$} nodes are represented
separately.
}

Examples that justify the above feature setting follow.
\enumsentence{
the person [by whom] this machine was invented ({\bf S$_{r}$.t:$\langle$mode$\rangle =$ind})}
\enumsentence{
a baker [in whom]$_{i}$ PRO to trust $\epsilon$$_{i}$ ({\bf S$_{r}$.t:$\langle$mode$\rangle =$
inf})}
\enumsentence{
*the fork [with which] (Geoffrey) eaten the pudding ({\bf S$_{r}$.t:$\langle$
mode$\rangle =$ppart})}
\enumsentence{
*the person [by whom] (this machine) inventing ({\bf S$_{r}$.t:$\langle$mode
$\rangle =$ger})}

\subsection{Further constraints on the null Comp $\epsilon$$_{C}$}
There are additional constraints on where the null Comp $\epsilon$$_{C}$
can occur. The null Comp is not permitted in cases of subject
extraction unless there is an intervening clause or or
the relative clause is a reduced relative ({\bf mode = ppart/ger}).
This can be seen in (\ex{1}-\ex{4}). 

\enumsentence{
*the toy [$\epsilon$$_{i}$ [$\epsilon$$_{C}$ [ $\epsilon$$_{i}$ likes Dafna]]]}
\enumsentence{
the toy [$\epsilon$$_{i}$ [$\epsilon$$_{C}$ Fred thinks [ $\epsilon$$_{i}$ likes Dafna]]]}
\enumsentence{
the boy [$\epsilon$$_{i}$ [$\epsilon$$_{C}$ [ $\epsilon$$_{i}$ eating the guava]]]}
\enumsentence{
the guava [$\epsilon$$_{i}$ [$\epsilon$$_{C}$ [ $\epsilon$$_{i}$ eaten by the boy]]]}

To model this paradigm, the feature {\bf $\langle$rel-pron$\rangle$} is used in
conjunction with the following equations:

\begin{itemize}
\item {\bf S$_{r}$.t:$\langle$rel-pron$\rangle =$ Comp.t:$\langle$rel-pron$\rangle$}
\item {\bf S$_{r}$.b:$\langle$rel-pron$\rangle =$ S$_{r}$.b:$\langle$mode$\rangle$}
\item {\bf Comp.b:$\langle$rel-pron$\rangle =$ppart/ger/adj-clause}
(for $\epsilon$$_{C}$)
\end{itemize}

The full set of the equations shown above is only present in Comp
substitution trees involving subject extraction. So (\ex{1}) will
not be ruled out.

\enumsentence{
the toy [$\epsilon$$_{i}$ [$\epsilon$$_{C}$ [ Dafna likes $\epsilon$$_{i}$ ]]]}

The feature mismatch induced by the above equations 
is not remedied by adjunction of just any S-adjunct
because all other S-adjuncts
are transparent to the {\bf $\langle$rel-pron$\rangle$} feature
because of the following equation:\\
{\bf S$_{m}$.b:$\langle$rel-pron$\rangle =$ S$_{f}$.t:$\langle$rel-pron$\rangle$}

%Chapter~\ref{scomps-section}.

\section{Reduced Relatives}
Reduced relatives are permitted only in cases of subject-extraction.
Past participial reduced relatives are only permitted on passive
clauses.
See (\ex{1}-\ex{8}).

\enumsentence{
the toy [$\epsilon$$_{i}$ [$\epsilon$$_{C}$ [ $\epsilon$$_{i}$ playing the banjo]]]
}
\enumsentence{
*the instrument [$\epsilon$$_{i}$ [$\epsilon$$_{C}$ [ Amis playing $\epsilon$$_{i}$ ]]]
}
\enumsentence{
*the day [$\epsilon$$_{w}$ [$\epsilon$$_{C}$ [ Amis playing the banjo]]]
}
\enumsentence{
the apple [$\epsilon$$_{i}$ [$\epsilon$$_{C}$ [ $\epsilon$$_{i}$ eaten by Dafna]]]
}
\enumsentence{
*the child [$\epsilon$$_{i}$ [$\epsilon$$_{C}$ [ the apple eaten by $\epsilon$$_{i}$ ]]]
}
\enumsentence{
*the day [$\epsilon$$_{w}$ [$\epsilon$$_{C}$ [ Amis eaten the apple]]]
}
\enumsentence{
*the apple [$\epsilon$$_{i}$ [$\epsilon$$_{C}$ [ Dafna eaten $\epsilon$$_{i}$ ]]]
}
\enumsentence{
*the child [$\epsilon$$_{i}$ [$\epsilon$$_{C}$ [ $\epsilon$$_{i}$ eaten the apple ]]]
}

These restrictions are built into the {\bf $<$mode$>$} specifications
of {\bf S.t}. So non-passive cases of subject extraction have the following
feature equation:\\
{\bf S$_{r}$.t:$\langle$mode$\rangle =$ ind/ger/inf}

Passive cases of subject extraction have the following
feature equation:\\
{\bf S$_{r}$.t:$\langle$mode$\rangle =$ ind/ger/ppart/inf}

Finally, all cases of non-subject extraction have the following
feature equation:\\
{\bf S$_{r}$.t:$\langle$mode$\rangle =$ ind/inf}\\

\subsection{Restrictive vs. Non-restrictive relatives}

The English XTAG grammar does not contain any  syntactic distinction between
restrictive and non-restrictive relatives because we believe this to
be a semantic and/or pragmatic difference.

\section{External syntax}
A relative clause can combine with the NP it modifies in at least 
the following two ways:

\enumsentence{\ [the [toy [$\epsilon$$_{i}$ [$\epsilon$$_{C}$ [Dafna likes $\epsilon$$_{i}$ ]]]]]
}
\label{n-attach-ex}
\enumsentence{\ [[the toy] [$\epsilon$$_{i}$ [$\epsilon$$_{C}$ [Dafna likes $\epsilon$$_{i}$ ]]]]
}
\label{np-attach-ex}

Based on cases like (\ex{1}) and (\ex{2}), which are problematic for the
structure in (\ref{n-attach-ex}), the structure in (\ref{np-attach-ex}) is adopted.

\enumsentence{ [[the man and the woman] [who met on the bus]]}
\enumsentence{ [[the man and the woman] [who like each other]]} 

As it stands, the RC analysis sketched so far will combine in two
ways with the Determiner tree shown in Figure~(\ref{trans-rel-clause-trees3}),
\footnote{The determiner tree shown has the {\bf $<$rel-clause$>$} 
feature built in. The RC analysis would give two
parses in the absence of this feature.}
giving us both the possiblities shown in (\ref{n-attach-ex}) and (\ref{np-attach-ex}). In order
to block the structure exemplified in (\ref{n-attach-ex}), the feature 
{\bf $\langle$rel-clause$\rangle$} is used in combination with the following
equations.

\begin{figure}[htb]
\begin{tabular}{cc}
\centerline{\psfig{figure=ps/rel_clauses-files/NbetaDnx.ps,height=10.0cm}}
\end{tabular}
\label{trans-rel-clause-trees3}
\caption{Determiner tree with {\bf $<$rel-clause$>$} feature: $\beta$Dnx}
\end{figure}


On the RC:\\
{\bf NP$_{r}$.b:$\langle$rel-clause$\rangle = +$}

On the Determiner tree:\\
{\bf NP$_{f}$.t:$\langle$rel-clause$\rangle = -$}

Together, these equations block
introduction of the determiner above the relative clause.


\section{Other Issues}

\subsection{Interaction with adjoined Comps}
The XTAG analysis now has two different ways of introducing a 
complementizer like {\em that} or {\em for}, depending upon whether
it occurs in a relative clause or in sentential complementation. 
Relative clause complementizers substitute in (using the
tree $\alpha$Comp), while sentential complementizers adjoin in
(using the tree $\beta$COMPs). Cases like (\ex{1}) where 
both kinds of complementizers illicitly occur together are blocked.

\enumsentence{*the book [$\epsilon$$_{w_{i}}$ [that [that [Muriel wrote 
$\epsilon$$_{i}$]]]]} 

This is accomplished by setting the {\bf S$_{r}$.t:$<$comp$>$} feature
in the relative clause tree to {\bf nil}. The {\bf S$_{r}$.t:$<$comp$>$} 
feature of the auxiliary tree that introduces 
(the sentential complementation) {\em that} is set to
{\bf that}. This leads to a feature clash ruling out (\ex{0}). On the
other hand, if a sentential complement taking verb is adjoined
in at S$_{r}$, this feature clash goes away (cf. \ex{1}).

\enumsentence{the book [$\epsilon$$_{w_{i}}$ [that Beth thinks [that [Muriel wrote
$\epsilon$$_{i}$]]]]}



\subsection{Adjunction on PRO}
Adjunction on PRO, which would yield the ungrammatical (\ex{1}) is blocked.

\enumsentence{*I want [[PRO [who Muriel likes] to read a book]].}
This is done by specifying the {\bf $<$case$>$} feature of {\bf NP$_{f}$} to be
{\bf nom/acc}. The {\bf $<$case$>$} feature of PRO is {\bf null}. This
leads to a feature clash and blocks adjunction of relative clauses on to
PRO.

\subsection{Adjunct relative clauses}
Two types of trees to handle adjunct relative clauses exist in the 
XTAG grammar: one in which there is {\bf PP$_{w}$} substitution with 
a null {\bf Comp} built in and one in which there is a null {\bf NP$_{w}$}
built in and a {\bf Comp} substitutes in. There is no {\bf NP$_{w}$}
substitution tree with a null {\bf Comp} built in. This is because of
the contrast between (\ex{1}) and (\ex{2}).
\enumsentence{the day [[on whose predecessor] [$\epsilon$$_{C}$ [Muriel left]]]}
\enumsentence{*the day [[whose predecessor] [$\epsilon$$_{C}$ [Muriel left]]]}
In general, adjunct relatives are not possible with an overt {\bf NP$_{w}$}. 
We do not consider (\ex{1}) and (\ex{2}) to be counterexamples to 
the above statements because we consider {\em where} and {\em when}
to be exhaustive {\bf PP}s that head a {\bf PP} initial tree.

\enumsentence{the place [where [$\epsilon$$_{C}$ [Muriel wrote her first book]]]}
\enumsentence{the time [when [$\epsilon$$_{C}$ [Muriel lived in Bryn Mawr]]]}

\subsection{ECM}
Cases where {\em for} assigns exceptional case (cf. \ex{1}, \ex{2}) are handled.

\enumsentence{a book [$\epsilon$$_{w_{i}}$ [for [Muriel to read $\epsilon$$_{i}$]]]}
\enumsentence{the time [$\epsilon$$_{w_{i}}$ [for [Muriel to leave Haverford]]]}

The assignment of case by {\em for} is implemented by a combination of the
following equations.\\
{\bf Comp.t:$\langle$assign-case$\rangle =$acc}\\
{\bf S$_{r}$.t:$\langle$assign-case$\rangle =$Comp.t:$\langle$assign-case$\rangle$}\\
{\bf S$_{r}$.b:$\langle$assign-case$\rangle =$NP$_{0}$.t:$\langle$case$\rangle$}

\section{Cases not handled}
\subsection{Partial treatment of free-relatives}
Free relatives are only partially handled. All free relatives on non-subject
positions and some free relatives on subject positions 
are handled. The structure assigned 
to free relatives treats the extracted {\em wh}-NP as the head NP of
the relative clause. The remaining relative clause modifies this
extracted {\em wh}-NP (cf. \ex{1}-\ex{3}).

\enumsentence{what(ever) [$\epsilon$$_{w_{i}}$ [$\epsilon$$_{C}$ 
[Mary likes $\epsilon$$_{i}$]]]}
\enumsentence{where(ever) [$\epsilon$$_{w}$ [$\epsilon$$_{C}$ 
[Mary lives]]]}
\enumsentence{who(ever) [$\epsilon$$_{w_{i}}$ [$\epsilon$$_{C}$ 
[Muriel thinks [$\epsilon$$_{i}$ likes Mary]]]]}

However, simple subject extractions without further emebedding are not
handled (cf. \ex{1}).

\enumsentence{who(ever) [$\epsilon$$_{w_{i}}$ [$\epsilon$$_{C}$ [$\epsilon$$_{i}$ likes Bill]]]}
This is because (\ex{-1}) is treated exactly like the ungrammatical (\ex{1}).
\enumsentence{*the person [ $\epsilon$$_{w_{i}}$ [$\epsilon$$_{C}$
[$\epsilon$$_{i}$ likes Bill]]]}


\subsection{Adjunct P-stranding}
The following cases of adjunct preposition stranding are not handled 
(cf. \ex{1}, \ex{2}).

\enumsentence{the pen Muriel wrote this letter with}
\enumsentence{the street Muriel lives on}

Adjuncts are not built into elementary trees in XTAG. So there is no
clean way to represent adjunct preposition stranding. A better
solution is, probably , available if we make use of multi-component
adjunction. 

\subsection{Overgeneration}
The following ungrammatical sentences are currently being 
accepted by the XTAG grammar. This is because no clean 
and conceptually attractive way of ruling them out
is obvious to us.

\subsubsection{{\em how} as {\em wh}-NP}
In standard American English, {\em how} is not acceptable as a 
relative pronoun (cf. \ex{1}).

\enumsentence{*the way [how [$\epsilon$$_{C}$ [PRO to solve this problem]]]}

However, (\ex{0}) is accepted by the current grammar.
The only way to rule (\ex{0}) out would be to introduce a special feature
devoted to this purpose. This is unappealing. Further, there exist
speech registers/dialects of English, where (\ex{0}) is acceptable. 

\subsubsection{{\em for}-trace effects}
(\ex{1}) is ungrammatical, being an instance of a violation of the
{\em for}-trace filter of early transformational grammar.

\enumsentence{the person [$\epsilon$$_{w_{i}}$ [for 
[$\epsilon$$_{i}$ to read the book]]]}

The XTAG grammar currently accepts (\ex{0}).\footnote{It may be of
some interest that (\ex{0}) is acceptable in certain dialects of Belfast
English.}


\subsubsection{Internal head constraint}
Relative clauses in English (and in an overwhelming number of languages)
obey a `no internal head' constraint. This constraint is exemplified in
the contrast between (\ex{1}) and (\ex{2}).

\enumsentence{the person [who$_{i}$ [$\epsilon$$_{C}$ 
Muriel likes $\epsilon$$_{i}$]]}
\enumsentence{*the person [[which person]$_{i}$ [$\epsilon$$_{C}$ 
Muriel likes $\epsilon$$_{i}$]]}

We know of no good way to rule (\ex{0}) out, while still ruling (\ex{1}) in.
\enumsentence{the person [[whose mother]$_{i}$ [$\epsilon$$_{C}$
Muriel likes $\epsilon$$_{i}$]]}

Dayal (1996) suggests that `full' NPs such as {\em which person} and
{\em whose mother} are R-expressions while {\em who} and {\em whose}
are pronouns. R-expressions, unlike pronouns, are subject to Condition C.
(\ex{-2}) is, then, ruled out as a violation of Condition C since {\em 
the person} and {\em which person} are co-indexed and {\em the person}
c-commands {\em which person}. If we accept Dayal's argument, we 
have a principled reason for allowing overgeneration of relative clauses
that violate the internal head constraint, the reason being that 
the XTAG grammar does generate binding theory violations.

\subsubsection{Overt Comp constraint on stacked relatives}
Stacked relatives of the kind in (\ex{1}) are handled.

\enumsentence{ [[the book [that Bill likes]] [which Mary wrote]]}

There is a constraint on stacked relatives: all but the relative clause
closest to the head-NP must have either an overt {\bf Comp} or 
an overt {\bf NP$_{w}$}. Thus (\ex{1}) is ungrammatical.

\enumsentence{*[[the book [that Bill likes]] [Mary wrote]]}

Again, no good way of handling this constraint is known to us 
currently. 
