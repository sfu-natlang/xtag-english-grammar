\section{Verb Classes}
\label{verb-classes}

Each main\footnote{Auxiliary verbs are handled under a different mechanism.
See section~\ref{aux-verbs} for details.} verb in the syntactic lexicon selects
at least one tree family\footnote{A tree family is a collection of trees that
would be considered related under a transformational grammar approach.  See
section~\ref{verb-transformations} for more details.}, or subcategorization
frame.  Since the tree database and syntactic lexicon are already separated for
space efficiency (see \ref{overview?}), each verb can efficiently select a
large number of trees by specifying a tree family, as opposed to each of the
individual trees.  This approach allows for a considerably reduction in the
number of trees that must be specified for any given verb or form of a verb.

There are currently 38 tree families in the system\footnote{An explanation of
the naming convention used in naming the trees and tree families is available
in Appendix \ref{naming-conventions}.}.  This section gives a brief description
of each tree family and and shows the corresponding declarative tree (before
lexicalization - the $\diamond$ indicates the anchor of the tree), along with
any peculiar characteristics or trees.  It also tells which transformations are
in each tree family, and gives the approximate number of verbs that select that
family.  A few sample verbs are given, along with example sentences.


\subsection{Intransitive: Tnx0V}\index{verbs, intransitive}
\label{nx0V-family}

\begin{description}

\item[Description:]  This tree family is selected by verbs that do not 
require an object complement of any type.  Adverbs, prepositional phrases and
other adjuncts may adjoin on, but are required for the sentences to be
grammatical.  Approximately 2,091 verbs select this family.

\item[Examples:]  {\it eat}, {\it kill}, {\it dance} \\
{\it Al ate.} \\ 
{\it Seth killed.} \\ 
{\it Hyun danced.}

\item[Declarative tree:]  See Figure~\ref{nx0V-tree}.

\begin{figure}[ht]
\centering
\begin{tabular}{c}
\psfig{figure=ps/verb-class-files/alphanx0V.ps,height=4.0cm}
\end{tabular}
\caption{Declarative Intransitive Tree:  $\alpha$nx0V}
\label{nx0V-tree}
\end{figure}

\item[Other available trees\footnote{Please see 
Section~\ref{verb-transformations} for description of each of these types of
trees.}:] Wh-moved subject, subject relative clause, imperative, determiner
gerund, NP gerund.

\end{description}




\subsection{Transitive: Tnx0Vnx1}\index{verbs,transitive}
\label{nx0Vnx1-family}

\begin{description}

\item[Description:] This tree family is selected by verbs that require only a 
NP object complement.  The NP's may be complex structures, including NP's that
take sentential complements and gerund NPs.  This does not include light verb
constructions (see Sections~\ref{nx0lVN1-family} and \ref{nx0lVN1Pnx2-family}).
Approximately 4,335 verbs select the transitive tree family.

\item[Examples:] eat, dance, take, like\\
{\it Al ate an apple.} \\ 
{\it Seth danced the tango.} \\ 
{\it Hyun is taking an algorithms course.} \\
{\it Anoop likes the fact that the semester is over.}

\item[Declarative tree:] See Figure~\ref{nx0Vnx1-tree}.

\begin{figure}[ht]
\centering
\begin{tabular}{c}
\psfig{figure=ps/verb-class-files/alphanx0Vnx1.ps,height=4.0cm}
\end{tabular}
\caption{Declarative Transitive Tree:  $\alpha$nx0Vnx1}
\label{nx0Vnx1-tree}
\end{figure}

\item[Other available trees:] wh-moved subject, wh-moved object, subject
relative clause, object relative clause, imperative, pre-sentential adjunct,
post-sentential adjunct, determiner gerund, NP gerund, passive with {\it by}
phrase, passive without {\it by} phrase, passive with wh-moved subject and {\it
by} phrase, passive with wh-moved subject and no {\it by} phrase, passive with
wh-moved object out of the {\it by} phrase, passive with wh-moved {\it by}
phrase, passive with relative clause on subject and {\it by} phrase, passive
with relative clause on subject and no {\it by} phrase, passive with relative
clause on object on the {\it by} phrase, ergative, ergative with wh-moved
subject, ergative with subject relative clause.  In addition, two other trees
that allowed transitive verbs to function as adjectives {\it the stopped truck}
are also in the family.

\end{description}





\subsection{Transitive Idioms: Tnx0Vdn1}\index{verbs,idiomatic}
\label{nx0Vdn1-family}

\begin{description}

\item[Description:]  This tree family is selected by idiomatic phrases in which
the verb, determiner, and NP are all frozen (as in {\it He kicked the
bucket.}).  Only a limited number of transformations are allowed, as compared
to the normal transitive tree family (see Section~\ref{nx0Vnx1-family}).  The
analysis of idioms has not been done for idioms in general; this tree is
included to illustrate how they could be handled in XTAG.  Other idioms that
have the same structure as {\it kick the bucket}, and that are limited to the
same transformations would select this tree, and more trees would be built to
handle other idioms.  Note that {\it John kicked the bucket} is actually
ambiguous, and would result in two parses - an idiom (meaning that John died),
a simple transitive sentences (meaning that there is an physical bucket that
John hit with his foot).

\item[Examples:] {\it kick} \\
{\it He kicked the bucket.}

\item[Declarative tree:]  See Figure~\ref{nx0Vdn1-tree}.

\begin{figure}[ht]
\centering
\begin{tabular}{c}
\psfig{figure=ps/verb-class-files/alphanx0Vdn1.ps,height=4.0cm}
\end{tabular}
\caption{Declarative Transitive Idiom Tree:  $\alpha$nx0Vdn1}
\label{nx0Vdn1-tree}
\end{figure}

\item[Other available trees:]  wh-moved subject, subject relative clause, 
imperative.

\end{description}




\subsection{Di-Transitive: Tnx0Vnx1nx2}\index{verbs,ditransitive}
\label{nx0Vnx1nx2-family}

\begin{description}

\item[Description:]  This tree family is selected by verbs that take exactly 
two NP complements.  It does {\bf not} include verbs that undergo the
ditransitive verb shift (see Section~\ref{nx0Vnx1Pnx2-family}).  Various
alternations that change a sentence in this class into a NP followed by a PP
are handled by the transitive version of the verb (see
Section~\ref{nx0Vnx1-family}) followed by a PP that adjoins on.  In these
cases, the PP is optional.  Currently, only one verb in our database ({\it
ensure}) selects the DiTransitive with PP tree family (see
Section~\ref{nx0Vnx1pnx2-family}).  In this case, the verb does not have a
transitive alternation, and so the prepositional phrase is required.
Approximately 54 verbs select the Di-Transitive tree family.

\item[Examples:] {\it ask}, {\it cook}, {\it win} \\
{\it Christy asked Mike a question.} \\ 
{\it Doug cooked his father dinner.} \\
{\it Dania won her sister a stuffed animal.}

\item[Declarative tree:]  See Figure~\ref{nx0Vnx1nx2-tree}

\begin{figure}[ht]
\centering
\begin{tabular}{c}
\psfig{figure=ps/verb-class-files/alphanx0Vnx1nx2.ps,height=4.0cm}
\end{tabular}
\caption{Declarative DiTransitive Tree:  $\alpha$nx0Vnx1nx2}
\label{nx0Vnx1nx2-tree}
\end{figure}

\item[Other available trees:] wh-moved subject, wh-moved direct object, 
wh-moved indirect object, subject relative clause, direct object relative
clause, indirect object relative clause, imperative, determiner gerund, NP
gerund, passive with {\it by} phrase, passive without {\it by} phrase, passive
with wh-moved subject and {\it by} phrase, passive with wh-moved subject and no
{\it by} phrase, passive with wh-moved object out of the {\it by} phrase,
passive with wh-moved {\it by} phrase, passive with wh-moved indirect object
and {\it by} phrase, passive with wh-moved indirect object and no {\it by}
phrase,  passive with relative clause on subject and {\it by} phrase, passive
with relative clause on subject and no {\it by} phrase, passive with relative
clause on object of the {\it by} phrase, passive with relative clause on the
indirect object and {\it by} phrase, passive with relative clause on the
indirect object and no {\it by} phrase.


\end{description}





\subsection{DiTransitive with PP:Tnx0Vnx1pnx2}\index{verbs, NP with VP verbs}
\label{nx0Vnx1pnx2-family}

\begin{description}

\item[Description:]  This tree family is selected by ditransitive verbs that
take a noun phrase followed by a prepositional phrase.  The preposition is not
constrained.  The prepositional must be required and not optional - that is,
the sentence must be ungrammatical with just the noun phrase (i.e. {\it John
put the table}).  No verbs, therefore, should select both this tree family and
the transitive tree family.  (see Section~\ref{nx0Vnx1-family}).  This tree
family is also distinguished from the ditransitive verbs, such as {\it give}
that undergo verb shifting (see Section~\ref{nx0Vnx1Pnx2-family}).  There are
approximately 61 verbs that select this tree family.

\item[Examples:] {\it associate}, {\it put}, {\it refer} \\
{\it Rostenkowski associated money with power.}   \\
{\it He put his reputation on the line.}  \\
{\it He referred all questions to his attorney.}

\item[Declarative tree:]  See Figure~\ref{nx0Vnx1pnx2-tree}

\begin{figure}[ht]
\centering
\begin{tabular}{c}
\psfig{figure=ps/verb-class-files/alphanx0Vnx1pnx2.ps,height=4.0cm}
\end{tabular}
\caption{Declarative DiTransitive with PP Tree:  $\alpha$nx0Vnx1pnx2}
\label{nx0Vnx1pnx2-tree}
\end{figure}

\item[Other available trees:]  wh-moved subject, wh-moved direct object, 
wh-moved object of PP, wh-moved PP, subject relative clause, direct object
relative clause, object of PP relative clause, imperative, determiner gerund,
NP gerund, passive with {\it by} phrase, passive without {\it by} phrase,
passive with wh-moved subject and {\it by} phrase, passive with wh-moved
subject and no {\it by} phrase, passive with wh-moved object out of the {\it
by} phrase, passive with wh-moved {\it by} phrase, passive with wh-moved object
out of the PP and {\it by} phrase, passive with wh-moved object out of the PP
and no {\it by} phrase, passive with wh-moved PP and {\it by} phrase, passive
with wh-moved PP and no {\it by} phrase, passive with relative clause on
subject and {\it by} phrase, passive with relative clause on subject and no
{\it by} phrase, passive with relative clause on object of the {\it by} phrase,
passive with relative clause on the object of the PP and {\it by} phrase,
passive with relative clause on the object of the PP and no {\it by} phrase.

\end{description}




\subsection{Ditransitive with PP shift: Tnx0Vnx1Pnx2}\index{verbs,ditransitive
with PP shift}
\label{nx0Vnx1Pnx2-family}

\begin{description}

\item[Description:]  This tree family is selected by ditransitive verbs that
undergo a shift to a {\it to} prepositional phrase.  These ditransitive verbs
are clearly constrained so that when they shift, the prepositional phrase must
start with {\it to}, unlike the di-transitives in Section~{\it
nx0Vnx1nx2-family}, in which verbs may shift to any of a number of
prepositions.  Both the shifted and non-shifted trees are included.
Approximately 55 verbs select this family.

\item[Examples:] {\it give}, {\it promise}, {\it tell} \\
{\it Bill gave Hillary flowers.} \\ 
{\it Bill gave flowers to Hillary.} \\
{\it Whitman had promised the voters a tax cut.} \\
{\it Whitman had promised a tax cut to the voters.} \\
{\it Pinnochino told Gepetto a lie.} \\
{\it Pinnochino told a lie to Gepetto.}

\item[Declarative tree:]  See Figure~\ref{nx0Vnx1Pnx2-tree}

\begin{figure}[ht]
\centering
\begin{tabular}{cc}
\psfig{figure=ps/verb-class-files/alphanx0Vnx1Pnx2.ps,height=5.0cm} &
\psfig{figure=ps/verb-class-files/alphanx0Vnx2nx1.ps,height=4.0cm} \\
$\alpha$nx0Vnx1Pnx2 & $\alpha$nx0Vnx2nx1
\end{tabular}
\caption{Declarative Ditransitive with PP shift Trees}
\label{nx0Vnx1Pnx2-tree}
\end{figure}

\item[Other available trees:] 
{\bf Non-shifted:}  wh-moved subject, wh-moved direct object, 
wh-moved indirect object, subject relative clause, direct object relative
clause, indirect object relative clause, imperative, NP
gerund, passive with {\it by} phrase, passive without {\it by} phrase, passive
with wh-moved subject and {\it by} phrase, passive with wh-moved subject and no
{\it by} phrase, passive with wh-moved object out of the {\it by} phrase,
passive with wh-moved {\it by} phrase, passive with wh-moved indirect object
and {\it by} phrase, passive with wh-moved indirect object and no {\it by}
phrase,  passive with relative clause on subject and {\it by} phrase, passive
with relative clause on subject and no {\it by} phrase, passive with relative
clause on object of the {\it by} phrase, passive with relative clause on the
indirect object and {\it by} phrase, passive with relative clause on the
indirect object and no {\it by} phrase; \\
{\bf Shifted:} wh-moved subject, wh-moved direct object, 
wh-moved object of PP, wh-moved PP, subject relative clause, direct object
relative clause, object of PP relative clause, imperative, determiner gerund,
NP gerund, passive with {\it by} phrase, passive without {\it by} phrase,
passive with wh-moved subject and {\it by} phrase, passive with wh-moved
subject and no {\it by} phrase, passive with wh-moved object out of the {\it
by} phrase, passive with wh-moved {\it by} phrase, passive with wh-moved object
out of the PP and {\it by} phrase, passive with wh-moved object out of the PP
and no {\it by} phrase, passive with wh-moved PP and {\it by} phrase, passive
with wh-moved PP and no {\it by} phrase, passive with relative clause on
subject and {\it by} phrase, passive with relative clause on subject and no
{\it by} phrase, passive with relative clause on object of the {\it by} phrase,
passive with relative clause on the object of the PP and {\it by} phrase,
passive with relative clause on the object of the PP and no {\it by} phrase.


\end{description}




\subsection{Sentential Complement with NP: Tnx0Vnx1s2}\index{verbs,Sentential
Complement with NP} 
\label{nx0Vnx1s2-family}

\begin{description}

\item[Description:]  The tree family is selected by verbs that take both a NP
and sentential complement.  The sentential complement may be an infinitive,
indicative, or small clause (see Section~\ref{small-clauses}).  The type of
clause is specified by each individual verb in its syntactic lexicon entry.  A
given verb may select more than one type of sentential complement.  The
declarative tree, and many other trees in this family, are auxiliary trees, as
opposed to the more normal initial trees.  These auxiliary trees adjoin onto an
S node in an existing tree of the type specified by the sentential complement.
This is the mechanism by which TAGs are able to maintain long-distance
dependencies (see Section~\ref{long-distance-dependencies}), even over multiple
embeddings (ie. {\it What did Bill tell Mary that John said?}.  Approximately
106 verbs select this tree family.

\item[Examples:] {\it beg}, {\it promise}, {\it regard} \\
{\it Srini begged Mark to increase his disk quota.} \\
{\it Jim promised that he would feed the dogs.} \\
{\it Dania regarded Carl a jerk.}

\item[Declarative tree:]  See Figure~\ref{nx0Vnx1s2-tree}

\begin{figure}[ht]
\centering
\begin{tabular}{c}
\psfig{figure=ps/verb-class-files/betanx0Vnx1s2.ps,height=4.0cm}
\end{tabular}
\caption{Declarative Transitive Tree:  $\beta$nx0Vnx1s2}
\label{nx0Vnx1s2-tree}
\end{figure}

\item[Other available trees:]  wh-moved subject, wh-moved object, wh-moved
sentential complement, subject relative clause, object relative clause,
imperative, determiner gerund, NP gerund, passive with {\it by} phrase before
sentential complement, passive with {\it by} phrase after sentential
complement, passive without {\it by} phrase, passive with wh-moved subject and
{\it by} phrase before sentential complement, passive with wh-moved subject and
{\it by} phrase after sentential complement, passive with wh-moved subject and
no {\it by} phrase, passive with wh-moved object out of the {\it by} phrase,
passive with wh-moved {\it by} phrase, passive with relative clause on subject
and {\it by} phrase before sentential complement, passive with relative clause
on subject and {\it by} phrase after sentential complement, passive with
relative clause on subject and no {\it by} phrase.

\end{description}



\subsection{Intransitive Verb Particle: Tnx0Vpl}\index{verbs,verb particle,intransitive}
\label{nx0Vpl}

\begin{description}

\item[Description:]  The trees in this tree family are anchored by both the
verb and the verb particle.  Both appear in the syntactic lexicon and together
select this tree family.  Intransitive verb particles can be difficult to 
distinguish from intransitive verbs with adverbs adjoined on. The main
diagnostics for including verbs in this class was whether the meaning seemed
compositional or not, and whether there existed a transitive version of the
verb/verb particle combination with the same meaning.  The existence of an
alternate compositional meaning was a strong indication that a verb particle
construction consisted.  There are approximately 164 verb/verb particle
combinations.

\item[Examples:] {\it add up}, {\it come out}, {\it sign off} \\
{\it The number never quite added up.} \\
{\it John finally came out (of the closet).} \\
{\it I think that I will sign off now.}

\item[Declarative tree:]  See Figure~\ref{nx0Vpl-tree}

\begin{figure}[ht]
\centering
\begin{tabular}{c}
\psfig{figure=ps/verb-class-files/alphanx0Vpl.ps,height=4.0cm}
\end{tabular}
\caption{Declarative Intransitive Verb Particle Tree:  $\alpha$nx0Vpl}
\label{nx0Vpl-tree}
\end{figure}

\item[Other available trees:] Wh-moved subject, subject relative clause, 
imperative, determiner gerund, NP gerund.

\end{description}




\subsection{Transitive Verb Particle: Tnx0Vplnx1}\index{verbs,particle,transitive}
\label{nx0Vplnx1-family}

\begin{description}

\item[Description:]  Verb/Verb particle combinations that take an NP complement
select this tree family.  Both the verb and the verb particle are anchors of
the trees.  Distinguishing verb particles with an NP complement from
prepositional phrases can be somewhat controversial, but we fell back on a
purely syntactic test.  If the Prep/particle lexical item moved, then it was a
particle.  If it did not, then it was a preposition heading a prepositional
phrase.  In many, but not all, of the verb particle cases, there was an
alternate prepositional meaning in which the lexical item did not move.
(ie. {\it He looked up the number (in the phonebook).  He looked the number
up. He looked up the road (to see if any cars were coming).  *He looked the
road up.})  There are approximately 841 verb/verb particle combinations.

\item[Examples:] {\it blow off}, {\it make up}, {\it pick out} \\
{\it He blew off classes for the third time this year.} \\
{\it He blew classes off for the third time this year.} \\
{\it The dyslexic leprechaun made up the syntactic lexicon.} \\
{\it The dyslexic leprechaun made the syntactic lexicon up.} \\
{\it I would like to pick out a new computer.} \\
{\it I would like to pick a new computer out.} 

\item[Declarative tree:]  See Figure~\ref{nx0Vplnx1-tree}.

\begin{figure}[ht]
\centering
\begin{tabular}{cc}
\psfig{figure=ps/verb-class-files/alphanx0Vplnx1.ps,height=4.0cm} &
\psfig{figure=ps/verb-class-files/alphanx0Vnx1pl.ps,height=4.0cm} \\
$\alpha$nx0Vplnx1 & $\alpha$nx0Vnx1pl
\end{tabular}
\caption{Declarative Transitive Verb Particle Tree}
\label{nx0Vplnx1-tree}
\end{figure}

\item[Other available trees:] wh-moved subject with particle before the NP,
wh-moved subject with particle after the NP, wh-moved object, subject relative
clause with particle before the NP, subject relative clause with particle after
the NP, object relative clause, imperative with particle before the NP,
imperative with particle after the NP, determiner gerund with particle before
the NP, NP gerund with particle before the NP, NP gerund with particle after
the NP, passive with {\it by} phrase, passive without {\it by} phrase, passive
with wh-moved subject and {\it by} phrase, passive with wh-moved subject and no
{\it by} phrase, passive with wh-moved object out of the {\it by} phrase,
passive with wh-moved {\it by} phrase, passive with relative clause on subject
and {\it by} phrase, passive with relative clause on subject and no {\it by}
phrase, passive with relative clause on object of the {\it by} phrase.

\end{description}




\subsection{Ditransitive Verb Particle: Tnx0Vplnx1nx2}\index{verbs,particle,ditransitive}
\label{nx0Vplnx1nx2}

\begin{description}

\item[Description:]  Verb/verb particle combinations that select this tree
family take 2 NP complements.  Both the verb and the verb particle anchor the
trees, and the verb particle can occur before, between, or after the noun
phrases.  Perhaps because of the complexity of the sentence, these verbs do not
seem to have passive alternations (*{\it A new bank account was opened Michelle
by me.})  There are 4 verb/verb particle combinations that select this tree
family.  The exhaustive list is given in the examples.

\item[Examples:] {\it dish out}, {\it open up}, {\it pay off}, {\it rustle up}
\\
{\it I opened up Michelle a new bank account.} \\
{\it I opened Michelle up a new bank account.} \\
{\it I opened Michelle a new bank account up.}


\item[Declarative tree:]  See Figure~\ref{nx0Vplnx1nx2-tree}

\begin{figure}[ht]
\centering
\begin{tabular}{ccc}
\psfig{figure=ps/verb-class-files/alphanx0Vplnx1nx2.ps,height=4.0cm} &
\psfig{figure=ps/verb-class-files/alphanx0Vnx1plnx2.ps,height=4.0cm} &
\psfig{figure=ps/verb-class-files/alphanx0Vnx1nx2pl.ps,height=4.0cm} \\
$\alpha$nx0Vplnx1nx2 & $\alpha$nx0Vnx1plnx2 & $\alpha$nx0Vnx1nx2pl
\end{tabular}
\caption{Declarative Ditransitive Verb Particle Tree}
\label{nx0Vplnx1nx2-tree}
\end{figure}

\item[Other available trees:] wh-moved subject with particle before the NPs,
wh-moved subject with particle between the NPs, wh-moved subject with particle
after the NPs, wh-moved indirect object with particle before the NPs, wh-moved
indirect object with particle after the NPs, wh-moved direct object with
particle before the NPs, wh-moved direct object with particle between the NPs,
subject relative clause with particle before the NPs, subject relative clause
with particle between the NPs, subject relative clause with particle after the
NPs, indirect object relative clause with particle before the NPs, indirect
object relative clause with particle after the NPs, direct object relative
clause with particle before the NPs, direct object relative clause with
particle between the NPs, imperative with particle before the NPs, imperative
with particle between the NPs, imperative with particle after the NPs,
determiner gerund with particle before the NPs, NP gerund with particle before the NPs, NP gerund with particle between the NPs, NP gerund with
particle after the NPs.

\end{description}





\subsection{Intransitive with PP:Tnx0Vpnx1}\index{verbs,intransitive with PP}
\label{nx0Vpnx1-family}
\begin{description}

\item[Description:]  The verbs that select this tree family are not strictly 
intransitive, in that they {\bf must} be followed by a prepositional phrase.
Verbs that are intransitive and simply {\bf can} be followed by a prepositional
phrase do not select this family, but instead have the PP adjoin onto the
intransitive sentence.  Accordingly, there should be no verbs in both this
class and the intransitive tree family (see Section~\ref{nx0V-family}).  The
prepositional phrase is not restricted to be headed by any particular lexical
item.  Note that these are not transitive verb particles (see
Section~\ref{nx0Vplnx1-family}, since the head of the PP does not move.
Approximately 171 verbs select this tree family.

\item[Examples:] {\it grab}, {\it impinge}, {\it provide} \\
{\it Seth grabbed for the brass ring.} \\
{\it The noise gradually impinged on Dania's thoughts.} \\
{\it A good host provides for everyone's needs.}

\item[Declarative tree:]  See Figure~\ref{nx0Vpnx1-tree}

\begin{figure}[ht]
\centering
\begin{tabular}{c}
\psfig{figure=ps/verb-class-files/alphanx0Vpnx1.ps,height=4.0cm}
\end{tabular}
\caption{Declarative Intransitive with PP Tree:  $\alpha$nx0Vpnx1}
\label{nx0Vpnx1-tree}
\end{figure}

\item[Other available trees:]  wh-moved subject, wh-moved object of the PP,
wh-moved PP, subject relative clause, object of the PP relative clause,
imperative, determiner gerund, NP gerund, passive with {\it by} phrase, passive
without {\it by} phrase, passive with wh-moved subject and {\it by} phrase,
passive with wh-moved subject and no {\it by} phrase, passive with wh-moved
{\it by} phrase, passive with relative clause on subject and {\it by} phrase,
passive with relative clause on subject and no {\it by} phrase, passive with
relative clause on object of the {\it by} phrase.

\end{description}



\subsection{Sentential Complement: Tnx0Vs1}\label{verbs,sentential complement}
\label{nx0Vs1-family}

\begin{description}

\item[Description:]  The tree family is selected by verbs that take just a
sentential complement.  The sentential complement may be of type infinitive,
indicative, or small clause (see Section~\ref{small-clauses}).  The type of
clause is specified by each individual verb in its syntactic lexicon entry, and
a given verb may select more than one type of sentential complement.  The
declarative tree, and many other trees in this family, are auxiliary trees, as
opposed to the more normal initial trees.  These auxiliary trees adjoin onto an
S node in an existing tree of the type specified by the sentential complement.
This is the mechanism by which TAGs are able to maintain long-distance
dependencies (see Section~\ref{long-distance-dependencies}), even over multiple
embeddings (ie. {\it What did Bill think that John said?}  Approximately 318
verbs select this tree family.

\item[Examples:]  {\it consider}, {\it think}, {\it want} \\
{\it Dania considered the algorithm unworkable.}
{\it Srini thought that the program was working.} \\
{\it They wanted to make the sentence parse correctly.}

\item[Declarative tree:]  See Figure~\ref{nx0Vs1-tree}

\begin{figure}[ht]
\centering
\begin{tabular}{c}
\psfig{figure=ps/verb-class-files/betanx0Vs1.ps,height=4.0cm}
\end{tabular}
\caption{Declarative Sentential Complement Tree:  $\beta$nx0Vs1}
\label{nx0Vs1-tree}
\end{figure}

\item[Other available trees:]  Wh-moved subject, Wh-moved sentential
complement, subject relative clause, imperative, determiner gerund, NP gerund.

\end{description}




\subsection{Intransitive with Adjective: Tnx0Va1}\index{verbs,intransitive with adjective}
\label{nx0Va1-family}

\begin{description}

\item[Description:]  The verbs that select this tree family take an adjective
as a complement.  The adjective may be regular, comparative, or superlative.
It may also be formed from the special class of adjectives derived from the
transitive verbs (i.e. {\it agitated, broken}.  See Section~\ref{AV-tree}).
Unlike the Intransitive with PP verbs (see Section~\ref{nx0Vpnx1-family}), some
of these verbs may also occur as bare intransitives as well.  This distinction
is drawn because adjectives do not normally adjoin onto sentences, as
prepositional phrases do.  Other intransitive verbs can only occur with the
adjective, and these select only this family.  The verb class is also
distinguished from the adjective small clauses (see
Section''\ref{nx0Ax1-family}) because these verbs are not raising verbs.
Approximately 34 verbs select this tree family.

\item[Examples:] {\it become}, {\it grow}, {\it smell} \\
{\it The greenhouse became hotter.} \\
{\it The plants grew tall and strong.} \\
{\it The flowers smelled wonderful.}

\item[Declarative tree:]  See Figure~\ref{nx0Va1-tree}

\begin{figure}[ht]
\centering
\begin{tabular}{c}
\psfig{figure=ps/verb-class-files/alphanx0Va1.ps,height=4.0cm}
\end{tabular}
\caption{Declarative Intransitive with Adjective Tree:  $\alpha$nx0Va1}
\label{nx0Va1-tree}
\end{figure}

\item[Other available trees:]  Wh-moved subject, Wh-moved adjective 
({\it how}), subject relative clause, imperative, NP gerund.

\end{description}




\subsection{Sentential Subject:Ts0Vnx1}\index{verbs,sentential subject}
\label{s0Vnx1-family}

\begin{description}

\item[Description:] The verbs that select this tree family all take sentential
subjects, and are often referred to as 'psych' verbs, since they all refer to
some psychological state of mind.  The sentential subject can be indicative
(complementizer required) or infinitive (complementizer optional).
Approximately 100 verbs that select this tree family.

\item[Examples:] {\it delight}, {\it impress}, {\it surprise} \\
{\it That the tea had rosehips in it delighted Christy.} \\
{\it To even attempt a marathon impressed Dania.} \\
{\it For Jim to have walked the dogs surprised Beth.}

\item[Declarative tree:]  See Figure~\ref{s0Vnx1-tree}

\begin{figure}[ht]
\centering
\begin{tabular}{c}
\psfig{figure=ps/verb-class-files/alphas0Vnx1.ps,height=4.0cm}
\end{tabular}
\caption{Declarative Sentential Subject Tree:  $\alpha$s0Vnx1}
\label{s0Vnx1-tree}
\end{figure}

\item[Other available trees:]  Wh-moved subject, Wh-moved object.

\end{description}





\subsection{Light Verbs: Tnx0lVN1, Tnx0lVdxN1}\index{verbs, light}
\label{nx0lVN1-family}

\begin{description}

\item[Description:] The verb/noun pairs that select this tree family are pairs
in which the interpretation is non-compositional and the noun contributes
argument structure to the predicate (i.e. {\it The man took a walk.} vs {\it
The man took a radio}).  The verb and the noun occur together in the syntactic
database, and both anchor the trees.  The verbs in the light verb constructions
are {\it do}, {\it give}, {\it have}, {\it make}, and {\it take}.  The noun
following the light verb is (usually) a bare infinitive form (ie. {\it have a
good cry}) and usually occurs with {\it a(n)}.  However, we include deverbal
nominals ({\it take a bath}, {\it give a demonstration}) as well.  onstructions
with nouns that do not contribute an argument structure ({\it have a
cigarette}, {\it give NP a black eye}) are excluded.  In addition to semantic
considerations of light verbs, they differ syntactically from transitive verbs
(see Section~\ref{nx0Vnx1-family}) as well in that the noun in the light verb
construction does not extract.  Because the noun is an anchor in the tree,
there are two different tree families representing nouns that require
determiners and those that occur without them (see Section~\ref{nouns} for more
information on noun phrases in general).  There are approximately 96 verb/noun
pairs that select the light verb tree without determiners, and 242 that select
the light verb tree with determiners.

\item[Examples:] {\it give groan}, {\it have discussion}, {\it make comment} \\
{\it The audience gave a collective groan.} \\
{\it We had a big discussion about closing the libraries.} \\
{\it The professors made comments on the paper.}

\item[Declarative tree:]  See Figure~\ref{nx0lVN1-tree}.

\begin{figure}[ht]
\centering
\begin{tabular}{cc}
\psfig{figure=ps/verb-class-files/alphanx0lVN1.ps,height=4.0cm}
\psfig{figure=ps/verb-class-files/alphanx0lVdxN1.ps,height=4.0cm} \\
$\alpha$nx0lVN1 & $\alpha$nx0lVdxN1
\end{tabular}
\caption{Declarative Light Verb Trees}
\label{nx0lVN1-tree}
\end{figure}

\item[Other available trees:] Wh-moved subject, subject relative clause, 
imperative, determiner gerund, NP gerund.

\end{description}




\subsection{Ditransitive Light Verbs with PP Shift: Tnx0lVN1Pnx2,Tnx0lVdxN1Pnx2}\index{verbs,ditransitive light verbs with PP shift}
\label{nx0lVN1Pnx2-family}

\begin{description}

\item[Description:]  The verb/noun pairs that select this tree family are pairs
in which the interpretation is non-compositional and the noun contributes
argument structure to the predicate (i.e. {\it Dania made Srini a cake.} vs
{\it Dania made Srini a loan.}).  The verb and the noun occur together in the
syntactic database, and both anchor the trees.  The verbs in these light verb
constructions are {\it give} and {\it make}.  The noun following the light verb
is (usually) a bare infinitive form (ie. {\it make promise to Anoop}).
However, we include deverbal nominals ({\it make a payment to Anoop}) as well.
Constructions with nouns that do not contribute an argument structure are
excluded.  In addition to semantic considerations of light verbs, they differ
syntactically from the ditransitive with PP shift verbs (see
Section~\ref{nx0Vnx1Pnx2-family}) as well in that the noun in the light verb
construction does not extract.  Also, passivization is severely restricted.
Because the noun is an anchor in the tree, there are two different tree
families representing nouns that require determiners and those that occur
without them (see Section~\ref{nouns} for more information on noun phrases in
general).  There are approximately 10 verb/noun pairs that select the trees
without determiners, and 18 that select the trees with determiners.

\item[Examples:] {\it give look}, {\it give wave}, {\it make promise} \\
{\it Dania gave Carl a look that would kill.} \\
{\it Amanda gave a little wave as she left.} \\
{\it Dania made Doug a promise.} 

\item[Declarative tree:]  See Figure~\ref{nx0lVN1Pnx2-tree}

\begin{figure}[ht]
\centering
\begin{tabular}{cc}
\psfig{figure=ps/verb-class-files/alphanx0lVN1Pnx2.ps,height=4.0cm} &
\psfig{figure=ps/verb-class-files/alphanx0lVnx2N1.ps,height=4.0cm} \\
$\alpha$nx0lVN1Pnx2 & $\alpha$nx0lVnx2N1 \\
\vspace{1.5cm}
\psfig{figure=ps/verb-class-files/alphanx0lVdxN1Pnx2.ps,height=4.0cm} &
\psfig{figure=ps/verb-class-files/alphanx0lVnx2dxN1.ps,height=4.0cm} \\
$\alpha$nx0lVdxN1Pnx2 & $\alpha$nx0lVnx2dxN1 \\
\end{tabular}
\caption{Declarative Light Verbs with PP Tree}
\label{nx0lVN1Pnx2-tree}
\end{figure}

\item[Other available trees:]
{\bf Non-shifted:}  wh-moved subject,
wh-moved indirect object, subject relative clause, indirect object relative 
clause, imperative, NP gerund, passive with {\it by} phrase \\
{\bf Shifted:} wh-moved subject,  wh-moved object of PP, wh-moved PP, subject 
relative clause, object of PP relative clause, imperative, determiner gerund,
NP gerund, passive with {\it by} phrase.
\end{description}




\subsection{NP It-Cleft: TItVnx1s2}
\label{ItVnx1s2-family}

\begin{description}

\item[Description:] ** Beth to fill this in **

\item[Examples:] {\it it be} \\
{\it It was Beth who agreed to do the demo.}

\item[Declarative tree:]  See Figure~\ref{ItVnx1s2-tree}

\begin{figure}[ht]
\centering
\begin{tabular}{c}
\psfig{figure=ps/verb-class-files/alphaItVnx1s2.ps,height=4.0cm}
\end{tabular}
\caption{Declarative NP It-Cleft Tree:  $\alpha$ItVnx1s2}
\label{ItVnx1s2-tree}
\end{figure}

\item[Other available trees:]  inverted question, wh-moved object with
{\it be} inverted, wh-moved object with {\it be} not inverted.

\end{description}



\subsection{PP It-Cleft: TItVpnx1s2}
\label{ItVpnx1s2-family}

\begin{description}

\item[Description:]  ** Beth to fill this in **

\item[Examples:] {\it it be} \\
{\it It was at Kent State that the police shot all those students.}

\item[Declarative tree:]  See Figure~\ref{ItVpnx1s2-tree}

\begin{figure}[ht]
\centering
\begin{tabular}{c}
\psfig{figure=ps/verb-class-files/alphaItVpnx1s2.ps,height=4.0cm}
\end{tabular}
\caption{Declarative PP It-Cleft Tree:  $\alpha$ItVnx1s2}
\label{ItVpnx1s2-tree}
\end{figure}

\item[Other available trees:] inverted question, wh-moved prepositional phrase
with {\it be} inverted, wh-moved prepositional phrase with {\it be} not
inverted.

\end{description}

\subsection{Adverb It-Cleft: TItVad1s2}
\label{ItVad1s2-family}

\begin{description}

\item[Description:]  ** Beth to fill this in **

\item[Examples:] {\it it be} \\
{\it It was reluctantly that Dania agreed to do the tech report.}

\item[Declarative tree:]  See Figure~\ref{ItVad1s2-tree}

\begin{figure}[ht]
\centering
\begin{tabular}{c}
\psfig{figure=ps/verb-class-files/alphaItVad1s2.ps,height=4.0cm}
\end{tabular}
\caption{Declarative Transitive Tree:  $\alpha$ItVad1s2}
\label{ItVad1s2-tree}
\end{figure}

\item[Other available trees:]  inverted question, wh-moved adverb {\it how}
with {\it be} inverted, wh-moved adverb {\it how} with {\it be} not
inverted.

\end{description}



\subsection{Adjective Small Clause Tree: Tnx0Ax1}\index{verbs,small-clause}
\label{nx0Ax1-family}

\begin{description}

\item[Description:]  These trees are not anchored by verbs, but by adjectives.
They are explained in much greater detail in the section on small clauses (see
Section~\ref{sm-clause-xtag-analysis}).  The section is presented here for
completeness.  Approximately 3312 adjectives select this tree family.

\item[Examples:] {\it addictive}, {\it dangerous}, {\it wary}\\
{\it Cigarettes are addictive.} \\
{\it Smoking cigarettes is dangerous.} \\
{\it John seems wary of the Surgeon General's warnings.}

\item[Declarative tree:]  See Figure~\ref{nx0Ax1-tree}

\begin{figure}[ht]
\centering
\begin{tabular}{c}
\psfig{figure=ps/verb-class-files/alphanx0Ax1.ps,height=4.0cm}
\end{tabular}
\caption{Declarative Adjective Small Clause Tree:  $\alpha$nx0Ax1}
\label{nx0Ax1-tree}
\end{figure}

\item[Other available trees:]  wh-moved subject, wh-moved adjective {\it how},
relative clause on subject, imperative, NP gerund.

\end{description}

\subsection{Adjective Small Clause with Sentential Complement: Tnx0Ax1s2}
\label{nx0Ax1s2-family}

\begin{description}

\item[Description:]  This tree family is selected by adjectives that take 
sentential complements.  The sentential complements can be indicative or
infinitive.  Note that these trees are not anchored by adjectives, not verbs.
Most adjectives that take the Adjective Small Clause tree family (see
Section~\ref{nx0Ax1-family}) takes this family as well\footnote{No great
attempt has been made to go through and decide which adjective actually should
take this family and which should not.}.  Small clauses are explained in much
greater detail in Section~\ref{sm-clause-xtag-analysis}).  This section is
presented here for completeness.  Approximately 3229 adjectives select this
tree family.

\item[Examples:] {\it able}, {\it curious}, {\it disappointed} \\
{\it Christy was able to find the problem.} \\
{\it Christy was curious whether the new analysis was working.} \\
{\it Christy was disappointed that the old analysis failed.} 

\item[Declarative tree:]  See Figure~\ref{nx0Ax1s2-tree}

\begin{figure}[ht]
\centering
\begin{tabular}{c}
\psfig{figure=ps/verb-class-files/alphanx0Ax1s2.ps,height=4.0cm}
\end{tabular}
\caption{Declarative  Adjective Small Clause with Sentential Complement Tree:  $\alpha$nx0Ax1s2}
\label{nx0Ax1s2-tree}
\end{figure}

\item[Other available trees:] wh-moved subject, wh-moved adjective {\it how},
relative clause on subject, imperative, NP gerund.

\end{description}

\subsection{Adjective Small Clause with Sentential Subject: Ts0Ax1}
\label{s0Ax1-family}

\begin{description}

\item[Description:]  This tree family is selected by adjectives that take 
sentential subjects.  The sentential subjects can be indicative or infinitive.
Note that these trees are not anchored by adjectives, not verbs.  Most
adjectives that take the Adjective Small Clause tree family (see
Section~\ref{nx0Ax1-family}) takes this family as well\footnote{No great
attempt has been made to go through and decide which adjective actually should
take this family and which should not.}.  Small clauses are explained in much
greater detail in Section~\ref{sm-clause-xtag-analysis}).  This section is
presented here for completeness.  Approximately 3227 adjectives select this
tree family.

\item[Examples:] {\it decadent}, {\it incredible}, {\it uncertain} \\
{\it To eat raspberry chocolate truffle ice cream is decadent.} \\
{\it That Carl could eat a large bowl of it is incredible.} \\
{\it Whether he will actually survive the experience is uncertain.}

\item[Declarative tree:]  See Figure~\ref{s0Ax1-tree}

\begin{figure}[ht]
\centering
\begin{tabular}{c}
\psfig{figure=ps/verb-class-files/alphas0Ax1.ps,height=4.0cm}
\end{tabular}
\caption{Declarative Adjective Small Clause with Sentential Subject Tree:  $\alpha$s0Ax1}
\label{s0Ax1-tree}
\end{figure}

\item[Other available trees:]  Wh-moved subject.

\end{description}



\subsection{Equitive {\it BE}: Tnx0BEnx1}
\label{nx0BEnx1-family}

\begin{description}

\item[Description:]  This tree family is selected only by the verb {\it be}.
It is distinguished from the predicative NPs (see Section~\ref{nx0N1-family},
or the section on English Copular (Section~\ref{english-copular} in that two
NP's are equated, and hence interchangeable.  The XTAG analysis for equative
{\it be} is explained in greater detail in
Section~\ref{equative-be-xtag-analysis}.

\item[Examples:] {\it be} \\
{\it That man is my uncle.}

\item[Declarative tree:]  See Figure~\ref{nx0BEnx1-tree}.

\begin{figure}[ht]
\centering
\begin{tabular}{c}
\psfig{figure=ps/verb-class-files/alphanx0BEnx1.ps,height=4.0cm}
\end{tabular}
\caption{Declarative Equitive {\it BE} Tree:  $\alpha$nx0BEnx1}
\label{nx0BEnx1-tree}
\end{figure}

\item[Other available trees:] Inverted-question.

\end{description}




\subsection{NP Small Clauses: Tnx0N1, Tnx0dxN1}
\label{nx0N1-family}

\begin{description}

\item[Description:]  These trees are not anchored by verbs, but by nouns.
Because they are anchored by nouns, there are two different tree families
representing nouns that require determiners and those that occur without them
(see Section~\ref{nouns} for more information on noun phrases in general).
Small clauses are explained in much greater detail in the
Section~\ref{sm-clause-xtag-analysis}).  The section is presented here for
completeness.  Approximately 9915 nouns select the tree family without
determiners, and 9888 nouns select the family with determiners.

\item[Examples:] {\it author}, {\it chair}, {dish} \\
{\it Dania is an author.} \\
{\it That blue, warped-looking thing is a chair.} \\
{\it Those broken pieces were dishes.}

\item[Declarative tree:]  See Figure~\ref{nx0N1-tree}

\begin{figure}[ht]
\centering
\begin{tabular}{cc}
\psfig{figure=ps/verb-class-files/alphanx0N1.ps,height=4.0cm} &
\psfig{figure=ps/verb-class-files/alphanx0dxN1.ps,height=4.0cm} \\
$\alpha$nx0N1 & $\alpha$nx0dxN1
\end{tabular}
\caption{Declarative NP Small Clause Trees}
\label{nx0N1-tree}
\end{figure}

\item[Other available trees:] Wh-moved subject, Wh-moved object, relative
clause on object, imperative, NP gerund.

\end{description}



%%  No nouns select this tree family.  What is up?
%%\subsection{NP Small Clauses with Sentential Complement: Tnx0dxN1s2, Tnx0N1s2}
%%\label{nx0N1s2-family}
%%
%%\begin{description}
%%
%%\item[Description:]
%%
%%\item[Examples:]
%%
%%\item[Declarative tree:]  See Figure~\ref{nx0N1s2-tree}
%%
%%\begin{figure}[ht]
%%\centering
%%\begin{tabular}{cc}
%%\psfig{figure=ps/verb-class-files/betanx0N1s2.ps,height=4.0cm} &
%%psfig{figure=ps/verb-class-files/betanx0dxN1s2.ps,height=4.0cm} \\
%%$\beta$nx0N1s2 &$\beta$nx0dxN1s2 
%%\end{tabular}
%%\caption{Declarative NP Small Clauses with Sentential Complement Tree}
%%\label{nx0N1s2-tree}
%%\end{figure}
%%
%%\item[Other available trees:]  Wh-moved subject, Wh-moved object, relative clause on object, imperative.
%%
%%\end{description}



\subsection{NP with Sentential Complement Small Clause: Tnx0dxN1s1, Tnx0N1s1}
\label{nx0N1s1-family}

\begin{description}

\item[Description:]  This tree family is selected by the small group of nouns
that take sentential complements by themselves (see
Section~\ref{NP-with-sentential-complement}).  The sentential complements can
be indicative or infinitive, depending on the noun.  Because the trees are
anchored by nouns, there are two different tree families representing nouns
that require determiners and those that occur without them (see
Section~\ref{nouns} for more information on noun phrases in general).  Small
clauses in general are explained in much greater detail in the
Section~\ref{sm-clause-xtag-analysis}).  The section is presented here for
completeness.  Approximately 216 nouns select both the family with and without
determiners.

\item[Examples:] {\it admission}, {\it claim}, {\it vow} \\
{\it The affidavits are admissions that they killed the sheep.} \\
{\it There is always is the claim that they were insane at the time.} \\
{\it This is his vow to fight the charges.}

\item[Declarative tree:]  See Figure~\ref{nx0N1s1-tree}

\begin{figure}[ht]
\centering
\begin{tabular}{cc}
\psfig{figure=ps/verb-class-files/alphanx0N1s1.ps,height=4.0cm} &
\psfig{figure=ps/verb-class-files/alphanx0dxN1s1.ps,height=4.0cm} \\
$\alpha$nx0N1s1 & $\alpha$nx0dxN1s1
\end{tabular}
\caption{Declarative NP with Sentential Complement Small Clause Tree}
\label{nx0N1s1-tree}
\end{figure}

\item[Other available trees:] Wh-moved subject, Wh-moved object, relative
clause on object, imperative, NP gerund.

\end{description}



\subsection{NP Small Clause with Sentential Subject: Ts0dxN1, Ts0N1}
\label{s0N1-family}

\begin{description}

\item[Description:]  This tree family is selected by nouns that take 
sentential subjects.  The sentential subjects can be indicative or infinitive.
Note that these trees are not anchored by nouns, not verbs.  Because they are
anchored by nouns, there are two different tree families representing nouns
that require determiners and those that occur without them (see
Section~\ref{nouns} for more information on noun phrases in general).  Most
nouns that take the NP Small Clause tree family (see
Section~\ref{nx0N1-family}) takes this family as well\footnote{No great attempt
has been made to go through and decide which nouns actually should take this
family and which should not.}.  Small clauses are explained in much greater
detail in Section~\ref{sm-clause-xtag-analysis}).  This section is presented
here for completeness.  Approximately 9888 nouns select both the tree family
with determiners and the tree family without determiners.

\item[Examples:] {\it dilemma}, {\it insanity}, {\it tragedy} \\
{\it Whether to keep the job he hates is a dilemma.} \\
{\it For Bill to invest all of his money in worms is insanity.} \\
{\it That the worms died is a tragedy.}

\item[Declarative tree:]  See Figure~\ref{s0N1-tree}

\begin{figure}[ht]
\centering
\begin{tabular}{cc}
\psfig{figure=ps/verb-class-files/alphas0N1.ps,height=4.0cm} &
\psfig{figure=ps/verb-class-files/alphas0dxN1.ps,height=4.0cm} \\
$\alpha$s0N1 & $\alpha$s0dxN1
\end{tabular}
\caption{Declarative NP Small Clause with Sentential Subject Tree}
\label{s0N1-tree}
\end{figure}

\item[Other available trees:]  Wh-moved subject.

\end{description}




\subsection{PP Small Clause: Tnx0Pnx1}
\label{nx0Pnx1-family}

\begin{description}

\item[Description:]  This family is selected by prepositions that can occur in
small clause constructions (see Section~\ref{sm-clause-xtag-analysis} for more
information on small clause constructions).  The section is presented here for
completeness.  Approximately 39 prepositions select this tree family.

\item[Examples:] {\it around}, {\it in}, {\it underneath} \\
{\it Chris is around the corner.} \\
{\it Trisha is in big trouble.} \\
{\it The dog is underneath the table.}

\item[Declarative tree:]  See Figure~\ref{nx0Pnx1-tree}

\begin{figure}[ht]
\centering
\begin{tabular}{c}
\psfig{figure=ps/verb-class-files/alphanx0Pnx1.ps,height=4.0cm}
\end{tabular}
\caption{Declarative PP Small Clause  Tree:  $\alpha$nx0Pnx1}
\label{nx0Pnx1-tree}
\end{figure}

\item[Other available trees:]  Wh-moved subject, Wh-moved object of PP, 
relative clause on subject, relative clause on object of PP, imperative, NP
gerund.

\end{description}





\subsection{Exhaustive PP Small Clause: Tnx0Px1}
\label{nx0Px1-family}

\begin{description}

\item[Description:] This family is selected by {\bf exhaustive} prepositions
that can occur in small clauses.  Exhaustive prepositions are prepositions that
function as prepositional phrases by themselves.  For more information on small
clause constructions, please see the Section~\ref{sm-clause-xtag-analysis} for
more information on small clause constructions).  The section is presented here
for completeness.  Approximately 8 exhaustive prepositions select this tree
family.

\item[Examples:] {\it abroad}, {\it below}, {\it outside} \\
{\it Dr. Joshi is abroad.} \\
{\it The workers are all below.} \\
{\it Clove is outside.}

\item[Declarative tree:]  See Figure~\ref{nx0Px1-tree}

\begin{figure}[ht]
\centering
\begin{tabular}{c}
\psfig{figure=ps/verb-class-files/alphanx0Px1.ps,height=4.0cm}
\end{tabular}
\caption{Declarative Exhaustive PP Small Clause Tree:  $\alpha$nx0Px1}
\label{nx0Px1-tree}
\end{figure}

\item[Other available trees:] Wh-moved subject, Wh-moved PP, relative clause 
on subject, imperative, NP gerund.

\end{description}


%%  I think that these trees are going away with the new sentential adjunct analysis.
%%\subsection{PP Small Clause with Sentential Complement: Tnx0Pnx1s2}
%%\label{nx0Pnx1s2-family}
%%
%%\begin{description}
%%
%%\item[Description:] Approximately 41 prepositions select this tree family.
%%
%%\item[Examples:] {\it } \\
%%
%%\item[Declarative tree:]  See Figure~\ref{nx0Pnx1s2-tree}
%%
%%\begin{figure}[ht]
%%\centering
%%\begin{tabular}{c}
%%\psfig{figure=ps/verb-class-files/alphanx0Pnx1s2.ps,height=4.0cm}
%%\end{tabular}
%%\caption{Declarative PP Small Clause with Sentential Complement Tree:  $\alpha$nx0Pnx1s2}
%%\label{nx0Pnx1s2-tree}
%%\end{figure}
%%
%%\item[Other available trees:]  Wh-moved subject, Wh-moved object of the PP, 
%%Wh-moved PP with wh+ noun, Wh-moved PP with wh+ preposition {\it where},
%%relative clause on subject, relative clause on object of PP, imperative, NP
%%gerund.
%%
%%\end{description}
%%
%%\subsection{Exhaustive PP Small Clause with Sentential Complement:Tnx0Px1s2}
%%\label{nx0Px1s2-family}
%%
%%\begin{description}
%%
%%\item[Description:]  Approximately 7 exhaustive prepositions select this
%%family. 
%%
%%\item[Examples:]
%%
%%\item[Declarative tree:]  See Figure~\ref{nx0Px1s2-tree}
%%
%%\begin{figure}[ht]
%%\centering
%%\begin{tabular}{c}
%%\psfig{figure=ps/verb-class-files/alphanx0Px1s2.ps,height=4.0cm}
%%\end{tabular}
%%\caption{Declarative Transitive Tree:  $\alpha$nx0Px1s2}
%%\label{nx0Px1s2-tree}
%%\end{figure}
%%
%%\item[Other available trees:]  Wh-moved subject,  Wh-moved PP with wh+ 
%%preposition {\it where}, relative clause on subject, imperative, NP gerund.
%%
%%\end{description}




\subsection{PP Small Clause with Sentential Subject: Ts0Pnx1}
\label{s0Pnx1-family}

\begin{description}

\item[Description:]  This tree family is selected by prepositions that take
sentential subjects.  The sentential subject can be indicative or infinitive.
Small clauses are explained in much greater detail in
Section~\ref{sm-clause-xtag-analysis}).  This section is presented here for
completeness.  Approximately 39 prepositions select this tree family.

\item[Examples:] {\it beyond}, {\it unlike} \\
{\it That Ken could forget to pay the taxes is beyond belief.} \\
{\it To explain how this happened is outside the scope of this discussion.} \\
{\it For Ken to do something right is unlike him.}


\item[Declarative tree:]  See Figure~\ref{s0Pnx1-tree}

\begin{figure}[ht]
\centering
\begin{tabular}{c}
\psfig{figure=ps/verb-class-files/alphas0Pnx1.ps,height=4.0cm}
\end{tabular}
\caption{Declarative PP Small Clause with Sentential Subject Tree:  $\alpha$s0Pnx1}
\label{s0Pnx1-tree}
\end{figure}

\item[Other available trees:] Wh-moved subject, relative clause on object of
the PP.

\end{description}




%%\section{Template}
%%
%%\begin{description}
%%
%%\item[Description:]
%%
%%\item[Declarative tree:]  See Figure~\ref{decl-???-tree}
%%
%%\begin{figure}[ht]
%%\centering
%%\begin{tabular}{c}
%%\psfig{figure=ps/verb-class-files/alpha???.ps,height=4.0cm}
%%\end{tabular}
%%\caption{Declarative Transitive Tree:  $\alpha$nx0Vnx1}
%%\label{decl-????-tree}
%%\end{figure}
%%
%%\item[Other available trees:]
%%
%%\item[Examples:]
%%
%%\end{description}


