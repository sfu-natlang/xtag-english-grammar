\section{Underview}

The morphology, syntactic, and tree databases together comprise the English
grammar.  Lexical items not in the databases are handled by default mechanisms.
The range of syntactic phenomena that can be handled is large and includes
auxiliaries (including inversion), copula, raising and small clause
constructions, topicalization, relative clauses, infinitives, gerunds,
passives, adjuncts, it-clefts, wh-clefts, PRO constructions, noun-noun
modifications, extraposition, determiner phrases, genitives, negation,
noun-verb contractions and imperatives.  The combination of large scale
lexicons and wide phenomena coverage result in a robust system.


\subsection{Subcategorization Frames}
\label{subcat-frames}

Elementary trees for non-auxiliary verbs are used to represent the linguistic
notion of subcategorization frames.  The anchor of the elementary tree
subcategorizes for the other elements that appear in the tree, forming a
clausal or sentential structure.  Tree families group together trees belonging
to the same subcategorization frame.  Consider the following uses of the verb
{\it buy}:

\enumsentence{Srini bought a book.}
\enumsentence{Srini bought Beth a book.}

In sentence (\ex{-1}), the verb {\it buy} subcategorizes for a direct object NP.
The elementary tree anchored by {\it buy} is shown in
Figure~\ref{subcat-trees}(a) and includes nodes for the NP complement of {\it
buy} and for the NP subject.  In addition to this declarative tree structure,
the tree family also contains the trees that would be related to each other
transformationally in a movement based approach, i.e passivization,
imperatives, wh-questions, relative clauses, and so forth.  Sentence (\ex{0})
shows that {\it buy} also subcategorizes for a double NP object.  This means
that {\it buy} also selects a the double NP object subcategorization frame, or
tree family, with its own set of transformationally related sentence
structures.  Figure~\ref{subcat-trees}(b) shows the declarative structure for
this set of sentence structures.

\begin{figure}[ht]
\centering
\rule[.1in]{5.0in}{0.01in}
\begin{tabular}{cc}
{\psfig{figure=ps/compl-adj-files/alphanx0Vnx1_bought.ps,height=2.0in}} & {\psfig{figure=ps/compl-adj-files/alphanx0Vnx1nx2_bought.ps,height=2.0in}}\\
(a) & (b) \\ 
\end{tabular}
\caption{Different subcategorization frames for the verb {\it buy}}
\rule[.1in]{5.0in}{0.01in}
\label{subcat-trees}
\end{figure}


\subsection{Complements and Adjuncts}

Complements and adjuncts have very different structures in our system.
Complements are included in the elementary tree anchored by the verb that
selects them, while adjuncts do not originate in the same elementary tree as
the verb anchoring the sentence, but are instead added to a structure by
adjunction.  The contrasts between complements and adjuncts have been
extensively discussed in the linguistics literature and the classification of a
given element as one or the other remains a matter of debate (see \cite{rizzi90}, 
\cite{larson88}, \cite{jackendoff90}, \cite{larson90}, \cite{cinque90}, 
\cite{obernauer84}, \cite{lasnik-saito84}, and \cite{chomsky86}).  The guiding
rule used in developing this grammar is whether or not the sentence is
ungrammatical without the questioned structure.\footnote{Iteration of a
structure can also be used as a diagnostic: {\it Srini bought a book at the
bookstore on Walnut Street for a friend at work}.} Consider the following
sentences:

\enumsentence{Srini bought a book.}
\enumsentence{Srini bought a book at the bookstore.}
\enumsentence{Srini arranged for a ride.}
\enumsentence{*Srini arranged.}

Prepositional phrases are generally considered adjuncts, and have the tree
structure shown in Figure \ref{compl-adj}a.  This adjunction tree would adjoin
into the tree shown in Figure \ref{subcat-trees}a to generate sentence (\ex{-2}).
There are verbs, however, such as {\it arrange}, {\it hunger}, and {\it
differentiate}, that take prepositional phrases as complements.  Sentences
(\ex{-1}) and (\ex{0}) clearly show that the prepositional phrase are not optional
for these verbs.  For these sentences, the prepositional phrase will be an
initial tree (as shown in Figure \ref{compl-adj}b) that substitutes into an
elementary tree, such as the one anchored by the verb {\it arrange} in Figure
\ref{compl-adj}c.

\begin{figure}[ht]
\centering
\rule[.1in]{5.0in}{0.01in}
\begin{tabular}{ccc}
{\psfig{figure=ps/compl-adj-files/betavxPnx_at.ps,height=2.0in}} & {\psfig{figure=ps/compl-adj-files/alphaPXPnx_for.ps,height=1.5in}} & {\psfig{figure=ps/compl-adj-files/alphanx0Vpnx1_arranged.ps,height=2.0in}}\\
(a) & (b) & (c) \\ 
\end{tabular}
\caption{Trees illustrating the difference between Complements and Adjuncts}
\rule[.1in]{5.0in}{0.01in}
\label{compl-adj}
\end{figure}

\subsection{Non-S constituents}

Although sentential trees are generally considered to be special cases in any
grammar, insofar as they make up a 'starting category', it is the case that any
initial tree constitutes a phrasal constituent.  These initial trees may have
substitution nodes that need to be filled (by other initial trees), and may be
modified by adjunct trees, exactly as the trees rooted in S.  Although grouping is
possible according to the heads or anchors of these trees, we have not found
any classification similar to the subcategorization frames for verbs that can
be used by a lexical entry to `group select' a set of trees.  These trees are
selected one by one by each lexical item, according to each lexical item's
idiosyncracies.  The grammar that accompanies this technical report places them
into several files for ease of use, but these files do not constitute tree
families in the way that the subcategorization frames do.

