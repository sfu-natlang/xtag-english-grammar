\section{Complements and Adjuncts}

Elementary trees in the LTAG formalism are used to represent the
linguistic notion of subcategorization frames.  The anchor of the
elementary tree subcategorizes for the other elements that appear in
the tree.  For example, the verb {\it buy} subcategorizes for a direct
object NP. The elementary tree anchored by {\it buy} shown in
Figure~\ref{comp-adj}(a) includes nodes for the NP complement of {\it
buy} and for the NP subject.  The contrasts between complements and
adjuncts have been extensively discussed in the linguistics
literature. and the classification of some element as one or the other
remains a matter of debate.  In the LTAG formalism, decisions about
whether some constituent is a complement or an adjunct leads to two
quite different structures. Complements are included in the elementary
tree anchored by the verb that selects them. Adjuncts are added to a
structure by adjunction and do not originate in the same elementary
tree as the verb anchoring the sentence. This difference is
illustrated by the tree structures shown in Figure~\ref{comp-adj}(a)
and Figure~\ref{comp-adj}(b).

\begin{figure}[ht]
\centering
\begin{tabular}{cc}
{\psfig{figure=ps/alphanx0Vnx1.ps}} & {\psfig{figure=ps/betaARBs.ps}}\\
(a) & (b) \\ 
\end{tabular}
\caption{Trees illustrating the difference between Complements and Adjuncts}
\label{comp-adj}
\end{figure}
