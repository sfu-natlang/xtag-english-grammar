\chapter{Underview}
\label{underview}

The morphology, syntactic, and tree databases together comprise the English
grammar.  A lexical item that is not in the databases receives a default tree
selection and features for its part of speech and morphology.  In designing the
grammar, a decision was made early on to err on the side of acceptance whenever
there were conflicting opinions as to whether or not a construction is
grammatical.  In this sense, the XTAG English grammar functions better as an
acceptor rather than a generator of English sentences.  The range of syntactic
phenomena that can be handled is large and includes auxiliaries (including
inversion), copula, raising and small clause constructions, topicalization,
relative clauses, infinitives, gerunds, passives, adjuncts, it-clefts,
wh-clefts, PRO constructions, noun-noun modifications, extraposition,
determiner phrases, genitives, negation, noun-verb contractions, sentential
adjuncts and imperatives.  The combination of large scale lexicons and wide
phenomena coverage result in a robust system.


\section{Subcategorization Frames}
\label{subcat-frames}

Elementary trees for non-auxiliary verbs are used to represent the linguistic
notion of subcategorization frames.  The anchor of the elementary tree
subcategorizes for the other elements that appear in the tree, forming a
clausal or sentential structure.  Tree families group together trees belonging
to the same subcategorization frame.  Consider the following uses of the verb
{\it buy}:

\enumsentence{Srini bought a book.}
\enumsentence{Srini bought Beth a book.}

In sentence (\ex{-1}), the verb {\it buy} subcategorizes for a direct object NP.
The elementary tree anchored by {\it buy} is shown in
Figure~\ref{subcat-trees}(a) and includes nodes for the NP complement of {\it
buy} and for the NP subject.  In addition to this declarative tree structure,
the tree family also contains the trees that would be related to each other
transformationally in a movement based approach, i.e passivization,
imperatives, wh-questions, relative clauses, and so forth.  Sentence (\ex{0})
shows that {\it buy} also subcategorizes for a double NP object.  This means
that {\it buy} also selects the double NP object subcategorization frame, or
tree family, with its own set of transformationally related sentence
structures.  Figure~\ref{subcat-trees}(b) shows the declarative structure for
this set of sentence structures.

\begin{figure}[ht]
\centering
\begin{tabular}{ccc}
{\psfig{figure=ps/compl-adj-files/alphanx0Vnx1_bought.ps,height=1.8in}} & 
\hspace*{0.5in} & 
{\psfig{figure=ps/compl-adj-files/alphanx0Vnx1nx2_bought.ps,height=1.8in}}\\
(a) & \hspace*{0.5in} & (b) \\ 
\end{tabular}
\caption{Different subcategorization frames for the verb {\it buy}}
\label{subcat-trees}
\end{figure}


\section{Complements and Adjuncts}
\label{compl-adj}

Complements and adjuncts have very different structures in the XTAG grammar.
Complements are included in the elementary tree anchored by the verb that
selects them, while adjuncts do not originate in the same elementary tree as
the verb anchoring the sentence, but are instead added to a structure by
adjunction.  The contrasts between complements and adjuncts have been
extensively discussed in the linguistics literature and the classification of a
given element as one or the other remains a matter of debate (see
\cite{rizzi90},
\cite{larson88}, \cite{jackendoff90}, \cite{larson90}, \cite{cinque90}, 
\cite{obernauer84}, \cite{lasnik-saito84}, and \cite{chomsky86}).  The guiding
rule used in developing the XTAG grammar is whether or not the sentence is
ungrammatical without the questioned structure.\footnote{Iteration of a
structure can also be used as a diagnostic: {\it Srini bought a book at the
bookstore on Walnut Street for a friend at work}.} Consider the following
sentences:

\enumsentence{Srini bought a book.}
\enumsentence{Srini bought a book at the bookstore.}
\enumsentence{Srini arranged for a ride.}
\enumsentence{$\ast$Srini arranged.}

Prepositional phrases frequently occur as adjuncts, and when they are used as
adjuncts they have the tree structure shown in Figure~\ref{compl-adjunct}(a).
This adjunction tree would adjoin into the tree shown in
Figure~\ref{subcat-trees}(a) to generate sentence (\ex{-2}).  There are verbs,
however, such as {\it arrange}, {\it hunger} and {\it differentiate}, that take
prepositional phrases as complements.  Sentences (\ex{-1}) and (\ex{0}) clearly
show that the prepositional phrase are not optional for these verbs.  For these
sentences, the prepositional phrase will be an initial tree (as shown in
Figure~\ref{compl-adjunct}(b)) that substitutes into an elementary tree, such
as the one anchored by the verb {\it arrange} in Figure~\ref{compl-adjunct}(c).

\begin{figure}[ht]
\centering
\begin{tabular}{ccccc}
{\psfig{figure=ps/compl-adj-files/betavxPnx_at.ps,height=1.8in}} &
\hspace*{0.5in} &
{\psfig{figure=ps/compl-adj-files/alphaPXPnx_for.ps,height=1.3in}} &
\hspace*{0.5in} & 
{\psfig{figure=ps/compl-adj-files/alphanx0Vpnx1_arranged.ps,height=1.8in}}\\
(a) & \hspace*{0.5in} & (b) & \hspace*{0.5in} & (c) \\ 
\end{tabular}
\caption{Trees illustrating the difference between Complements and Adjuncts}
\label{compl-adjunct}
\label{2;1,9}
\end{figure}


Virtually all parts of speech, except for main verbs, function as both
complements and adjuncts in the grammar.  More information is available in this
tech report on various parts of speech as complements: adjectives (e.g. section
\ref{nx0Va1-family}), nouns (e.g.  section~\ref{nx0Vnx1-family}), and
prepositions (e.g. section~\ref{nx0Vpnx1-family}); and as adjuncts: adjectives
(section~\ref{adj-modifier}), adverbs (section~\ref{adv-modifier}), nouns
(section~\ref{noun-modifier}), and prepositions (section~\ref{prep-modifier}).

\section{Non-S constituents}

Although sentential trees are generally considered to be special cases in any
grammar, insofar as they make up a `starting category', it is the case that any
initial tree constitutes a phrasal constituent.  These initial trees may have
substitution nodes that need to be filled (by other initial trees), and may be
modified by adjunct trees, exactly as the trees rooted in S.  Although grouping
is possible according to the heads or anchors of these trees, we have not found
any classification similar to the subcategorization frames for verbs that can
be used by a lexical entry to `group select' a set of trees.  These trees are
selected one by one by each lexical item, according to each lexical item's
idiosyncrasies.  The grammar described by this technical report places them
into several files for ease of use, but these files do not constitute tree
families in the way that the subcategorization frames do.

\section{Case and PRO}

GB theory proposes the ``case filter'' as a requirement on
S-structure\footnote{There are certain problems with applying the case
filter as a requrement at the leveel of S-structure.  These issues are
not crucial to the discussion of the English LTAG implementation of
case so will not be discussed here.  Interested readers are referred
to \cite{LasnikUriagereka}.}

\begin{verse}
\underline{Case Filter}
Every overt NP must be assigned abstract case
\end{verse}

Abstract case is taken to be universal.  Languages with rich
morphological case marking such as Latin and languages like English
with very limited morphological case marking are presumed to both have
full systems of abstract case and differ only in the extent of
morphological realization.

In GB theory, abstract case is assignedd to NP's by various case
assigners.  For English, one instance of abstract case, accusative, is
assigned by propositions and verbs.  Nominative case is assigned in
English by finite INFL.  

The notion of abstract case and the case filter are useful in
accounting for a number of phenomena including the distribution of
over nominative and accusative case, the distributions of overt NP's
and empty categories and case provides motivation for movement.  

The English LTAG grammar adopts the notion of case and the case filter
for many of the same reasons argued for in the GB literature.
However, because TAG and unification differ considerabley form the
mechanisms usually assumed discussions of case in GB, the English LTAG
grammar implementation of case appears somewhat different than the
system outlined in the GB literature.

\subsection{Case assignment in the English LTAG grammar}

Case assignment occurs in one of two ways in the English LTAG grammar.
Case assigners can anchor a tree in which the case value is built into
certain NP nodes.  This is how accusative case assignment is handled
for objects of verbs and prepositions.  Nominative case is assigned
through co-indexing of the value of the case feature of the subject NP
and the assign-case value of the verb.  In lexicalized TAG tree
structures and features values are attached to fully inflected lexical
items.  There is no categorical division between INFL and V.  Tense
and agreement are expressed as features of verbs and not as separate
nodes in the derivation structure.

