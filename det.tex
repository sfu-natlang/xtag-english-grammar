\chapter{Determiners and Noun Phrases}
\label{det-comparitives}

{\sc NB: The determiner analysis in the XTAG grammar has changed, and
this section will soon be updated with the current analysis. Briefly,
in the new analysis determiners adjoin to NPs; this means that there
is only one NP tree for all NPs ($\alpha$NXN). The features still work
exactly as described here.}


Previous approaches to syntactic determiner ordering (e.g.\ \cite{quirk85})
have simply divided determiners into subcategories (predet, det, postdet).
This type of approach is inadequate because it allows ungrammatical sequences
like {\it $\ast$all what no}, and misses the finer distinctions among
particular determiners. These finer distinctions are modeled very naturally in
a lexicalized grammar formalism such as FB-LTAG in which pieces of syntactic
structure and features representing linguistic properties are associated with
individual lexical items.

In the English XTAG grammar,\footnote{This chapter is a shortened version of
\cite{HockeyEgedi94}, which contains a more extensive discussion of this 
analysis.} there are two kinds of basic noun phrases (NP), those that take
determiner phrases\footnote{Henceforth DetP's, not to be confused with DP's as
in the DP Hypothesis.} and those that do not.  Nouns that take (or require)
determiners have a DetP substitution site. Complex DetP's are formed by having
determiners adjoin onto each other. There are two basic determiner trees: an
initial tree and an auxiliary tree.  Figure~\ref{det-trees} shows the initial
and auxiliary trees anchored by the determiner {\it these}.  Since any single
determiner can function as a complete DetP,\footnote{By definition.  Our main
criteria in classifying something as a determiner was that it be able to stand
alone with a noun to form an NP.} every determiner selects the initial tree in
Figure~\ref{det-trees}(a).  Determiners that can modify other determiners also
select the auxiliary tree in Figure~\ref{det-trees}(b).

\begin{figure}[hbt]
\centering
\begin{tabular}{ccc}
{\psfig{figure=ps/det-files/alphaD-these.ps,height=12.3cm}} & 
\hspace{1.0in}&
{\psfig{figure=ps/det-files/betaD-these.ps,height=12.3cm}}\\
(a)&&(b)
\end{tabular}
\caption{Determiner Trees with Features: $\alpha$DXD (a) and $\beta$Ddx (b)}
\label{det-trees}
\end{figure}

The current grammar includes a DetP substitution node in the NP but having
determiners adjoin on has also been proposed in the literature
(\cite{Abeille90:TAG}).  The correct ordering of determiners and reasonable
coverage is possible with either approach. In fact, the core of our analysis is
based on the features and would essentially be the same with adjoined
DetP's. We are currently considering whether there would be any compelling
advantages of an adjunction analysis for the XTAG grammar.

In the XTAG grammar, features are crucial to ordering determiners correctly.
We have identified eight features which are sufficient to order the
determiners.  These features are: {\bf definiteness}, {\bf quantity}, {\bf
cardinality}, {\bf genitive}, {\bf decreasing}, {\bf constancy}, {\bf wh} and
{\bf agr}.  These features have all been previously proposed as semantic
properties of determiners.  The semantic definitions underlying the features
are given below.

\begin{description}

\item[Definiteness:] Possible Values [+/--]. \\
A function f is definite iff f is non-trivial and whenever
f(s)~$\neq~\emptyset$ then it is always the intersection of one or
more individuals.  \cite{KeenanStavi86:LP}

\item[Quantity:]  Possible Values [+/--]. \\
If A and B are sets denoting an NP and associated predicate, respectively; E is
a domain in a model M, and F is a bijection from M$_{1}$ to M$_{2}$, then we
say that a determiner satisfies the constraint of quantity if
Det$_{E_{1}}$AB~$\leftrightarrow$~Det$_{E_{2}}$F(A)F(B). \cite{Partee90:BK}

\item[Cardinality:]  Possible Values [+/--]. \\
A determiner D is cardinal iff D $\in$ cardinal numbers~$\geq$~1.

\item[Genitive:]  Possible Values [+/--]. \\
Possessive pronouns and the possessive morpheme ({\it 's}) are marked {\bf
gen+}; all other nouns are {\bf gen--}.

\item[Decreasing:]  Possible Values [+/--]. \\
A set of Q properties is decreasing iff whenever s$\leq$t and t$\in$Q then
s$\in$Q. A function f is decreasing iff for all properties f(s) is a decreasing
set.

A non-trivial NP (one with a Det node) is decreasing iff its denotation in any
model is decreasing. \cite{KeenanStavi86:LP}

\item[Constancy:] Possible Values [+/--]. \\
If A and B are sets denoting an NP and associated predicate, respectively, and
E is a domain, then we say that a determiner displays constancy if
(A$\cup$B)~$\subseteq$~E~$\subseteq$~E$^{\prime}$ then
Det$_{E}$AB~$\leftrightarrow$~Det$_{E^{\prime}}$AB. Modified from
\cite{Partee90:BK}

\item[Wh:]  Possible Values [+/--]. \\
Interrogative determiners are {\bf wh+}; all other determiners are
{\bf wh--}. 

\item[Agreement:] Possible Values [3sg, 3pl, 3sgpl]. \\
Although English does not have the morphological marking of determiners for
case, gender or number, we hold that most determiners in English are
semantically marked for number.

\end{description}

The initial determiner tree in Figure~\ref{det-trees}(a) shows the appropriate
feature values for the determiner {\it these}, while Table~\ref{det-values}
shows the corresponding feature values of several other common determiners.

%\tiny
\begin{table}[hbt]
\centering
\begin{tabular}{|l||c|c|c|c|c|c|c|c|}
\hline
Det&defin&quan&card&gen&wh&decreas&const&agr\\
\hline
\hline
all&$+$&$+$&$-$&$-$&$-$&$-$&$+$&3pl\\
this&$+$&$-$&$-$&$-$&$-$&$-$&$+$&3sg\\
that&$+$&$-$&$-$&$-$&$-$&$-$&$+$&3sg\\
what&$+$&$-$&$-$&$-$&$+$&$-$&$+$&3sgpl\\
the&$+$&$-$&$-$&$-$&$-$&$-$&$+$&3sgpl\\
every&$+$&$+$&$-$&$-$&$-$&$-$&$+$&3sg\\
each&$+$&$+$&$-$&$-$&$-$&$-$&$+$&3sg\\
any&$-$&$+$&$-$&$-$&$-$&$-$&$+$&3sg\\
a&$-$&$+$&$-$&$-$&$-$&$-$&$+$&3sg\\
no&$+$&$+$&$-$&$-$&$-$&$-$&$+$&3sgpl\\
few&$-$&$+$&$-$&$-$&$-$&$+$&$-$&3pl\\
many&$-$&$+$&$-$&$-$&$-$&$-$&$-$&3pl\\
GEN&$+$&$-$&$-$&$+$&$-$&$-$&$+$&\\
CARD&$+$&$+$&$+$&$-$&$-$&$-$&$+$&3pl\footnotemark\ \\
PART&$-$&&$-$&$-$&$-$&$-$&$+$&\\
\hline
\end{tabular}
 \caption{Determiner Features}
\label{det-values}
\end{table}\addtocounter{footnote}{0}\footnotetext{{except {\it one} which is 3sg.}} 

%\normalsize

In addition to the features that represent their own properties, determiners
that select the auxiliary tree have features to represent the selectional
restrictions these determiners impose on the determiners they modify.  The
selectional restriction features of a determiner appear on the DetP foot node
of the auxiliary tree that the determiner anchors.  The DetP$_{f}$ node in the
auxiliary tree in Figure~\ref{det-trees}(b) shows the selectional feature
restriction imposed by {\it these},\footnote{In addition to this tree, {\it
these} would also anchor another auxiliary tree that adjoins onto {\bf
$<$card$>$=+} determiners.} while Table~\ref{det-ordering} shows the
corresponding selectional feature restrictions of several other determiners.

%\tiny
\begin{table}[hbt]
\vspace*{-2mm}
\centering
\begin{tabular}{|l||c|c|c|c|c|c|c|c|}
\hline
Det&defin&quan&card&gen&wh&decreas&const&agr\\
\hline
\hline
all&$+$&$-$&$-$&&$-$&&&\\
&&&$+$&&&&&\\ \hline
this&$-$&&&&$-$&$+$&&\\
&&&$+$&&&&&\\ \hline
that&$-$&&&&$-$&$+$&&\\
&&&$+$&&&&&\\ \hline
what&$-$&&&&$-$&$+$&&\\
&&&$+$&&&&&\\ \hline
the&$-$&&&&$-$&&$-$&\\ 
&&&$+$&&&&&\\ \hline
every&$-$&&&&$-$&$+$&&\\
&&&$+$&&&&&\\ \hline
each&$-$&&&&$-$&$+$&&\\
&&&$+$&&&&&\\ \hline
any&$-$&&&&$-$&$+$&&\\ 
&&&$+$&&&&&\\ \hline
a&$-$&&&&$-$&$+$&&\\ \hline
many&\multicolumn{8}{c|}{only nouns}\\ \hline
no&\multicolumn{8}{c|}{only nouns}\\ \hline
GEN&\multicolumn{8}{c|}{only nouns}\\ \hline
CARD&\multicolumn{8}{c|}{only nouns}\\ \hline
PART&&&&&$-$&&&\\ \hline
\end{tabular}
\caption{Selectional Restrictions Imposed by Determiners}
\label{det-ordering}
\end{table}

%\normalsize

\section{Multi-word Determiners}
The system recognizes the multi-word determiners {\it a few} and {\it many a}.
The features for these constituents are located at the D parent node of both
components (see Figure~\ref{multi-det-tree}).  We chose to represent these 
determiners as multi-word constituents because neither determiner retains the 
same set of features as either of its parts.  For example, the determiner 
{\it a} is 3sg and {\it few} is decreasing, while {\it a few} is 3pl and 
increasing.  Additionally, {\it many} is 3pl and {\it a} displays constancy, 
but {\it many a} is 3sg and does not display constancy.  Example sentences
appear in (\ex{1})-(\ex{2}).

\begin{itemize}
\item{Multi-word Determiner}
\enumsentence{{\bf a few} teaspoons of sugar should be adequate .}
\enumsentence{{\bf many a} man has attempted that stunt, but none have
succeeded .}

\end{itemize}

\begin{figure}[htb]
\centering
\begin{tabular}{cc}
{\psfig{figure=ps/det-files/betaDDnx.ps,height=3.0in}}
\end{tabular}\\
\caption{Multi-word Determiner tree:  $\beta$DDnx}
\label{multi-det-tree}
\end{figure} 

\section{Wh and Agr Features}
\label{agr-section}
A determiner with a {\bf $<$wh$>$=+} feature is always the leftmost
determiner since no determiners can adjoin onto it.  The presence of a wh+
determiner makes the entire NP wh+, so this feature is always passed through to
the NP node, unlike other features which are considered internal to the
determiner system.

The {\bf $<$agr$>$} feature functions differently from most of the features in
the determiner sequencing system.  Notice that in the auxiliary tree in
Figure~\ref{det-trees}(b), the {\bf $<$agr$>$} feature is the only feature not
passed from the D node to the root DetP node, but is passed instead from the
foot DetP to the root DetP.  In the determiner system, the {\bf $<$agr$>$}
feature is generally propagated from the rightmost determiner (i.e.\ the one
closest to the noun), although some adjoining determiners require that they
also agree with that determiner.  This distinction is captured in XTAG by
having each determiner specify in its lexical entry whether or not its
agreement feature is passed to the root DetP (i.e.\ from the D node to the
DetP$_{r}$ node).

\section{Genitive Constructions}

There are two kinds of genitive constructions: genitive pronouns, and
genitive NP's (which have an explicit genitive marker, {\it 's},
associated with them).  It is clear from examples such as {\it her
favorite artist prefers oils\/} vs. {\it $\ast$favorite artist prefers
oils\/} that genitive pronouns function as determiners and as such,
they sequence with the rest of the determiners.  The features for the
genitives are the same as for other determiners, and are given in
Table~\ref{det-values}.  No {\bf $<$agr$>$} is specified however,
since the number and person of the genitive will depend on its
particular form (e.g.\ {\it my} vs. {\it their}).  Genitives are not
required to agree with either the determiners or the nouns that they
modify.

%Figure of alphaDXnxG-features and betanxGdx-features here 

\begin{figure}[hbt]
\centering
\begin{tabular}{ccc}
{\psfig{figure=ps/det-files/alphaDXnxG-features.ps,height=11.3cm}} & 
\hspace{1.0in}&
{\psfig{figure=ps/det-files/betanxGdx-features.ps,height=11.8cm}}\\
(a)&&(b)
\end{tabular}
\caption{Initial: $\alpha$DXnxG (a) and Auxiliary: $\beta$nxGdx (b) Genitive Determiner Trees}
\label{gen-trees}
\end{figure}


Genitive NP's are particularly interesting because they are potentially
recursive structures.  Complex NP's can easily be embedded in a determiner
phrase.

\enumsentence{[[John]'s friend from high school]'s mother came to
town .}

There are two things to note in sentence~(\ex{0}).  One is that in embedded
NP's, the genitive morpheme comes at the end of the NP phrase, even if the head
of the NP is at the beginning of the phrase.  The other is that the determiner
of an embedded NP can also be a genitive NP, hence the possibility of recursive
structures.

In the XTAG grammar, the genitive marker, {\it 's}, is separated from the
lexical item that it is attached to and given its own category (G).  In this
way, we can allow the full complexity of NP's to come from the existing NP
system, including any recursive structures.  The two trees in
Figure~\ref{gen-trees} demonstrate how easily the complexity of genitive NP's
are captured in XTAG.  As with the standard determiner trees, there are two
trees - one for the determiner that stands alone and one for a determiner that
adjoins onto another.


\section{Partitive Constructions}

Partitive constructions (e.g.\ {\it some kind of}, {\it all of\/}) are
another type of complex determiner construction.  Partitive constructions
interact with other determiners.  Since they can modify the noun itself ({\it
[a certain kind of] machine}), or modify other determiners ({\it [some parts
of] these machines}), the partitive construction has both an initial and
auxiliary tree that are anchored by the preposition {\it of}. The partitive
trees are shown in Figure~\ref{partitive-trees}.

\begin{figure}[hbt]
\centering
\begin{tabular}{ccc}
{\psfig{figure=ps/det-files/alphaDXnxP.ps,height=11.5cm}} & 
\hspace{1.0in}&
{\psfig{figure=ps/det-files/betanxPdx.ps,height=12.0cm}}\\
(a)&&(b) \\
\end{tabular}
\caption{Initial: $\alpha$DXnxP (a) and Auxiliary: $\beta$nxPdx (b) Partitive trees}
\label{partitive-trees}
\end{figure}




