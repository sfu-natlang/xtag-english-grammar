\section{Small Clauses, Raising Verbs and the Copula}
Small clauses offer examples of clauses that are headed by adjectives, nouns or prepositions rather than verbs.  For example, the embedded clause in (\ex{1}) is headed by the adjective {\it frugal}.

\enumsentence{I consider John frugal.}


The analysis developed for the English LTAG grammar by Caroline Heycock provides a uniform analysis of the  embedded clause in (\ex{0}) and the predicative sentences in (\ex{1}) and (\ex{2})

\enumsentence{John is frugal.}
\enumsentence{John seems frugal.}

All three cases use the elementary tree $\alpha$nx0Ax1 tree shown in
figure \ref{nx0Ax1}. \\

\begin{figure}[htb]
\centerline{
\psfig{figure=/mnt/linc/extra/xtag/work/beth/ps/alphanx0Ax1.ps,height=10.0cm}}
\caption{\label{nx0Ax1} Tree:  $\alpha$nx0Ax1}
\end{figure}


This tree is anchored by the adjective {\it frugal\/} and contains
only the subject, {\it John\/} and the anchor, exactly as in the
embedded clause in (\ex{-2}).  For examples (\ex{-1}) and (\ex{0})
either {\it seem\/} or {\it is\/} adjoins on to $\alpha$nx0Ax1.  To
cover the range of possible small clauses, the following tree families
contain trees anchored by either nouns, prepositions or adjectives:

\begin{description}
\item Tnx0Nx1
\item Tnx0Px1
\item Tnx0Ax1
\item Tnx0dNx1
\item Ts0Nx1
\item Ts0Px1
\item Ts0Ax1
\item Ts0dNx1
\end{description}

The two tree families anchored by nouns, Tnx0Nx1 and Tnx0dNx1, are needed to handle noun anchored cases with and without determiners as in (\ex{1}) and (\ex{2})

\enumsentence{He is a fireman.}
\enumsentence{They are firemen.}

The trees in these tree families differ from verb-anchored trees in
the value of their mode feature as well as the category of their
anchor. As can be seen in (\ref{nx0Ax1}), these small clause trees have
the VP top feature {\bf $<$mode = nom/prep$>$}.  This value for mode serves two
major purposes, 
\begin{itemize}
\item restricting small clause complements to the
small set of verbs like consider that select for them; and, 
\item forcing adjunction of the
appropriate form of {\it be\/} or of a raising verb like {\it seem\/}  in main
clauses and in sentential complements of verbs that do not select for small
clauses, as in (\ex{-5})-(\ex{-3}) and  (\ex{-1})-(\ex{2}).
\end{itemize}

\enumsentence{Mary thinks John is happy.}
\enumsentence{Mary thinks John seems happy.}

Adjunction of {\it be\/}  or a raising verb is forced in main clauses by
the conflict between the {\bf nom/prep} value of {\bf mode} on VP top and the ind
value of {\bf mode} enforced on S-top by the parser.  A similar conflict is
created for embedded sentences by the verbs that select them as
complements and allow only {\bf mode} values other than {\bf nom/prep}.  For
example $\beta$nx0Vs1 anchored by think in (\ex{-1}) and (\ex{0}),
requires S1 to be {\bf $<$mode = ind$>$} as shown in Figure \ref{nx0Vs1}.

\begin{figure}[ht]
\centering
\psfig{figure=/mnt/linc/extra/xtag/work/beth/ps/betanx0Vs1.ps,height=10.0cm}
\caption{\label{nx0Vs1} Tree:  $\beta$nx0Vs1}
\end{figure}
