
\chapter{Imperatives}
\label{imperatives}



Imperatives in English do not require overt subjects.  The subject in
imperatives is second person, i.e.\ {\it you}, whether it is overt or
not, as is clear from the verbal agreement and the interpretation.
Imperatives with overt subjects can be  parsed using the trees already
needed for declaratives.  The imperative cases in which the subject is
not overt are handled by the imperative trees discussed in this section.

The imperative trees in English XTAG grammar are identical to the declarative
tree except that the NP$_{0}$ subject position is filled by an $\epsilon$, the
NP$_{0}$ {\bf $<$agr~pers$>$} feature is set in the tree to the value {\bf 2nd}
and the {\bf $<$mode$>$} feature on the root node has the value {\bf imp}.  The
value for {\bf $<$agr~pers$>$} is hardwired into the epsilon node and insures
the proper verbal agreement for an imperative.  The {\bf $<$mode$>$} value of
{\bf imp} on the root node is recognized as a valid mode for a matrix clause.
The {\bf imp} value for {\bf $<$mode$>$} also allows imperatives to be blocked
from appearing as embedded clauses.  Figure \ref{alphaInx0Vnx1} is the
imperative tree for the transitive tree family.

\begin{figure}[htbp]
\centering{
\begin{tabular}{c}
\psfig{figure=ps/imperatives-files/alphaInx0Vnx1.ps,height=6in}
\end{tabular}
}
\caption{Transitive imperative tree: $\alpha$Inx0Vnx1}
\label{alphaInx0Vnx1}
\label{2;11,1}
\end{figure}

\section{Negative Imperatives}
\label{neg-imp}

All Negative imperatives in English require {\it do}-support even for those
that are formed with {\it be} and auxliary {\it have}.

\enumsentence{Dont' leave.}
\enumsentence{*Not leave.}

\enumsentence{Do not open the window!}
\enumsentence{*Not open the window!}

\enumsentence{Do not be talking so loud!}
\enumsentence{*Not be talking so loud!}

\enumsentence{Don't have eaten everything before the guests arrive!}
\enumsentence{*Not have eaten everything before the guests arrive!}

In English XTAG, negative imperatives receive similar structural analysis
as {\it yes-no} questions, as in \cite{potsdamdiss97} and \cite{handiss}.
That is, {\it do} and {\it don't} in negative imperatives are treated as an
instance of {\it do}-support which adjoin to a clause.  The crucial
strucural evidence for our analysis is that when there is an overt subject
in negative imperatives formed with {\it don't}, the subject must follow
{\it don't}, just as it does in {\it yes-no} questions.

\enumsentence{Don't you worry!}
\enumsentence{Don't you move!}


\enumsentence{Don't you like carrots?}
\enumsentence{Didn't you finish your paper yet?}


{\it Do}-support in negative imperatives is handled by the elementary tree
$\beta$IVs, as shown in Figure \ref{fig:doimp}, which anchors {\it do} or
{\it don't}.  This tree adjoins onto the root node of the imperative tree.
The feature $\langle$ mode $\rangle$ =imp on the S foot node on $\beta$IVs
restricts this tree to adjoin only to imperative trees. Further, the S root
node of $\beta$IVs is specified with $\langle$ mode $\rangle$ =imp, which
allows imperatives with {\it do}-support to be blocked from appearing as
embedded clauses.
 
\begin{figure}[htbp]
\centering
\begin{tabular}{ccc}
{\psfig{figure=ps/imperatives-files/betaIVs-do.ps,height=10cm}} &
{\ } & 
{\psfig{figure=ps/imperatives-files/betaIVs-dont.ps,height=10cm}} \\
$\beta$IVs[do] & {\ } & $\beta$IVs[don't] 
\end{tabular}
\caption{Trees anchored by {\it do} and {\it don't}}
\label{fig:doimp}
\end{figure}
 

In negative imperatives formed with {\it don't}, the $\beta$IVs[don't] tree in
Figure \ref{fig:doimp} adjoins to the root node of the imperative tree. The
derived tree for the negative imperative {\it Don't leave!} is given in
Figure \ref{fig:dont-leave}.

\begin{figure}[htbp]
  \begin{center} \leavevmode \psfig{figure=ps/imperatives-files/dont-leave.ps,height=8cm}
  \end{center}
  \caption{Derived tree for {\it Don't leave!}}
\label{fig:dont-leave}
\end{figure} 


In negative imperatives formed with {\it do not}, the $\beta$IVs[do] tree
in Figure \ref{fig:doimp} adjoins to the root node of the imperative tree
and the $\beta$NEGvx tree that anchors {\it not} as represented in Figure
\ref{fig:not} adjoins to VP node of the imperative tree.

\begin{figure}[htbp]
  \begin{center} \leavevmode \psfig{figure=ps/imperatives-files/betaNEGvx-not.ps,height=10cm}
  \end{center}
  \caption{Tree anchored by {\it not}}
\label{fig:not}
\end{figure} 
The two derived trees for the negative imperative {\it Do not leave!} are
given in Figure \ref{fig:do-not-leave}.  


\begin{figure}[htbp]
 \centering
\begin{tabular}{ccc}
{\psfig{figure=ps/imperatives-files/do-not-leave1.ps,height=8cm}} &
{\ } & 
{\psfig{figure=ps/imperatives-files/do-not-leave2.ps,height=8cm}} \\
(a)  & {\ } & (b)
\end{tabular}

\caption{Derived trees for {\it Do not leave!}}
\label{fig:do-not-leave}
\end{figure} 

Note that trees in Figure \ref{fig:dont-leave} and Figure
\ref{fig:do-not-leave} have an empty verb.  This is due to the feature
$\langle$ disp-const $\rangle$ in $\beta$IVs.  This feature ensures that
when $\beta$IVs is adjoined to an elementary tree, $\beta$Vvx that anchors
an empty verb must also adjoin onto the VP of that same elementary tree.
This tree is represented in Figure \ref{fig:epsilon}.  The empty verb
represents the originating position of {\it do} and {\it don't}.  This
mechanism is also used in interrogatives that have subject-verb inversion
to simulate auxiliary verb movement.  For more on this, see
Section~\ref{auxs.dosupport} on {\it do}-support and inversion.

\begin{figure}[htbp]
  \begin{center} \leavevmode \psfig{figure=ps/imperatives-files/betaVvx-epsilon.ps,height=5cm}
  \end{center}
  \caption{$\beta$Vvx[$\epsilon$]}
\label{fig:epsilon}
\end{figure} 

If {\it do} in negative imperatives is in the same position as {\it do} in
{\it yes-no} questions, the fact that an overt subject cannot intervene
between {\it do} and {\it not} is puzzling.

\enumsentence{Do not open the window.}
\enumsentence{*Do you not open the window.}

We adopt the account given in \cite{akmajian84} that this fact is not due
to syntax but due to an intonational constraint in imperatives.  He argues
that (i) when an imperative sentence has an overt subject, the subject must be
the only intonation center preceding the verb phrase and (ii) that in
negative imperatives with {\it do} and {\it not}, either {\it do} or {\it
not} must be the intonation center.  These two contraints conspire to rule
out {\it do not} imperatives with an overt subject.

Another puzzling fact that needs to be explained  is that in negative
imperatives even {\it be} and auxiliary {\it have} require {\it
do}-support, while it is prohibited in negative declaratives and negative
questions.  

\enumsentence{He isn't talking loud.}
\enumsentence{*He doesn't be talking loud.}

\enumsentence{Isn't he talking loud?}
\enumsentence{*Doesn't he be talking loud?}

This fact does not pose a problem for the XTAG analysis of negative
imperatives if we adopt the line of approach given in \cite{handiss}.  She
points out that while declaratives and questions are tensed, imperatives
are not, and argues that this is exactly why negative imperatives require
{\it do}-support even for {\it be} and auxiliary {\it have}.  Assuming a
clause structure in which CP dominates IP and IP dominates VP (for
expository purposes), she argues that it is the tense features in I$^0$
that attract {\it be} and auxiliary {\it have} in declaratives and
questions.  In declaratives, {\it be} or auxiliary {\it have} moves to and
stays in I$^0$, and in questions, once {\it be} or auxiliary {\it have}
moves to I$^0$, they further move to C$^0$.  Moreover, main verbs cannot
move at all to I$^0$ in the overt syntax.  Instead, they undergo movement
at LF.  But negation blocks LF movement and so as a last resort {\it do} is
inserted in I$^0$ to support INFL.  In imperatives, I$^0$ does not have
tense features and so it cannot attract {\it be} and auxiliary {\it have}.
Thus, {\it be} and auxiliary {\it have} as well as main verbs undergo
movement at LF in imperatives.  And so in negative imperatives, since
negation blocks LF verb movement, {\it do} is inserted in I$^0$ as a last
resort device even for {\it be} and auxiliary {\it have} and it further
moves to C$^0$ in the overt syntax.


Given our analsis of negative imperatives, if the subject precedes {\it do}
or {\it don't}, we are forced to treat it as a vocative and not a
sentential subject.  Vocatives are considered to be outside the clause
structure and does not have any structural relation with any element in the
clause.

\enumsentence{You don't drink the water.}
\enumsentence{You do not leave the room.}

Given the fact that the imperatives in \ex{-1} and \ex{0} seem to be
degraded unless there is an intonational break between {\it you} and the
rest of the sentence, treating {\it you} as a vocative seems to be the
correct approach.  Currently, XTAG grammar does not handle vocatives.
  

Another case where imperatives have {\it do}-support is emphatic
imperatives.

\enumsentence{Do open the window.}
\enumsentence{Do show up for the lecture.}

In English XTAG grammar, {\it do} in emphatic imperatives is treated just
as {\it do} in negative imperatives.  It is adjoined to the clause.  Again,
the crucial evidence for this analysis comes from word order facts.  When
emphatic imperatives have an overt subject, it must follow {\it do}.

\enumsentence{Do somebody bring me some water.}
\enumsentence{Do at least some of you show up for the lecture.}

One remaining task for imperatives is to handle those with overt subjects.
The type of overt subjects allowed in imperatives are restricted: 2nd
person pronouns and some quantified noun phrases. Currently, the English
XTAG grammar only has imperative trees with empty subjects.  Hence,
imperatives with overt subjects as in \ex{-1} and \ex{0} cannot be parsed.

%\bibliography{diss} 
%\bibliographystyle{linquiry} 

