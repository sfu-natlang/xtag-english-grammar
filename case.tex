\section{Case Assignment}
\subsection{Approaches to Case}
\subsubsection{Case in GB theory}

GB (Government and Binding) theory proposes the following ``case filter'' as a
requirement on S-structure.\footnote{There are certain problems with applying
the case filter as a requirement at the level of S-structure.  These issues are
not crucial to the discussion of the English LTAG implementation of case and so
will not be discussed here.  Interested readers are referred to
\cite{lasnik-uriagereka88}.}

\begin{verse}
\underline{Case Filter}
Every overt NP must be assigned abstract case.
\end{verse}

Abstract case is taken to be universal.  Languages with rich morphological case
marking, such as Latin, and languages with very limited morphological case
marking, like English, are all presumed to have full systems of abstract case
that differ only in the extent of morphological realization.

In GB, abstract case is assigned to NPs by various case assigners, namely
verbs, prepositions, and INFL.  Verbs and prepositions are said to assign
accusative case to NPs that they govern, and INFL assigns nominative case to
NPs that it governs.  These governing categories are constrained in where they
can assign case by means of `barriers' based on `minimality conditions',
although these are relaxed in `exceptional case marking' situations.  The
details of the GB analysis are beyond the scope of this technical report, but
see \cite{chomsky86} for the original analysis or \cite{haegeman91} for an
overview.  Let it suffice for us to say that the notion of abstract case and
the case filter are useful in accounting for a number of phenomenon including
the distribution of nominative and accusative case, and the distribution of
overt NPs and empty categories (such as PRO).

\subsubsection{Minimalism and Case} 

A major conceptual difference between GB theories and minimalism is that in
minimalism, lexical items carry their features with them rather than being
assigned their features based on the nodes that they end up at.  For nouns,
this means that they carry case with them, and that case is 'checked' by
AGR$_s$ or AGR$_o$, which then disappears \cite{chomsky92}.

\subsection{Case in XTAG}

The English XTAG grammar adopts the notion of case and the case filter for
many of the same reasons argued in the GB literature.  However, the English
XTAG grammar implementation of case more closely resembles the treatment in
Chomsky's minimalism framework \cite{chomsky92} than the system outlined in the
GB literature \cite{chomsky86}.  As in minimalism, nouns in the XTAG approach
carry case with them, which is eventually 'checked' against the case values
assigned by the verb during the unification of the feature structures.  Unlike
Chomsky's minimalism, there is no separate AGR nodes; the case checking comes
from the verbs directly.

Most nouns in English do not have separate forms for nominative and accusative
case, and so they are ambiguous between the two.  Pronouns, of course, are
morphologically marked for case, and each carries the appropriate case in its
feature.  Figures \ref{nouns-with-case}a and \ref{nouns-with-case}b show the NP
tree anchored by a noun and a pronoun, respectively, along with the feature
values associated with each word.

\begin{figure*}[ht]
\centering
\rule[.1in]{3.5in}{0.01in} \\
\begin{tabular}{cc}
{\psfig{figure=ps/case-files/alphaNXN_books.ps,height=3.0in}}  &
{\psfig{figure=ps/case-files/alphaNXN_she.ps,height=3.2in}} \\
(a)&(b)\\
\end{tabular}\\
\caption {Lexicalized NP trees with case markings}
\rule[.1in]{3.5in}{0.01in}
\label {nouns-with-case}
\end{figure*}

\subsection{Case Assigners}

\subsubsection{Prepositions}

Case is assigned in the XTAG English grammar by two components - verbs and
prepositions\footnote{{\it For} also assigns case as a complementizer.  See
section \ref{for-complementizer} for more details.}.  Prepositions assign
accusative case ({\bf acc})through their {\bf assign-case} feature, which is
linked directly to the {\bf case} feature of their objects.  Figure
\ref{PXPnx-with-case}a shows a lexicalized preposition tree, while
\ref{PXPnx-with-case}b shows the same tree with the NP tree from
\ref{nouns-with-case}a substituted into the NP position.  Figure
\ref{PXPnx-with-case}c is the tree \ref{PXPnx-with-case}b after unification has
taken place.  Note that the case ambiguity of {\it books} has been resolved to
accusative case.

\begin{figure*}[ht]
\centering
\rule[.1in]{6.0in}{0.01in}
\begin{tabular}{ccc}
{\psfig{figure=ps/case-files/alphaPXPnx_of.ps,height=1.7in}}  &
{\psfig{figure=ps/case-files/NXN-substituted-into-PXPnx.ps,height=3.5in}}  &
{\psfig{figure=ps/case-files/NXN-substituted-into-PXPnx-unified.ps,height=2.8in}} \\
(a)&(b)&(c)\\
\end{tabular}\\
\caption {Assigning case in prepositional phrases}
\rule[.1in]{6.0in}{0.01in}
\label {PXPnx-with-case}
\end{figure*}

\subsubsection{Verbs}
\label{case-for-verbs}
Verbs are the other part of speech in XTAG that can assign case.  Because
XTAG does not distinguish INFL and VP nodes\footnote{See section
\ref{VP-INFL-collapse} for an explanation of how this was done.}, verbs must
provide case assignment on the subject position in addition to the
case assigned to their NP complements.

Assigning case to NP complements is handled by building the case values of the
complements directly into the tree that the case assigner (the verb) anchors.
Figures \ref{S-tree-with-case}a and \ref{S-tree-with-case}b show an S
tree\footnote{Features not pertaining to this discussion have been taken out to
improve readability and to make the trees easier to fit onto the page.} that
would be anchored\footnote{The diamond marker ($\diamond$) indicates the
anchor(s) of a structure if the tree has not yet been lexicalized.} by a
transitive and ditransitive verb, respectively.  Note that the case assignments
for the NP complements are already in the tree, even though there is not yet a
lexical item anchoring the tree.  Since every verb that selects these trees
(and other trees in each respective subcategorization frame) assigns the same
case to the complements, building case features into the tree has exactly the
same result as putting the case feature value in each verb's lexical entry.

\begin{figure*}[ht]
\centering
\rule[.1in]{5.0in}{0.01in}
\begin{tabular}{cc}
{\psfig{figure=ps/case-files/alphanx0Vnx1-case-features.ps,height=2.0in}}  &
{\psfig{figure=ps/case-files/alphanx0Vnx1nx2-case-features.ps,height=2.0in}} \\
(a)&(b)\\
\end{tabular}\\
\caption {Case assignment to NP complements}
\rule[.1in]{5.0in}{0.01in}
\label {S-tree-with-case}
\end{figure*}

The case assigned to the subject position varies with verb form.  Since the
XTAG grammar treats the inflected verb as a single unit rather than dividing
it into INFL and V nodes, case, along with tense and agreement, is expressed in
the features of verbs, and must be passed in the appropriate manner.  The trees
in Figure \ref {S-tree-with-case} show the path of linkages that joins the {\bf
assign-case} feature of the V to the {\bf case} feature of the subject NP.  The
morphological form of the verb determines the value of the {\bf assign-case}
feature.  Figures \ref{lexicalized-S-tree-with-case}a and
\ref{lexicalized-S-tree-with-case}b show the same tree anchored by different
morphological forms of the verb {\it sing}, which give different values for the
assign-case feature\footnote{Again, the feature structures shown have been
restricted to those that pertain to the V/NP interaction.}.

\begin{figure*}[ht]
\centering
\rule[.1in]{5.0in}{0.01in}
\begin{tabular}{cc}
{\psfig{figure=ps/case-files/alphanx0Vnx1_sings-case-features.ps,height=3.2in}}  &
{\psfig{figure=ps/case-files/alphanx0Vnx1_singing-case-features.ps,height=2.9in}} \\
(a)&(b)\\
\end{tabular}\\
\caption {Assigning case according to verb form}
\rule[.1in]{5.0in}{0.01in}
\label {lexicalized-S-tree-with-case}
\end{figure*}

The adjunction of an auxiliary verb onto the VP node breaks the {\bf
assign-case} link from the main V and substitutes a link from the auxiliary
verb instead\footnote{see section \ref{aux-non-inverted} for a more complete
explanation of how this relinking occurs.}. The progressive form of the verb in
Figure \ref{lexicalized-S-tree-with-case}b assigns case {\bf none}, but this is
overridden by the adjunction of the appropriate form of the auxiliary word {\it
be}.  Figure \ref{Vvx-with-case}a shows the lexicalized auxiliary tree, while
\ref{Vvx-with-case}b shows it adjoined into the transitive tree shown in Figure
\ref{lexicalized-S-tree-with-case}b.  The case value passed to the NP is now
{\bf nom} (nominative).

\begin{figure*}[ht]
\centering
\rule[.1in]{5.0in}{0.01in}
\begin{tabular}{cc}
{\psfig{figure=ps/case-files/betaVvx_is-with-case.ps,height=2.1in}}  &
{\psfig{figure=ps/case-files/betaVvx_is-adjoined-into-nx0Vnx1_singing.ps,height=3.5in}} \\
(a)&(b)\\
\end{tabular}\\
\caption {Proper case assignment with auxiliary verbs}
\rule[.1in]{5.0in}{0.01in}
\label {Vvx-with-case}
\end{figure*}


\subsection{PRO in a unification based framework}

Most forms of a verb assign nominative case, although some forms, such as past
participle, assign no case whatsoever.  This is different than assigning case
{\bf none}, as the progressive form of the verb {\it sing} does in Figure
\ref{lexicalized-S-tree-with-case}b.  The distinction of a case {\bf none} from
no case is indicative of a divergence from the standard GB theory.  In GB
theory, the absence of case on an NP means that only PRO can fill that NP.  In
XTAG, the absence of case on an NP means that *any* NP can fill it,
regardless of its case.  This is due to the mechanism of unification, in which
if something is unspecified, it can unify with anything.  Thus we have a
specific case {\bf none} to handle verb forms that in GB theory do not assign
case.  PRO is the only NP with case {\bf none}.  Verbs forms that assign no
case, as the past participle mentioned above, can do so because they cannot
occur without an auxiliary verb which takes care of the case assignment.  Note
that although we are drawn to this treatment by our use of unification for
feature manipulation, \cite{watanabe93} proposes a very similar approach within
Chomsky's minimalist framework for entirely different reasons.
