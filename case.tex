\section{Case and PRO}

GB theory proposes the ``case filter'' as a requirement on
S-structure\footnote{There are certain problems with applying the case
filter as a requrement at the leveel of S-structure.  These issues are
not crucial to the discussion of the English LTAG implementation of
case so will not be discussed here.  Interested readers are referred
to \cite{LasnikUriagereka}.}

\begin{verse}
\underline{Case Filter}
Every overt NP must be assigned abstract case
\end{verse}

Abstract case is taken to be universal.  Languages with rich
morphological case marking such as Latin and languages like English
with very limited morphological case marking are presumed to both have
full systems of abstract case and differ only in the extent of
morphological realization.

In GB theory, abstract case is assignedd to NP's by various case
assigners.  For English, one instance of abstract case, accusative, is
assigned by propositions and verbs.  Nominative case is assigned in
English by finite INFL.  

The notion of abstract case and the case filter are useful in
accounting for a number of phenomena including the distribution of
over nominative and accusative case, the distributions of overt NP's
and empty categories and case provides motivation for movement.  

The English LTAG grammar adopts the notion of case and the case filter
for many of the same reasons argued for in the GB literature.
However, because TAG and unification differ considerabley form the
mechanisms usually assumed discussions of case in GB, the English LTAG
grammar implementation of case appears somewhat different than the
system outlined in the GB literature.

\subsection{Case assignment in the English LTAG grammar}

Case assignment occurs in one of two ways in the English LTAG grammar.
Case assigners can anchor a tree in which the case value is built into
certain NP nodes.  This is how accusative case assignment is handled
for objects of verbs and prepositions.  Nominative case is assigned
through co-indexing of the value of the case feature of the subject NP
and the assign-case value of the verb.  In lexicalized TAG tree
structures and features values are attached to fully inflected lexical
items.  There is no categorical division between INFL and V.  Tense
and agreement are expressed as features of verbs and not as separate
nodes in the derivation structure.
