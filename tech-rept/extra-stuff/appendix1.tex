In this appendix, we will present a detail example illustrating the
various steps the parser performs in deriving the parse for a given
input sentence.

Consider the sentence in \ref{ex}.

\begin{example}
\label{ex}
John sleeps irregularly.	
\end{example}

This sentence is subjected to morphological analysis and
part-of-speech tagging before being input to the parser in the form
shown in \ref{tagged}

\begin{example}
\label{tagged}
Morph/POS tagged output.
\end{example}

The first stage of the parser, tree selection phase, is to use the
morphological information of each word, to select trees from the
lexicon for each word. For the purpose of this example, we assume the
trees shown in Figure~\ref{parse-det} to be the only trees selected.
\begin{figure}
\caption{Trees for the sentence: {\it John sleeps irregularly}.}
\label{parse-det}
\end{figure}
The selected trees are input to the tree grafting stage, which we will
discuss in detail.


-- 










