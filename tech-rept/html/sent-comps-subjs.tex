%talk about ECM and object control verbs 
%being treated the same (at least, sharing a tree family). 
%format for features? angle brackets or no 
 
\chapter{Sentential Subjects and Sentential Complements} 
\label{scomps-section} 
 
In the XTAG grammar, arguments of a lexical item, including 
subjects, appear in the elementary tree anchored by that lexical item.  A 
sentential argument appears as an S node in the appropriate position 
within an elementary tree anchored by the lexical item that selects 
it. This is the case for sentential complements of verbs, prepositions 
and nouns and for sentential subjects. The distribution of 
complementizers in English is intertwined with the distribution of 
embedded sentences.  A successful analysis of complementizers in 
English must handle both the cooccurrence restrictions between 
complementizers and various types of clauses, and the distribution of 
the clauses themselves, in both subject and complement positions. 
 
\section{S or VP complements?} 
 
Two comparable grammatical formalisms, Generalized Phrase Structure 
Grammar (GPSG) \cite{gazdar85} and Head-driven Phrase Structure 
Grammar (HPSG) \cite{PollardSag94:BK}, have rather different 
treatments of sentential complements (S-comps).  They both treat 
embedded sentences as VP's with subjects, which generates the correct 
structures but misses the generalization that S's behave similarly in 
both matrix and embedded environments, and VP's behave quite 
differently.  Neither account has PRO\label{PRO} subjects of 
infinitival clauses-- they have subjectless VP's instead.  GPSG has a 
complete complementizer system, which appears to cover the same range 
of data as our analysis.  It is not clear what sort of complementizer 
analysis could be implemented in HPSG. 
 
Following the standard GB approach, the English XTAG grammar does not 
allow VP complements but treats verb-anchored structures without overt 
subjects as having PRO subjects. Thus, indicative clauses, infinitives 
and gerunds all have a uniform treatment as embedded clauses using the 
same trees under this approach. Furthermore, our analysis is able to 
preserve the selectional and distributional distinction between S's and 
VP's, in the spirit of GB theories, but without having to posit `extra' 
empty categories, such as empty complementizers.\footnote{We do have PRO and NP traces in the grammar.} Consider the alternation between {\it that} and the null complementizer,\footnote{Although we will continue to refer to `null' complementizers, in our analysis this is actually the absence of a complementizer.} shown in sentences \ref{ex:555} and \ref{ex:556}. 
 
\beginsentences
\sitem{He hopes $\emptyset$ Muriel wins .}\label{ex:555} 
\sitem{He hopes that Muriel wins .}\label{ex:556} 
\endsentences

 
 In GB both {\it Muriel wins} in \ref{ex:555} and {\it that Muriel wins} in 
\ref{ex:556} are CPs even though there is no overt complementizer to head the 
phrase in \ref{ex:555}.  Our grammar does not distinguish by category label 
between the phrases that would be labeled in GB as IP and CP.  We label 
both of these phrases S.  The difference between these two levels is the 
presence or absence of the complementizer (or extracted WH constituent), and is 
represented in our system as a difference in feature values (here, of the {\bf $<$comp$>$} feature), and the presence of the additional structure contributed 
by the complementizer or extracted constituent.  This illustrates an important 
distinction in XTAG, that between features and node labels.  Because we have a 
sophisticated feature system, we are able to make fine-grained distinctions 
between nodes with the same label which in another system might have to be 
realized by using distinguishing node labels. 
 
\section{Complementizers and Embedded Clauses in English:  The Data} 
\label{data} 
 
Verbs selecting sentential complements place restrictions on 
their complements, in particular, on the form of the embedded verb 
phrase.\footnote{Other considerations, such as the relationship between the tense/aspect of the matrix clause and the tense/aspect of a complement clause are also important but are not currently addressed in the current English XTAG grammar.}  Furthermore, complementizers are constrained to appear with certain 
types of clauses, again, based primarily on the form of the embedded VP.  For 
example, {\it hope\/} selects both indicative and infinitival complements. With 
an indicative complement, it may only have {\it that\/} or null as possible 
complementizers; with an infinitival complement, it may only have a null 
complementizer.  Verbs that allow wh+ complementizers, such as {\it ask}, can 
take {\it whether} and {\it if} as complementizers.  The possible combinations 
of complementizers and clause types is summarized in Table \ref{facts}. 
 
As can be seen in Table \ref{facts}, sentential subjects differ from 
sentential complements in requiring the complementizer {\it that\/} 
for all indicative and subjunctive clauses.  In sentential complements, 
{\it that\/} often varies freely with a null complementizer, as 
illustrated in \ref{ex:557}-\ref{ex:562}. 
 
\beginsentences
\sitem{Christy hopes that Mike wins .}\label{ex:557} 
\sitem{Christy hopes Mike wins .}\label{ex:558} 
\sitem{Dania thinks that Newt is a liar .}\label{ex:559} 
\sitem{Dania thinks Newt is a liar .}\label{ex:560} 
\sitem{That Helms won so easily annoyed me .}\label{ex:561} 
\sitem{$\ast$Helms won so easily annoyed me .}\label{ex:562} 
\endsentences

 
 
\begin{table}[ht] 
\centering 
\begin{tabular}{|l|llllll|} \hline 
Complementizer:&&that&whether&if&for&null\\ 
\hline 
Clause type&&&&&&\\ 
\hline 
indicative&subject&Yes&Yes&No&No&No\\ 
&complement&Yes&Yes&Yes&No&Yes\\ 
\hline 
infinitive&subject&No&Yes&No&Yes&Yes\\ 
&complement&No&Yes&No&Yes&Yes\\ 
\hline 
subjunctive&subject&Yes&No&No&No&No\\ 
&complement&Yes&No&No&No&Yes\\ 
\hline 
gerundive\footnotemark\ &complement&No&No&No&No&Yes\\ 
\hline 
base & complement & No & No & No & No & Yes \\ 
\hline 
small clause & complement & No & No & No & No & Yes \\ 
\hline 
\end{tabular} 
\vspace{.2in} 
\begin{rawhtml} <dl> <dt>{Summary of Complementizer and Clause Combinations <p> </dl> \end{rawhtml}
\label{facts} 
\end{table} 
\footnotetext{Most gerundive phrases are treated as NP's.  In fact, all gerundive subjects are treated as NP's, and the only gerundive complements which receive a sentential parse are those for which there is no corresponding NP parse.  This was done to reduce duplication of parses. See Chapter~\ref{gerunds-chapter} for further discussion of gerunds.\label{gerund-footnote}} 
 
 
Another fact which must be accounted for in the analysis is that in infinitival 
clauses, the complementizer {\it for} must appear with an overt subject NP, 
whereas a complementizer-less infinitival clause never has an overt subject, as 
shown in \ref{ex:563}-\ref{ex:566}. (See section~\ref{for-complementizer} for more 
discussion of the case assignment issues relating to this construction.) 
 
\beginsentences
\sitem{To lose would be awful .}\label{ex:563} 
\sitem{For Penn to lose would be awful .}\label{ex:564} 
\sitem{$\ast$For to lose would be awful .}\label{ex:565} 
\sitem{$\ast$Penn to lose would be awful .}\label{ex:566} 
\endsentences

 
In addition, some verbs select {\bf $<$wh$>$=+} complements (either questions 
or clauses with {\it whether} or {\it if}) \cite{grimshaw90}: 
 
\beginsentences
\sitem{Jesse wondered who left .}\label{ex:567} 
\sitem{Jesse wondered if Barry left .}\label{ex:568} 
\sitem{Jesse wondered whether to leave .}\label{ex:569} 
\sitem{Jesse wondered whether Barry left .}\label{ex:570} 
\sitem{$\ast$Jesse thought who left .}\label{ex:571} 
\sitem{$\ast$Jesse thought if Barry left .}\label{ex:572} 
\sitem{$\ast$Jesse thought whether to leave .}\label{ex:573} 
\sitem{$\ast$Jesse thought whether Barry left .}\label{ex:574} 
\endsentences

 
\section{Features Required} 
\label{s-features} 
 
As we have seen above, clauses may be {\bf $<$wh$>$=+} or {\bf $<$wh$>$=--}, 
may have one of several complementizers or no complementizer, and can be of 
various clause types.  The XTAG analysis uses three features to capture these 
possibilities: {\bf $<$comp$>$} for the variation in complementizers, 
{\bf$<$wh$>$} for the question vs.  non-question alternation and {\bf $<$mode$>$}\footnote{{\bf $<$mode$>$} actually conflates several types of information, in particular verb form and mood.} for clause types.  In addition 
to these three features, the {\bf $<$assign-comp$>$} feature represents 
complementizer requirements of the embedded verb.  More detailed discussion of 
the {\bf $<$assign-comp$>$} feature appears below in the discussions of 
sentential subjects and of infinitives.  The four features and their possible 
values are shown in Table \ref{feat}. 
 
 
\begin{table}[th] 
\centering 
\begin{tabular}{|l|c|} \hline 
Feature&Values\\ 
\hline 
{\bf $<$comp$>$}&that, if, whether, for, rel, nil\\ 
\hline 
{\bf$<$mode$>$}&ind, inf, subjnt, ger, base, ppart, nom/prep\\ 
\hline 
{\bf$<$assign-comp$>$}&that, if, whether, for, rel, ind\underline{~}nil, inf\underline{~}nil\\ 
\hline 
{\bf$<$wh$>$}&+,--\\ 
\hline 
\end{tabular} 
\begin{rawhtml} <dl> <dt>{Summary of Relevant Features <p> </dl> \end{rawhtml}
\label{feat} 
\end{table} 
 
 
\section{Distribution of Complementizers} 
\label{comp-distr} 
 
Like other non-arguments, complementizers anchor an auxiliary tree (shown in 
Figure \ref{comp-tree}) and adjoin to elementary clausal trees.  The auxiliary 
tree for complementizers is the only alternative to having a complementizer 
position `built into' every sentential tree.  The latter choice would mean 
having an empty complementizer substitute into every matrix sentence and a 
complementizerless embedded sentence to fill the substitution node.  Our choice 
follows the XTAG principle that initial trees consist only of the arguments of 
the anchor\footnote{See section~\ref{compl-adj} for a discussion of the difference between complements and adjuncts in the XTAG grammar.} -- the S tree 
does not contain a slot for a complementizer, and the $\beta$COMP tree has only 
one argument, an S with particular features determined by the complementizer. 
Complementizers select the type of clause to which they adjoin through 
constraints on the {\bf $<$mode$>$} feature of the S foot node in the tree 
shown in Figure~\ref{comp-tree}.  These features also pass up to the root node, 
so that they are `visible' to the tree where the embedded sentence 
adjoins/substitutes. 
 
\begin{rawhtml} <p> \end{rawhtml}
\centering 
\hspace{0.0in} 
\htmladdimg{ps/sent-comps-subjs-files/betaCOMPs_that_.ps.gif} 
\begin{rawhtml} <dl> <dt>{Tree $\beta$COMPs, anchored by  that <p> </dl> \end{rawhtml}
\label{comp-tree} 
\begin{rawhtml} <p> \end{rawhtml}
 
The grammar handles the following complementizers: {\it that\/}, {\it whether\/}, {\it if\/}, {\it for\/}, and no complementizer, and the 
clause types: indicative, infinitival, gerundive, past participial, 
subjunctive and small clause ({\bf nom/prep}).  The {\bf $<$comp$>$} feature in a clausal tree reflects the value of the 
complementizer if one has adjoined to the clause. 
 
The {\bf $<$comp$>$} and {\bf $<$wh$>$} features receive their root 
node values from the particular complementizer which anchors the tree. 
The $\beta$COMPs tree adjoins to an S node with the feature {\bf $<$comp$>$=nil}; this feature indicates that the tree does not already 
{\bf have} a complementizer adjoined to it.\footnote{ Because root S's cannot have complementizers, the parser checks that the root S has {\bf $<$comp$>$=nil} at the end of the derivation, when the S is also checked for a tensed verb.} We ensure that there are no stacked complementizers by 
requiring the foot node of $\beta$COMPs to have {\bf $<$comp$>$=nil}. 
 
% as well 
%as using the {\bf $<$conj$>$=nil} feature to prevent complementizers from 
%adjoining above subordinating conjunctions. 
 
\section{Complementizer {\it for\/} and Case Assignment of the Subject} 
\label{for-complementizer} 
 
The {\bf $<$assign-comp$>$} feature is used to represent the 
requirements of particular types of clauses for particular 
complementizers.  So while the {\bf $<$comp$>$} feature represents 
constraints originating from the VP dominating the clause, the {\bf $<$assign-comp$>$} feature represents constraints originating from the 
highest VP in the clause. {\bf $<$assign-comp$>$} is used to control 
%appearance of subjects in infinitival clauses,  to 
%ensure the correct distribution of complementizers in sentential 
%subjects, and to block `that-trace' violations. 
the appearance of subjects in infinitival clauses (see discussion of 
ECM constructions in \ref{ecm-verbs}), to block bare indicative 
sentential subjects (bare infinitival subjects are allowed), and to 
block `that-trace' violations. 
 
Examples \ref{ex:576}, \ref{ex:577} and \ref{ex:578} show that an accusative 
case subject is obligatory in an infinitive clause if the 
complementizer {\it for\/} is present. The infinitive clause in 
\ref{ex:575} is analyzed in the English XTAG grammar as 
having a PRO subject.  
 
%The apparent subject of {\it to win\/} in 
%(\ref{ex:575}) is taken to be an object of the verb rather than the subject 
%of the infinitive clause. 
%\enumsentence{Mike wants her to pass the exam.} 
%Note: I (Seth) took out this sentence, since the tech report 
%claims it gets an object-control analysis, while it in fact gets an 
% ECM analysis.  It may be the case that it *should* get an ECM analysis, 
% but for now I took it out, because it doesn't seem to have anything to 
% do anyway with the point of this section. 
 
\beginsentences
\sitem{Christy wants to pass the exam .}\label{ex:575} 
\sitem{Mike wants for her to pass the exam .}\label{ex:576} 
\sitem{$\ast$Mike wants for she to pass the exam .}\label{ex:577} 
\sitem{$\ast$Christy wants for to pass the exam .}\label{ex:578} 
\endsentences

 
%The {\it for-to\/} construction is particularly illustrative of the 
%difficulties and benefits faced in using a lexicalized grammar.  
 
It is commonly accepted that {\it for\/} behaves as a case-assigning 
complementizer in this construction. It can assign accusative case to the 
embedded subject since the infinitival verb can not assign 
(nominative) case to this position.  
%However, in our featurized grammar, the 
%absence of a feature licenses anything, so we must have overt null 
%case assigned by infinitives to ensure the correct distribution of PRO 
%subjects. (See section~\ref{case-assignment} for more discussion of 
%case assignment.)  This null case assignment clashes with accusative 
%case assignment if we simply add {\it for\/} as a standard 
%complementizer, since NP's (including PRO) are drawn from the lexicon 
%already marked for case.  
In \ref{ex:577} there is a feature clash between the nominative case subject {\it she} and the accusative case assigning complementizer, thus accounting for its 
ungrammaticality. Similarly, the sentence in \ref{ex:578} is ruled out because PRO 
has a feature {\bf $<$case$>$=none} which is coindexed with with the {\bf $<$assign-case$>$} feature on S. This feature clashes with the {\bf $<$assign-case$>$=acc} feature in the {\it for} auxiliary tree. 
 
 
%Thus, we must use the {\bf %$<$assign-comp$>$} feature to pass information about the verb up to 
%the root of the embedded sentence.  To capture these facts, two 
%infinitive {\it to}'s are posited. One infinitive {\it to\/} has {\bf %$<$assign-case$>$=none} which forces a PRO subject, and {\bf %$<$assign-comp$>$=inf\_nil} which prevents {\it for\/} from 
%adjoining. The other infinitive {\it to\/} has no value at all for 
%{\bf $<$assign-case$>$} and has {\bf $<$assign-comp$>$=for/ecm}, so that 
%it can only occur either with the complementizer {\it for\/} or with 
%ECM constructions. In those 
%instances either {\it for} or the ECM verb 
%supplies the {\bf $<$assign-case$>$} value, assigning 
% accusative case to the overt subject. 
%% 
