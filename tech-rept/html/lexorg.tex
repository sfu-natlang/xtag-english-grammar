 
%\documentstyle[11pt]{article} 
%\documentstyle[11pt]{article} 
 
\chapter{Lexical Organization} 
\label{lexorg} 
 
 
%\input{psfig} 
 
 
%\begin{document} 
 
%\bibliographystyle{acl} 
 
 
\section{Introduction} 
 
% problem 
 
An important characteristic of an 
FB-LTAG is that it is lexicalized, i.e., each lexical item is anchored to a 
tree structure that encodes subcategorization information.  Trees with the same 
canonical subcategorizations are grouped into tree families.  The reuse of tree 
substructures, such as wh-movement, 
 in many different trees 
creates redundancy, which poses a problem for grammar development 
and maintenance \cite{vijay-schabes92}.  To consistently implement a change in 
some general aspect of the design of the grammar, all the relevant trees 
currently must be inspected and edited.  Vijay Shanker and Schabes suggested 
the use of hierarchical organization and of tree descriptions to specify 
substructures that would be present in several elementary trees of a grammar. 
Since then, in addition to ourselves, 
Becker, \cite{becker94}, Evans et al. \cite{Evans95}, and 
Candito\cite{Candito96} have 
developed systems for organizing trees of a TAG which could be used for 
developing and maintaining grammars. 
 
 Our system is based 
on the ideas expressed in Vijay-Shanker and Schabes, \cite{vijay-schabes92}, to 
use partial-tree descriptions in specifying a grammar by separately defining 
pieces of tree structures to encode independent syntactic principles. Various 
individual specifications are then combined to form the elementary trees of the 
grammar. The chapter begins with a description of 
our grammar development system, and its implementation. We will then 
show the main results of using this tool to generate 
the Penn English grammar as well as a Chinese 
TAG.  We describe the significant properties of both grammars, pointing out the 
major differences between them, and the methods by which our system is informed 
about these language-specific properties.  The chapter ends with the conclusion 
and future work. 
 
 
 
 
\section{System Overview} 
 
In our approach,  
three types of components -- subcategorization frames, blocks and 
lexical redistribution rules -- are used to describe lexical and 
syntactic information. 
%Lexical Redistribution Rule (LRR). 
Actual trees are generated automatically from these abstract descriptions, 
as shown in Figure \ref{system-overview}. 
In maintaining the grammar only the abstract 
descriptions need ever be manipulated; the tree descriptions and the 
actual trees which they subsume are computed deterministically from these 
high-level descriptions. 
 
 
\begin{rawhtml} <p> \end{rawhtml}
\centerline{\htmladdimg{ps/lexorg/overview.eps.gif}} 
\begin{rawhtml} <dl> <dt>{Lexical Organization: System Overview <p> </dl> \end{rawhtml}
\label{system-overview} 
\begin{rawhtml} <p> \end{rawhtml}
 
\subsection{Subcategorization frames} 
Subcategorization frames specify 
the category of the main anchor, 
the number of  arguments, 
each argument's category and position with respect to the anchor, 
and other information 
such as feature equations or node expansions. 
Each tree family has one canonical subcategorization frame. 
 
\subsection{Blocks} 
 
Blocks are used to represent the tree substructures that are reused 
in different trees, i.e. blocks 
subsume classes of trees. Each block includes 
a set of nodes, dominance relation, parent relation, 
precedence relation between nodes, 
and feature equations. This follows the definition of the tree 
descriptions specified in a logical language patterned after Rogers and 
Vijay-Shanker\cite{rogers-vijay94}. 
 
Blocks are divided into two types according to their functions: 
subcategorization  blocks 
and transformation blocks. 
The former describes structural configurations incorporating 
the various information in a subcategorization frame. 
For example, some of the subcategorization blocks 
used in the 
development of the English grammar are shown in Figure 
\ref{subcat-blocks}.\footnote{ In order to focus on the use of tree descriptions and to make the figures less cumbersome, we show only the structural aspects and do not show the feature value specification. The parent, (immediate dominance), relationship is illustrated by a plain line and the dominance relationship by a dotted line. The arc between nodes shows the precedence order of the nodes are unspecified. The nodes' categories are enclosed in parentheses.} 
 
When the subcategorization frame for a  verb is given 
 by the grammar 
developer, the system will automatically create a new block (of code) by 
essentially selecting the appropriate primitive subcategorization blocks 
corresponding to the argument information specified in that verb 
frame. 
 
The transformation blocks are used for various transformations such as 
wh-movement. These transformation blocks do not encode 
rules for modifying trees, but rather describe 
 the properties of a particular syntactic construction. 
 Figure \ref{trans-blocks} depicts 
our representation of phrasal extraction. 
This can be specialized to give the 
blocks for wh-movement, topicalization, relative clause formation, etc.  
For example, the 
wh-movement block is defined by further specifying that the ExtractionRoot is 
labeled S, the NewSite 
has a +wh feature, and so on. 
 
 
\begin{rawhtml} <p> \end{rawhtml}
\centerline{\htmladdimg{ps/lexorg/demosub.eps.gif}} 
\begin{rawhtml} <dl> <dt>{Some subcategorization blocks <p> </dl> \end{rawhtml}
\label{subcat-blocks} 
\begin{rawhtml} <p> \end{rawhtml}
 
\begin{rawhtml} <p> \end{rawhtml}
\centerline{\htmladdimg{ps/lexorg/extract.eps.gif}} 
\begin{rawhtml} <dl> <dt>{Transformation blocks for extraction <p> </dl> \end{rawhtml}
\label{trans-blocks} 
\begin{rawhtml} <p> \end{rawhtml}
 
 
\subsection{Lexical Redistribution Rules (LRRs)} 
The third type of machinery available for a grammar developer is the 
Lexical Redistribution Rule (LRR). An LRR is a pair 
($r_{l}$, $r_{r}$) of subcategorization frames, 
which produces a new frame when applied to a subcategorization frame s, 
by first {\it matching}\footnote{Matching occurs successfully when frame s is compatible with $r_{l}$ in the type of anchors, the number of arguments, their positions, categories and features. In other words, incompatible features etc. will block certain LRRs from being applied.} 
 the 
left frame $r_{l}$ of r to s, then 
combining information in $r_{r}$ and s. 
LRRs are introduced to incorporate 
the connection between subcategorization frames. For example, 
most transitive verbs have a frame for active(a subject and an object) 
and another frame for passive, where the object in the former frame becomes 
the subject in the latter. An LRR, denoted as  passive LRR, 
is built to produce the passive subcategorization frame from 
the active one. 
Similarly, applying dative-shift LRR to the frame with 
one NP subject and two NP objects will produce 
a frame with an NP subject and an PP object. 
 
 
Besides the distinct content, LRRs and blocks also differ in several 
aspects: 
\begin{itemize} 
  \item They have different functionalities: Blocks represent 
    the substructures that are reused in different trees. They 
    are used to reduce the redundancy among trees; LRRs are introduced 
    to incorporate the connections between the closely 
     related subcategorization 
    frames. 
  \item Blocks are strictly additive and can be added in any order. 
       LRRs, on the other hand, 
    produce different results depending on the order they are applied in, 
    and 
    are allowed to be non-additive, i.e., to remove information from 
    the subcategorization frame they are being applied to, as in 
    the procedure of passive from active. 
\end{itemize} 
    
 
 
\begin{rawhtml} <p> \end{rawhtml}
\centerline{\htmladdimg{ps/lexorg/newelem.eps.gif}} 
\begin{rawhtml} <dl> <dt>{Elementary trees generated from combining blocks <p> </dl> \end{rawhtml}
\label{elem} 
\begin{rawhtml} <p> \end{rawhtml}
 
 
 
\subsection{Tree generation} 
To generate elementary trees, 
 we begin with a canonical subcategorization frame. 
The system will first generate 
related subcategorization frames 
by applying LRRs, then select 
subcategorization blocks corresponding to the information in 
the subcategorization frames, next 
the combinations of these blocks 
are further combined with the blocks corresponding to 
 various transformations, finally,  a set of trees are generated 
from those combined blocks, and they are 
the tree family for this subcategorization frame. 
Figure \ref{elem} shows some 
of the trees produced in this way.  For instance, the last tree is obtained by 
incorporating information from the ditransitive verb subcategorization frame, 
applying the dative-shift and passive LRRs, and then combining them with the 
wh-non-subject extraction block. 
Besides, in our system the hierarchy for subcategorization frames is implicit 
as shown in Figure \ref{lattice}. 
 
\begin{rawhtml} <p> \end{rawhtml}
\centerline{\htmladdimg{ps/lexorg/frame.eps.gif}} 
\begin{rawhtml} <dl> <dt>{Partial inheritance lattice in English <p> </dl> \end{rawhtml}
\label{lattice} 
\begin{rawhtml} <p> \end{rawhtml}
 
 
 
 
\section{Implementation} 
 
The input of our system is the description of the language, which includes 
the subcategorization frame list, LRR list, subcategorization block 
list and transformation lists. The output  is a list of 
trees generated automatically by the system, as shown in Figure 
\ref{impl}. The tree generation module is written in Prolog, and 
the rest part is in C. We also have a graphic interface to input 
the language description. Figure \ref{interface-firstlevel} and 
\ref{interface-block} are two snapshots of the interface. 
 
\begin{rawhtml} <p> \end{rawhtml}
\centerline{\htmladdimg{ps/lexorg/impl.eps.gif}} 
\begin{rawhtml} <dl> <dt>{Implementation of the system <p> </dl> \end{rawhtml}
\label{impl} 
\begin{rawhtml} <p> \end{rawhtml}
 
 
\begin{rawhtml} <p> \end{rawhtml}
\centerline{\htmladdimg{ps/lexorg/java.interface.ps.gif}} 
\begin{rawhtml} <dl> <dt>{Interface for creating a grammar <p> </dl> \end{rawhtml}
\label{interface-firstlevel} 
\begin{rawhtml} <p> \end{rawhtml}
 
 
\begin{rawhtml} <p> \end{rawhtml}
\centerline{\htmladdimg{ps/lexorg/java.block.ps.gif}} 
\vspace{0.4in} 
\begin{rawhtml} <dl> <dt>{Part of the Interface for creating blocks <p> </dl> \end{rawhtml}
\label{interface-block} 
\begin{rawhtml} <p> \end{rawhtml}
 
 
\section{Generating grammars} 
 
We have used our tool to specify a grammar for English in order to produce 
the trees used in the current English XTAG grammar. 
We have also used our tool 
to generate a large grammar for Chinese. 
In designing these 
grammars, we have tried to specify the grammars to reflect the similarities and 
the differences between the languages. The major features of our specification 
of these two grammars\footnote{Both grammars are still under development, so the contents of these two tables might change a lot in the future according to the analyses we choose for certain phenomenon. For example, the majority of work on Chinese grammar treat ba-construction as some kind of object-fronting where the character {\it ba} is either an object marker or a preposition. According to this analysis, an LRR rule for ba-construction is used in our grammar to generate the preverbal-object frame from the postverbal frame. However, there has been some argument for treating {\it ba} as a verb. If we later choose that analysis, the main verbs in the patterns ``NP0 VP'' and ``NP0 ba NP1 VP'' will be different, therefore no LRR will be needed for it. As a result, the numbers of LRRs, subcat frames and tree generated will change accordingly.} 
 are summarized in Table \ref{table} and \ref{table2}. 
 
\begin{table}[ht] 
\centering 
\begin{tabular}{|l|l|l|}     \hline 
         & English & Chinese \\ \hline 
examples     & passive       & bei-construction  \\ 
of LRRs         & dative-shift  & object fronting  \\ 
         & ergative  & ba-construction   \\ \hline 
examples  & wh-question  & topicalization  \\ 
of transformation         & relativization & relativization \\ 
blocks         & declarative & argument-drop \\ \hline 
\#  LRRs &   6     & 12  \\ \hline 
\#  subcat blocks & 34 & 24 \\ \hline 
\#  trans  blocks & 8  & 15 \\ \hline 
\#  subcat frames  & 43 & 23 \\ \hline 
\#  trees generated   & 638 & 280 \\ \hline 
\end{tabular}  \\ 
\begin{rawhtml} <dl> <dt>{Major features of English and Chinese grammars <p> </dl> \end{rawhtml}
\label{table} 
\end{table} 
 
\begin{table}[ht] 
\centering 
\begin{tabular}{|l|l|l|l|}     \hline 
         & both grammars & English & Chinese \\ \hline 
         & causative & long passive &  VO-inversion \\ 
LRRs     & short passive & ergative  & ba-const \\ 
         &               & dative-shift  &       \\  \hline 
         & topicalization  &  &   \\ 
trans blocks    & relativization & gerund & argument-drop \\ 
        & declarative &  &  \\ \hline 
        & NP/S subject &  &  zero-subject \\ 
 subcat blocks & S/NP/PP object & PL object & preverbal object \\ 
               & V predicate &  prep predicate & \\ \hline 
\end{tabular}  \\ 
\begin{rawhtml} <dl> <dt>{Comparison of the two grammars <p> </dl> \end{rawhtml}
\label{table2} 
\end{table} 
 
 
 
By focusing on the specification of individual grammatical information, we 
have been able to generate nearly all of the trees from the tree families 
used in the current English grammar developed at Penn\footnote{We have not yet attempted to extend our coverage to include punctuation, it-clefts, and a few idiosyncratic analyses.}. 
 Our approach, has also exposed certain gaps in the Penn grammar. 
%in \ref{envgram97}.  
We are encouraged with the utility of our tool and the ease with which this 
large-scale grammar was developed. 
 
 
We are currently working on expanding the contents of 
subcategorization frame to include 
trees for other categories of words. For example, 
a frame which has no specifier and one NP complement 
and whose predicate is a preposition will correspond to 
PP $\rightarrow$ P NP tree. We'll also introduce a modifier field 
and semantic features, so that the head features will propagate 
from modifiee to modified node, while non-head features from 
the predicate as the head of the modifier will be passed to 
the modified node. 
 
% insert figures here 
 
 
 
 
 
 
\section{Summary} 
 
We have described a tool for grammar development in which tree descriptions are 
used to provide an abstract specification of the linguistic phenomena relevant 
to a particular language.  In grammar development and maintenance, only the 
abstract specifications need to be edited, and any changes or corrections will 
automatically be proliferated throughout the grammar.  In addition to 
lightening the more tedious aspects of grammar maintenance, this approach also 
allows a unique perspective on the general characteristics of a language. 
Defining hierarchical blocks for the grammar both necessitates and facilitates 
an examination of the linguistic assumptions that have been made with regard to 
feature specification and tree-family definition. This can be very useful for 
gaining an overview of the theory that is being implemented and exposing gaps 
that remain unmotivated and need to be investigated.  The type of gaps that can 
be exposed could include a missing subcategorization frame that might arise 
from the automatic combination of blocks and which would correspond to an 
entire tree family, a missing tree which would represent a particular type of 
transformation for a subcategorization frame, or inconsistent feature 
equations.  By focusing on syntactic properties at a higher level, our 
approach allows new opportunities for the investigation of how languages relate 
to themselves and to each other. 
 
 
 
%\bibliography{/mnt/linc/home/fxia/paper/mpaper/ACL98/ACL98} 
 
%\end{document} 
