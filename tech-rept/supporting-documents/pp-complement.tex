\chaptert{\Large PP Complement Verbs}
\label{pp-complement-chapter}

Just as some verbs require noun phrase complements, others require
preposition phrase complements.  For an intransitive verb like {\it
sleep}, a PP which follows it is an adjunct, as can be seen in
(\ex{1}) and (\ex{2}), but for a verb such as {\it venture}, the PP is
required, as the ungrammaticality of (\ex{4}) attests.

\enumsentence{The spelunker slept in the cave .}
\enumsentence{The spelunker slept .}
\enumsentence{The spelunker ventured into the cave .}
\enumsentence{*The spelunker ventured .}

\noindent
We will call verbs such as {\it venture} transitive PP
complement verbs.  There are also ditransitive PP complement
verbs, such as {\it put}.

\enumsentence{Bill put the cigar on the table .}
\enumsentence{*Bill put the cigar .}


Because the PP is an argument of the verb, PP complement verbs must
provide a substitution site for the PP. Two structures are available
to do this: the PP substitution node structure (shown in
Figure~\ref{two-pp-comp-analyses}a) and the expanded PP structure
(shown in Figure~\ref{two-pp-comp-analyses}b).

\begin{figure}[htbp]
\centering
\begin{tabular}{ccc}
{\psfig{figure=ps/pp-complement-files/PP-subst.ps,height=0.3in}}  &
\hspace{0.6in}
{\psfig{figure=ps/pp-complement-files/PP-expanded.ps,height=1.0in}} \\
(a) PP substitition& \qquad(b) Expanded PP \\
\end{tabular}\\
\caption {Two analyses of the PP complement}
\label {two-pp-comp-analyses}
\end{figure}

We have chosen the expanded PP structre for two reasons.  One is that
it is possible to extract the NP object of the P, as shown in
(\ex{1}).  Expanding the PP allows the NP node to be available to
metarules which create the trees for the extraction.

\enumsentence{What did Bill put the cigar on ?}

\noindent
Another reason is semantic in nature.  The elementary tree is the
domain in which a verb's inherent semantics is represented, and we
have decided at present to regard the NP object of the P as an
argument of the main predicate, rather than the whole PP itself.  In
order to have access to that NP for the semantic representation, it
must be explicitly represented in the elementary tree.

However, exhaustive prepositions such as {\it there}, {\it somewhere},
and {\it where} do present a problem for expanding the PP in this way.
It does not give us a way to substitute a whole PP in at once, which
is what these elements require.  Since our decision is to expand the
PP, we currently cannot parse sentences such as (\ex{1}).

\enumsentence{Bill put the cigar there .}

There is another class of PP complement verbs for which the question
of expansion does not arise.  These are the verbs which require a
particular preposition to head their PP complement, examples of which
are given in the following:

\enumsentence{Calvin depends on Hobbes .}
\enumsentence{*Calvin depends with Hobbes .}
\enumsentence{Calvin thought of the transmogrifier .}
\enumsentence{*Calvin thought on the transmogrifier .\footnote{This
is fine as an adjunct, but not as a complement.}}
\enumsentence{The expedition leader geared the hikers for the trip .}
\enumsentence{*The expedition leader geared the hikers into the trip .}
\enumsentence{The speaker subjected the audience to his dull wit .}

\noindent
For such verbs, the preposition is intrinsic to the construction.
Note that sentence (\ex{-4}) is ambiguous between two interpretations,
one in which the PP is an adjunct and the other in which it is a
complement.  The adjunct interpretation of this sentence is something
like ``Calvin was thinking and the topic of this thoughts was the
transmogrifier'' while the complement interpretation is ``Calvin
created the idea for the transmogrifier''.  The preposition correlated
with the verb thus conveys a noncompositional meaning, which motivates
us to analyze these constructions as being multiply-anchored by a verb
and a corresponding preposition.  Having the P as one of the anchors
necessitates the expansion of the PP complement for these
constructions.

We have thus determined one important question of the basic structure
of PP complement verbs.  Another important aspect of these verbs is
that certain adjuncts may come between the verb and the PP complement,
unlike the case of NP complements.  Note that this is different from
NP complements, which cannot be separated from the verb by modifiers
unless they are heavy.

\enumsentence{*Bill borrowed quickly the car .}
\enumsentence{*Bill borrowed for many days the car .}
\enumsentence{Bill borrowed for many days the car which his father had
bought from his uncle two weeks ago .}
\enumsentence{*John gave Mary often the book .}

PP complements, on the other hand, allow several types of adjuncts in
this position.  For example, adverbs are able to intercede the verb
and the preposition:

\enumsentence{John depends heavily on his staff .}
\enumsentence{John thinks often of new ideas .}
\enumsentence{John bases his opinions extensively on public attitudes .}

\noindent Certain prepositional phrases can too:

\enumsentence{John depends in many ways on his staff .}
\enumsentence{John thought for many hours of new ideas .}
\enumsentence{John based his opinions for more than ten years on public attitudes .}

\noindent However, the type of PP is quite restricted.  The locative
PP's in the following examples cannot intervene.

\enumsentence{*John depends from Honolulu on his staff .}
\enumsentence{*John thought in the kitchen of new ideas .}
\enumsentence{*John based his opinions in Honolulu on public attitudes .}

\noindent 
Nominal adjuncts are possible:

\enumsentence{John depends quite a bit on his staff .}
\enumsentence{John thought all the time of new ideas .}
\enumsentence{John bases his opinions a good deal on public attitudes .}

\noindent
Adjunct clauses are bad:

\enumsentence{*John depends because he needs help on his wonderful and caring staff .}
\enumsentence{*John thought because he was so creative of new ideas that
provided the basis of many present technologies .}
\enumsentence{*John bases his opinions because he cannot think for himself on
the public attitudes that prevail today .}

There are several ways in which we could handle these adjuncts.  One
of these is to allow certain things to adjoin to V.  However, this is
unappealing because it would cause right-edge ambiguities (V vs. VP
adjunction), and it simply fails because it cannot handle examples
like (\ex{-12}), (\ex{-9}), (\ex{-6}), and (\ex{-3}) since there is
an NP intervening between the verb and the preposition in these cases.

Another option is to make the claim that the adjuncts get into the
intervening position because of heavy-shift.  Then, the structures of
the trees for prepositional complements should not provide a site for
adjunction between the verb and the preposition.  However, there is no
sense of heaviness in the examples (\ex{-14}-\ex{-3}), and we would
expect (\ex{-2}),(\ex{-1}), and (\ex{0}) to be fine since the PP's are
quite heavy in those cases.

The analysis we provide is structural, and is given in
Figure~\ref{pp-comp-trees} for the multi-anchor PP complement verbs.
The analysis suggests short verb movement, and is reminiscent of
Larsonian shells.  However, the most important aspect of these trees
for the purposes of the English XTAG grammar is that it provides a
site for left adjunction to VP that intervenes between the verb and
the preposition.  This immediately rules out the adjunct clauses as
desired, since such clauses only right adjoin to VP in the grammar.
Most prepositional phrases are not good in the intervening position
either, and this works nicely with the fact that the grammar does not
have left VP adjoining prepositional phrases.  For the restricted
class of prepositions which are permitted in the intervening position,
we currently do not parse them, but entries with the left adjoining
possibility can be added at some later date if we wish to be able to
do so.  Adverbs, on the other hand, already left adjoin to VP's in the
present grammar, so the structure accomodates that possibility without
need of further modifications to the grammar.

The comparatives are also an excellent indication that providing a VP
for left adjunction is the right idea.  Comparativized adverbs such as
{\it more heavily} may either left or right adjoin, so that portion of
a comparative can adjoin to the intervening position.  However, the
{\it than}-clause portion of a comparative can only right adjoin, and
together, these two adjoining behaviors work with our PP complement
analysis to correctly capture that distinctions in the following
examples:

\enumsentence{John depends more heavily on his staff than Mary .}
\enumsentence{John depends on his staff more heavily than Mary .}
\enumsentence{*John depends more heavily than Mary on his staff .}

\noindent
In (\ex{-2}), the comparativized adverb has left adjoined to the lower
VP and the {\it than}-clause has right adjoined.  Both right adjoin in
(\ex{-1}).  However, in (\ex{0}) the {\it than}-clause cannot adjoin
since there is no right-adjoinable VP available between the verb and
the preposition.

\begin{figure}[htbp]
\centering
\begin{tabular}{ccc}
{\psfig{figure=ps/pp-complement-files/alphanx0VPnx1.ps,height=3.0in}}  &
\hspace{0.6in}
{\psfig{figure=ps/pp-complement-files/alphanx0Vnx1Pnx2.ps,height=3.0in}} \\
(a) $\alpha$nx0VPnx1 & \qquad(b) $\alpha$nx0Vnx1Pnx2\\
\end{tabular}\\
\caption {Declarative trees for multi-anchor PP complement families}
\label {pp-comp-trees}
\end{figure}

This basic schema of short verb movement for PP complement families
captures the adjunction possibilities directly and appears to partion
the different types of adjoining elements in the appropriate way, with
very little modification to the grammar.  This analysis is operative
in the families Tnx0Vpnx1, Tnx0Vnx1pnx2, Tnx0VPnx1, and Tnx0Vnx1Pnx2.

Notice that having two VP nodes along the right spine of the trees
will engender right edge ambiguities.  This arises in other
structures, and right adjoiners to VP have the restriction {\bf mainv
= +} on their footnodes to avoid right edge ambiguities in sentences
which have auxiliaries.  This extends nicely to the present case.  We
restrict right VP adjunction to occur at the higher VP by letting {\bf
mainv} be negative in the lower VP, and thus we do not get right edge
ambiguities.
