%%%%%%%%%%%%%%%%%%
There are two features responsible for case-assignment:\\
{\bf $\langle$case$\rangle$}, possible values: {\bf nom, acc, gen, none}\\
{\bf $\langle$assign-case$\rangle$}, possible values: {\bf nom, acc, none}

Case assigners (prepositions and verbs) as well as the VP, S and PP
nodes that dominate them have an {\bf $\langle$assign-case$\rangle$}
case feature. Phrases and lexical items that have case i.e. Ns and NPs
have a {\bf $\langle$case$\rangle$} feature.

Case assignment by prepositions involves the following equations:

\enumsentence{ {\bf PP.b:$\langle$assign-case$\rangle =$ P.t:$\langle$case$\rangle$}}


\enumsentence{ {\bf NP.t:$\langle$case$\rangle =$ P.t:$\langle$case$\rangle$}}

Prepositions come specified from the lexicon with their {\bf $\langle$assign-case$\rangle$}
feature.

\enumsentence{ {\bf P.b:$\langle$assign-case$\rangle =$ acc}}


Case assignment by verbs has two parts: assignment of case to the
object(s) and assignment of case to the subject. Assignment of case to
the object is simpler.  English verbs always assign accusative case to
their NP objects (direct or indirect).  Hence this is built into the
tree and not put into the lexical entry of each individual verb.

\enumsentence{ {\bf NP$_{object}$.t:$\langle$case$\rangle =$ acc}}

Assignment of case to the subject involves the following two equations.

\enumsentence{ {\bf NP$_{subj}$:$\langle$case$\rangle =$ VP.t:$\langle$assign-case$\rangle$}}


\enumsentence{ {\bf VP.b:$\langle$assign-case$\rangle =$ V.t:$\langle$assign-case$\rangle$}}

This is a two step process -- the final case assigned to the subject
depends upon the {\bf $\langle$assign-case$\rangle$} feature of the
verb as well as whether an auxiliary verb adjoins in.

Finite verbs like {\em sings} have {\bf nom} as the value of their
{\bf $\langle$assign-case$\rangle$} feature. Non-finite verbs have
no value for the {\bf $\langle$assign-case$\rangle$}
feature. In all grammatical examples with lexical NP subjects, a tensed
auxiliary adjoins in which assigns {\bf nom} to the subject, or accusative case
is assigned from above (from the complementizer {\it for} or from an ECM
verb). PRO subjects have {\bf $\langle$case$\rangle =$ none} which would clash
with the {\bf $\langle$assign-case$\rangle$} of a tensed auxiliary or anything
else that would try to assign case to the subject.\footnote{Note that tensed
auxiliaries are independently ruled out from adjoining into PRO trees by the
{\bf $\langle$mode$\rangle$} feature.}

%So if no auxiliary adjoins in, the only subject they can have
%is {\bf PRO} which is the only NP with {\bf none} as the value its
%{\bf $\langle$case$\rangle$} feature.

\subsection{ECM}
Certain verbs e.g. {\em want, believe, consider} etc. and one complementizer
{\em for} are able to assign case to the subject of their complement clause. 

The complementizer {\em for}, like the preposition {\em for}, has the
{\bf $\langle$assign-case$\rangle$} feature of its complement set to
{\bf acc}. Since the {\bf $\langle$assign-case$\rangle$} feature of
the root S$_{r}$ of the complement tree and the {\bf
$\langle$case$\rangle$} feature of its NP subject are co-indexed, this
leads to the subject being assigned accusative case.

ECM verbs have the {\bf $\langle$assign-case$\rangle$}  feature of their
foot S node set to {\bf acc}. The co-indexation between the 
{\bf $\langle$assign-case$\rangle$} feature of
the root S$_{r}$ and the {\bf $\langle$case$\rangle$} feature of the NP subject
leads to the subject being assigned accusative case.

\subsection{Agreement and Case}
The {\bf $\langle$case$\rangle$} features of a moved NP and its trace 
are co-indexed. This captures the fact that movement does not disrupt 
a pre-existing relationship of case-assignment between a verb and an NP.

\enumsentence{ Her$_{i}$/*She$_{i}$, I think that Odo like t$_{i}$.}


%%%%%%%%%%%%%%%%%%%%%%%%%%%%%%%%%%%%%%%%%%%%%%
