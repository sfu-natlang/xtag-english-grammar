\chapter{Tree Families in XTAG}
\label{families}

In the XTAG English grammar, tree families provide us with a compact
means to specify the set of trees which a particular lexical anchor
takes. This set represents all syntactic environments in which this
anchor appears. One way to represent tree families is to have a
universal set of trees, with each individual anchor taking some subset
of that larger set.  However, since syntactic transformations are not
usually sensitive to the anchor of the tree family, in practice we
assign a tree family to entire classes of words. 

Chapter~\ref{underview} defines the notion of tree family used in the
XTAG grammar. In this chapter we discuss in more detail what it means
for a particular anchor to select several tree families keeping in
mind that entire classes of words which share the same
subcategorization are assigned tree families in the syntactic
database. Also, there are some syntactic transformations that are
sensitive to the properties of a particular word within the same
subcategorization frame.

In general, the trees within a particular tree family can be thought
of as being those which are syntactically or transformationally
related. The tree in Figure~\ref{trans-base-tree} allows us to parse
sentences such as (\ex{1}). This tree represents the subcategorization
of the verb and forms the base for the generation of an entire tree
family. Each family has one such {\em base tree}. One method for
generating all the trees in the family that are syntactically related to
the base tree is to use metarules (see Appendix~\ref{metarules}). One
such member of the transitive tree family is the tree in
Figure~\ref{trans-extracted-tree} which is the tree where the object
of the $\alpha$nx0Vnx1 has been extracted. This new tree
$\alpha$W1nx0Vnx1 now allows us to parse sentences such as (\ex{2}).

\enumsentence{Susie likes Hobbes.}
\enumsentence{Who does Susie like?}


\begin{figure}[htb]
\centering
\begin{tabular}{c}
\psfig{figure=ps/verb-class-files/alphanx0Vnx1.ps,height=3.4cm}
\end{tabular}
\caption{Declarative Transitive Tree:  $\alpha$nx0Vnx1}
\label{trans-base-tree}
\end{figure}

\begin{figure}[htb]
\centering
\mbox{}
\psfig{figure=ps/extraction-files/alphaW1nx0Vnx1.ps,height=10.0cm}
\caption{Transitive tree with object extraction: $\alpha$W1nx0Vnx1}
\label{trans-extracted-tree}
\end{figure} 

The semantic interpretation of arguments across different trees in a
family remains constant.  That is, if $\mbox{\em NP}_1$ is the
recipient argument of the predicate in the base indicative tree, then
it is the recipient when it is fronted as well.

There are some syntactic transformations, however, that are sensitive to the
properties of a particular word within the same subcategorization frame. The
ergative (or transitive-inchoative alternation) for transitive verbs is one
such transformation. Only a subset of the transitive verbs can undergo this
transformation. Let us call such transformations {\em lexical rules} to
distinguish them from more general {\em syntactic transformations} (like
wh-extraction) which are not lexically idiosyncratic.

To explain how these two different kinds of transformations interact
let us take the example of the ergative subset of the transitive
verbs:

Ergative verbs such as {\it melt} undergo the ergative transformation (see
Chapter~\ref{ergatives}). Such verbs are a subset of the set of transitive
verbs, which do not all take this transformation; for example verbs like {\it
borrow} cannot take this transformation. To handle the transitive/ergative
distinction, we could create a tree family which exclusively handles the
regular transitives and another one which handles the ergatives.  In doing so,
however, we would be duplicating structure since the ergative family would
contain all of the trees found in the transitive family in addition to the
purely ergative trees.  An alternative is to consider the ergative distinction
as arising from a {\em lexical rule} which takes the base tree of the
transitive family ($\alpha$nx0Vnx1) and produces the base tree for a strictly
ergative family ($\alpha$Enx1V).  Then all of the syntactic transformations
which are available in the grammar can be applied to that new base tree to form
the entire ergative tree family.  In this particular instance, the indexation
on the syntactic object NP of $\alpha$nx0Vnx1 ($\mbox{\em NP}_1$) is retained
on the syntactic subject of the ergative tree, which has the same semantic role
in both constructions.

The same problem crops up in the ditranstive tree family with the subset of
anchors also undergo the dative-shift transformation. We give an identical
solution to this case as we do for the ergatives.

Hence, tree families can be related to each other by lexical rules.
Construing tree families in this way allows us to take advantage of
shared structure when creating a grammar. Introducing this shallow
hierarchy in the definition of family allows us to have a more compact
organization of the grammar. This departs from earlier conceptions of
tree families which maintained that argument positions in different
tree families were in no way related; thus, before, if a lexical item
anchored two different tree families, each anchoring instance was
considered a different predicate.  Under both the previous and the
present conceptions, tree families are the exhausitive domain of a
base tree and the trees generated by syntactic transformations on that
base.  However, the present conception of tree families provides for a
new level of distinction -- that of families related by a lexical
rule. These related families are the exhaustive domain of a particular
predicate that is affected by a lexical rule.

The XTAG grammar encodes the notion of lexical rules implicitly rather
than explicitly. It is implicit in the anchorings of certain lexical
items.  For example, ergative verbs such as {\it melt} anchor both
Tnx0Vnx1 and TEnx1V, and their lexical family is the set of trees in
the union of these two tree families.  See the chapters on ergatives
(see Chapter~\ref{ergatives}) and ditransitives (see
Chapter~\ref{double-objs}) for more in-depth discussions of particular
cases of lexical families.

To briefly summarize, we can now give the following definitions:

\begin{description}
\item[tree family:] the set of trees generated from a base tree by the
syntactic transformations which are available in the grammar.

\item[lexical family:] the set of tree families generated for a given
predicate from an initial base tree and the base trees which are
generated by the lexical rules which are specified for that predicate.

\end{description}

It should be noted, however, that we will continue to refer to tree families
and lexical families both by the generic name {\it tree family}, unless the
distinction is crucial to the point we are trying to make.
