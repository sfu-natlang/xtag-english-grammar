\section{Modifiers}

This section covers a various types of modifiers:  adverbs, prepositions,
adjectives, and noun modifiers in noun-noun compounds.  These categories
 optionally modify other lexical items and phrases by adjoining
onto them.  In their modifier function these items are adjuncts; they
are not part of the subcategorization frame of the items they modify.
Examples of some of these modifiers are shown in (\ex{1})-(\ex{3}).

\enumsentence{[$_{ADV}$ Certainly $_{ADV}$], the Oct. 13 sell-off
didn't settle any stomachs.(WSJ)}
\enumsentence{Mr. Bakes [$_{ADV}$ previously $_{ADV}$] had a turn at running Continental.(WSJ)}
\enumsentence{Most [$_{ADJ}$ foreign $_{ADJ}$] [$_{N}$ government
$_{N}$] [$_{N}$ bond $_{N}$] [$_{N}$ prices $_{N}$] rose [$_{PP}$ in light trading $_{PP}$.(WSJ)}

The trees used for the various modifiers are quite similar in form.
The modifier anchors the tree and the root and foot nodes of the tree are
of the category that the particular anchor modifies. Some modifiers, for
example prepositions, have arguments that are also included in the
tree.  The footnode may be to the right or the left of the anchoring
modifier (and its arguments) depending on whether that modifier occurs
before or after the category it modifies. For example, almost all
adjectives appear to the left of the nouns they modify, while
prepositions appear to the right when modifying nouns. 


\subsection{Adjectives}
\label{adj-modifier}

In addition to being modifiers, adjectives in the FBLTAG English
grammar can be predicative and anchor clauses (see
Section~\ref{small-clauses}). \label{2;2,14} There is also one tree family, Tnx0Va,
that has an adjective as an argument and is used for sentences such as 
{\it Seth felt happy}. In this tree family the adjective substitutes
into the tree rather than adjoining as is the case for modifiers.

As modifiers, adjectives anchor the tree $\beta$An shown in Figure \ref{An-tree}.

\begin{figure}[ht]
\centering
\begin{tabular}{cc}
{\psfig{figure=/mnt/linc/extra/xtag/work/doc/tech-rept/ps/modifiers-files/betaAn.ps,height=1.5in}}
\end{tabular}\\
\caption {Standard Tree for Adjective modifying a Noun}
\label {An-tree}
\end{figure}

As can be seen in Figure \ref{An-tree}, the features of the N onto
which $\beta$An tree adjoins are passed to the top node of the
resulting N.  The null adjunction marker (NA) on the N footnode
imposes right binary branching such that each subsequent adjective
must adjoin on top of the leftmost adjective that has already
adjoined.  Due to the NA constraint, a sequence of adjectives will
have only one derivation in our grammar. The adjective's morphological
features such as superlative or comparitive are not currently used in the tree.
At this point, the treatment of adjectives in the FBLATG English grammar does
not include selectional or ordering restrictions. Consequently, any
adjective can adjoin onto any noun and on top of any other adjective
already modifying a noun. All of the modified noun phrases shown in
(\ex{1})-(\ex{4}) currently parse with the same structure shown
for {\it colorless green ideas\/} in Figure
\ref{colorless-green-adj}.

\enumsentence{big green bugs}
\enumsentence{big green ideas}
\enumsentence{colorless green ideas}
\enumsentence{$\ast$green big ideas}


\begin{figure}[ht]
\centering
\begin{tabular}{cc}
{\psfig{figure=/mnt/linc/extra/xtag/work/doc/tech-rept/ps/cant-get-to-dir-stuff/colorless-green-ideas.ps,height=4in}}
\end{tabular}\\
\caption {multiple adjectives modifying an N}
\label {colorless-green-adj}
\end{figure}


While (\ex{-2})-(\ex{0}) are all semantically anomalous, (\ex{0}) also
suffers from an ordering problem that makes it seem ungrammatical as
well. We would argue that the grammar should accept (\ex{-2}) and
(\ex{-1})\footnote{This is, in fact, the point of the famous
linguistic example in (\ex{-1})} but not (\ex{0}).  One of the future
goals for the grammar is to develop a treatment of adjective ordering
similar to that developed by Hockey and Egedi \shortcite{HockeyEgedi94} for determiners\footnote{See
\ref{det} or
\cite{HockeyEgedi94} for details of the determiner analysis}. An adequate implementation of
ordering restrictions for adjectives would rule out (\ex{0}).

Another area in which we plan to have future grammar development is
comparatives.  Comparatives that involve ellipsis will require a
general solution of the problem of representing ellipsis.  Simpler
comparatives without ellipsis, such as {\it fewer than nine\/} in
(\ex{1}), should be amenable to analysis as complex determiners (see
Section \ref{det-comparitives} for brief discussion of how this might
be done).

\enumsentence{Cats actually have fewer than nine lives.}

%\begin{figure}[ht]
%\centering
%\rule[.1in]{3.5in}{0.01in} \\
%\begin{tabular}{c}
%{\psfig{figure=/mnt/linc/extra/xtag/work/doc/tech-rept/ps/modifiers-files/beta%An.ps,height=1.5in}}
%\end{tabular}\\
%\caption {Beginning attempt at simple comparitives}
%\rule[.1in]{3.5in}{0.01in}
%\label {Apnx-tree}
%\end{figure}

\subsection{Noun-Noun Modifiers}
\label{noun-modifer}

Noun-noun compounding in the English FBLTAG grammar is very similar to
adjective-noun modification.  The noun modifier tree, $\beta$Nn which
is shown in Figure \ref{noun-compound-tree}, has essentially
the same structure as the adjective modifier tree, $beta$An, in Figure
\ref{An-tree}. There are two differences between the trees, the syntactic
category label of the anchor and the NA constraint on the footnode,
which appears only in the adjective modifier tree.  Noun compounds
have a variety of scope possiblities not availible to adjectives,
consequently the NA constraint on the footnode would incorrectly
restrict noun-noun compounding. The difference in scope possiblities
for noun modifiers and adjective modifiers is illustrated by the
single bracking possibility in (\ex{1}) and the two possibilies for
(\ex{2}).

\enumsentence{ [$_{N}$ big [$_{N}$ blue design $_{N}$]$_{N}$]}
\eenumsentence{\item[a.] [$_{N}$computer [$_{N}$furniture
design$_{N}$]$_{N}$]
\item[b.]  [$_{N}$ [$_{N}$ computer furniture $_{N}$] design $_{N}$]}

This lack of restriction results in noun-noun compounds regularly
having mulitple derivations. However, the multiple derivations are not
a defect in the grammar, because they are neccesary to correctly
represent the genuine ambiguity of these phrases.


\begin{figure}[ht]
\centering
\begin{tabular}{cc}
{\psfig{figure=/mnt/linc/extra/xtag/work/doc/tech-rept/ps/cant-get-to-dir-stuff/betaNn.ps,height=4in}}
\end{tabular}\\
\caption {multiple adjectives modifying a N}
\label {noun-compound-tree}
\end{figure}




\subsection{Prepositions}
\label{prep-modifier}

There are three basic types of prepositional phrases, and three places at
which they can adjoin.  The three types of prepositional phrases are:
Preposition with NP complement, Preposition with Sentential Complement, and
Exhaustive Preposition.  The three places are to the right of an NP, to the
right of a VP, and to the left of an S.  Each of the three types of PP can
adjoin at each of these three places, for a total of nine PP modifier
trees. Table \ref{prep-summary} gives the tree family names for the
various combinations of type and location.

\begin{table}
\centering
\begin{tabular}{|l||c|c|c|}
\hline
\multicolumn{1}{|c||}{}&\multicolumn{3}{c|}{position and category modified}\\
\cline{2-4}
\multicolumn{1}{|c||}{}&pre-sentential&post-NP&post-VP\\
\multicolumn{1}{|c||}{Complement type}&S modifier&NP modifier&VP modifier\\
\hline
\hline
S-complement&$\beta$Pss&$\beta$nxPs&$\beta$vxPs\\
\hline
NP-complement&$\beta$Pnxs&$\beta$nxPnx&$\beta$vxPnx\\
\hline
no complement&$\beta$Ps&$\beta$nxP&$\beta$vxP\\
(exhaustive)&&&\\
\hline
\end{tabular}
\caption{Preposition Anchored Modifiers}
\label{prep-summary}
\end{table}

The subset of preposition anchored modifier trees in Figure
\ref{prep-trees} illustrates the three locations and the three PP
types.

\begin{figure}[ht]
\centering
\begin{tabular}{ccc}
{\psfig{figure=/mnt/linc/extra/xtag/work/doc/tech-rept/ps/modifiers-files/betaPss.ps,height=1.5in}}  &
{\psfig{figure=/mnt/linc/extra/xtag/work/doc/tech-rept/ps/modifiers-files/betanxPnx.ps,height=1.5in}}  &
{\psfig{figure=/mnt/linc/extra/xtag/work/doc/tech-rept/ps/modifiers-files/betavxP.ps,height=1.5in}}
\\
$\beta$Pss&$\beta$nxPnx&$\beta$vxP\\
\end{tabular}\\
\caption {Selected Prepositional Phrase Modifier trees}
\label {prep-trees}
\end{figure}


Example sentences using the trees in Figure \ref{prep-trees} are shown
in (\ex{1})-(\ex{2}).\\


\enumsentence{[$_{PP}$ With Clove healthy $_{PP}$], the veteranarian's bill will be more affordable. ($\beta$Pss)}
\enumsentence{The frisbee [$_{PP}$ in the brambles $_{PP}$] was hidden.
($\beta$nxPnx)}
\enumsentence{Clove played frisbee [$_{PP}$ in the morning $_{PP}$]. ($\beta$vx)}


Prepositions that take NP complements assign case to those complements (see
Section \ref{prep-case} for details).  Most prepositions take NP complements.
There are just a few prepositions that take sentential complements. (see Section \ref{NPA}).



\subsection{Adverbs}
\label{adv-modifier}

In the English FBLTAG grammar VP and S-modifying adverbs anchor the
auxiliary trees $\beta$ARBs, $\beta$sARB, $\beta$vxARB and $\beta$ARBvx.
Besides the VP and S-modifying adverbs, the English FBLTAG grammar includes
adverbs that modify other categories. Examples of adverbs modifying
an adjective, an adverb, a PP and a determiner are shown in (\ex{1})-(\ex{4}). 

\enumsentence{modifying an adjective:\\ {\em extremely\/} good\\
{\em rather\/} tall\\rich {\em enough\/} }
\enumsentence{modifying an adverb:\\ oddly {\em enough\/}\\ {\em very\/} well}
\enumsentence{ modifying a PP:\\ {\em right\/} through the wall}
\enumsentence{modifying a determiner: \\He has {\em hardly\/} any
friends\\ {\em over\/} two hundred deaths}

We have separate trees for each of the modified categories and for
pre- and post-modification where needed.  The kind of treatment we
give to adverbs here is very much in line with the base-generation
approach proposed by \cite{Ernst84}, which assumes all positions where
an adverb can occur to be base-generated, and that the semantics of
the adverb specifies a range of possible positions occupied by each
adverb. While the relevant semantic features of the adverbs are not
currently implemented, implementation of semantic features is
scheduled for future work.  The trees for adverb anchored modifiers
are very similar in form to the adjective anchored modifier trees.
Examples of two of the basic adverb modifier trees are shown in
\ref{adv-trees}.

\begin{figure}[ht]
\centering
\begin{tabular}{cc}
{\psfig{figure=/mnt/linc/extra/xtag/work/doc/tech-rept/ps/cant-get-to-dir-stuff/betaARBs.ps,height=4in}}&
{\psfig{figure=/mnt/linc/extra/xtag/work/doc/tech-rept/ps/cant-get-to-dir-stuff/betavxARB.ps,height=4in}}\\
$\beta$ARBs&$\beta$vxARB\\
(a)&(b)
\end{tabular}
\caption {Adverb Trees}
\label {prep-trees}
\end{figure}

Like the adjective anchored trees, these trees also have the NA
constraint on the footnode to restrict the number of derivations
produced for a sequence of adverbs.  Features of the modified category
are passed from the footnode to the rootnode, reflecting correctly
that these types of properties are unaffected by the adjunction of an
adverb.  A summary of the categories modified and the position of
adverbs is given in Table \ref{adv-summary}.

\begin{table}
\centering
\begin{tabular}{|c||c|c|}
\hline
&\multicolumn{2}{c|}{Position with respect to item modified}\\
\cline{2-3}
Category Modified&Pre&Post\\
\hline
\hline
S&$\beta$ARBs&$\beta$sARB\\
\hline
VP&$\beta$ARBvx&$\beta$vxARB\\
\hline
A&$\beta$ARBa&$\beta$aARB\\
\hline
PP&$\beta$ARBpx&$\beta$pxARB\\
\hline
ADV&$\beta$ARBarb&$\beta$arbARB\\
\hline
N&$\beta$ARBn&\\
\hline
Det&$\beta$ARBdx&\\
\hline
\end{tabular}
\caption{Simple Adverb Anchored Modifiers}
\label{adv-summary}
\end{table}

The one tree in the grammar that has a complex adverb phrase,
$\beta$ARBarbs, is not included in Table \ref{adv-summary}. This tree
shown in Figure \ref{weird-adv-tree} and is used for two adverb
phrases that occur sentence initially such as in (\ex{1}).

\begin{figure}[ht]
\centering
{\psfig{figure=/mnt/linc/extra/xtag/work/doc/tech-rept/ps/cant-get-to-dir-stuff/betaARBarbs.ps,height=4in}}
\caption {Complex adverb phrase modifier}
\label {prep-trees}
\end{figure}

\enumsentence{How quickly the time passed.}

In the English FBLTAG grammar, no traces are posited for wh-adverbs in
line with the base-generation approach \cite{Ernst84} for various
positions of adverbs. Since convincing arguments have been made
against traces for adjuncts of other types (e.g. \cite{Baltin}), and
since reasons for wanting traces do not seem to apply to adjuncts, we
make the general assumption in our grammar that adjuncts do not leave
traces.  Sentence initial wh-adverbs select the same auxiliary tree
used for other sentence initial adverbs ($\beta$adS) with the feature
<{\bf +wh}> added on the {\em ad\/} anchor node.

Under this treatment, the derived tree of the sentence in (\ex{1}) is
as in \ref{how-did-you-fall}, with no trace.

\enumsentence{\em How did you fall?\/}


\begin{figure}[ht]
\centering
{\psfig{figure=/mnt/linc/extra/xtag/work/doc/tech-rept/ps/cant-get-to-dir-stuff/how-did-you-fall.ps,height=4.5in}}
\caption {Derived Tree for {\it How did you fall?}}
\label {how-did-you-fall}
\end{figure}









